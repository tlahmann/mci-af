\chapter{Anhang 1}

\begin{table*}
	\caption{Numerische Auflistung der Ergebnisse der Frage "`Please select your gender"'.}~\label{tab:sc_results_gender}
	
	\setlength\tabcolsep{3pt}
	\renewcommand{\arraystretch}{1.4}% for the vertical padding
	\begin{tabularx}{\textwidth}{ | x || r | r | }
		\hline
		Geschlecht & Absolutwerte 	& Prozentwerte \\ \hline\hline
		Männlich & 33 & 73.3\% \\ \hline
		Weiblich & 12 & 26.7\% \\ \hline
		Divers & 0 & 0.0\% \\ \hline
	\end{tabularx}
\end{table*}

\begin{table*}
	\caption{Numerische Auflistung der Ergebnisse der Frage "`Please enter your age in years"'.}~\label{tab:sc_results_age}
	
	\setlength\tabcolsep{3pt}
	\renewcommand{\arraystretch}{1.4}% for the vertical padding
	\begin{tabularx}{\textwidth}{ | x | x | x | x | x | x | }
		\hline
		Min & Max & Range & Median & Mean  & Standard Deviation \\ \hline\hline
		19  & 30  & 11    & 23     & 23.04 & 2.53              \\ \hline
	\end{tabularx}
\end{table*}

\begin{table*}
	\caption{Verteilung der Antworten zur Frage "`How much experience do you have with VR?"'.}~\label{tab:sc_results_expVR}
	
	\setlength\tabcolsep{3pt}
	\renewcommand{\arraystretch}{1.4}% for the vertical padding
	\begin{tabularx}{\textwidth}{ | x || r | r | }
		\hline
		Studienfach 						& Absolutwerte 	& Prozentwerte \\ \hline\hline
		[A1] No experience at all 			& 10 			& 22.2\% \\ \hline
		[A2] Almost no experience 			& 15 			& 33.3\% \\ \hline
		[A3] Less than average experience 	& 3 			& 6.7\% \\ \hline
		[A4] Some experience 				& 10 			& 22.2\% \\ \hline
		[A5] More than average experience 	& 2 			& 4.4\% \\ \hline
		[A6] Experienced 					& 2 			& 4.4\% \\ \hline
		[A7] Very highly experienced 		& 3 			& 6.7\% \\ \hline
	\end{tabularx}
\end{table*}

\begin{table*}
	\caption{Numerische Auflistung der Ergebnisse der Frage "`How much experience do you have with VR?"'.}~\label{tab:sc_numbers_expVR}
	
	\setlength\tabcolsep{3pt}
	\renewcommand{\arraystretch}{1.4}% for the vertical padding
	\begin{tabularx}{\textwidth}{ | x | x | x | x | x | x | }
		\hline
		Min & Max & Range & Median & Mean  & Standard Deviation \\ \hline\hline
		1  & 7  & 6    & 2     & 2.93 & 1.78              \\ \hline
	\end{tabularx}
\end{table*}

\begin{table*}
	\caption{Verteilung der Antworten zur Frage "`How much experience do you have with AR?"'.}~\label{tab:sc_results_expAR}
	
	\setlength\tabcolsep{3pt}
	\renewcommand{\arraystretch}{1.4}% for the vertical padding
	\begin{tabularx}{\textwidth}{ | x || r | r | }
		\hline
		Studienfach 						& Absolutwerte 	& Prozentwerte \\ \hline\hline
		[A1] No experience at all 			& 17 			& 37.7\% \\ \hline
		[A2] Almost no experience 			& 10 			& 22.2\% \\ \hline
		[A3] Less than average experience 	& 7 			& 15.5\% \\ \hline
		[A4] Some experience 				& 8 			& 17.7\% \\ \hline
		[A5] More than average experience 	& 2 			& 4.4\% \\ \hline
		[A6] Experienced 					& 1 			& 2.2\% \\ \hline
		[A7] Very highly experienced 		& 0 			& 0.0\% \\ \hline
	\end{tabularx}
\end{table*}

\begin{table*}
	\caption{Numerische Auflistung der Ergebnisse der Frage "`How much experience do you have with AR?"'.}~\label{tab:sc_numbers_expAR}
	
	\setlength\tabcolsep{3pt}
	\renewcommand{\arraystretch}{1.4}% for the vertical padding
	\begin{tabularx}{\textwidth}{ | x | x | x | x | x | x | }
		\hline
		Min & Max & Range & Median & Mean  & Standard Deviation \\ \hline\hline
		1  & 6  & 5    & 2     & 2.36 & 1.38              \\ \hline
	\end{tabularx}
\end{table*}

\begin{table*}
	\caption{Verteilung der Antworten zur Frage "`What subject, if any, did you study or are you currently studying?"'.}~\label{tab:sc_results_study}
	
	\setlength\tabcolsep{3pt}
	\renewcommand{\arraystretch}{1.4}% for the vertical padding
	\begin{tabularx}{\textwidth}{ | x || r | r | }
		\hline
		Studienfach & Absolutwerte & Prozentwerte \\ \hline\hline
		Biologie & 1 & 2.2\% \\ \hline
		Informatik & 8 & 17.8\% \\ \hline
		Informationssystemtechnik & 1 & 2.2\% \\ \hline
		Mathematik & 1 & 2.2\% \\ \hline
		Medieninformatik & 18 & 40.0\% \\ \hline
		Physik & 3 & 6.7\% \\ \hline
		Psychologie & 2 & 4.4\% \\ \hline
		Software Engineering & 8 & 17.8\% \\ \hline
		Wirtschaftsmathematik & 1 & 2.2\% \\ \hline
		Wirtschaftsphysik & 2 & 4.4\% \\ \hline
	\end{tabularx}
\end{table*}

\begin{table*}
	\caption{Verteilung der Einstellungen des Stuhls.}~\label{tab:sc_results_chair}
	
	\setlength\tabcolsep{3pt}
	\renewcommand{\arraystretch}{1.4}% for the vertical padding
	\begin{tabularx}{\textwidth}{ | x || r | r | }
		\hline
		Winkeleinstellungen	in Grad	& Absolutwerte 	& Prozentwerte \\ \hline\hline
		0 							& 7 			& 15.6\% \\ \hline
		30 							& 23			& 51.1\% \\ \hline
		60	 						& 10 			& 22.2\% \\ \hline
		90							& 5 			& 11.1\% \\ \hline
	\end{tabularx}
\end{table*}

\begin{itemize}
	\captionof{anno}{Anmerkungen und Hinweise von Studienteilnehmern}
	\item "`Die Musik war sehr störend, um in einen Ruhezustand zu kommen"'
	\item "`Die VR Umgebung war schön gestaltet, aber die rumschwebenden Partikel waren eher verwirrend, ich dachte ich kann mit diesen interagieren"'
	\item "`Der Stuhl war sehr entspannend und bequem"'
	\item "`Es fiel mir schwer einzuschlafen, da ich zum 1. mal VR gemacht habe und dann neugierig war"'
	\item "`Die Musik war sehr angenehm"'
	\item "`Das lange gedrückt halten zur Interaktion war störend"'
	\item "`haptisches Feedback durch Controller wäre gut gewesen"'
	\item "`Die Brille war sehr unangenehm"'
	\item "`Der Ton fürs Wecken hat mich erschrocken"'
	\item "`Mit meiner Brille war es unangenehm die VR Brille zu tragen"'
	\item "`Ich konnte mich sehr gut entspannen, richtig eingeschlafen bin ich      aber nicht"'
	\item "`Die Interaktion mit dem Controller war sehr intuitiv"'
	\item "`Mir kam die Zeit zum entspannen deutlich länger als 15 Minuten vor "'
	\item "`Egal wie ich die Brille verstellte, richtig scharf konnte ich nie sehen "'
	\item "`Noch fünf bis zehn Minuten länger und ich wäre komplett eingeschlafen "'

\end{itemize}

\begin{table*}
	\caption{Wahrgenommene Schlafdauer.}~\label{tab:sleepduration}
	
	\setlength\tabcolsep{3pt}
	\renewcommand{\arraystretch}{1.4}% for the vertical padding
	\begin{tabularx}{\textwidth}{ | x || r | r | }
		\hline
		wahrgenommene Schlafdauer in min & Absolutwerte & Prozentwerte \\ \hline\hline
		8						   	     & 2			   & 4.4\% \\ \hline
		10   					         & 5			   & 11.1\% \\ \hline
		11						   	     & 1 		   & 2.2\% \\ \hline
		12						   	     & 3			   & 6.7\% \\ \hline
		13							     & 2			   & 4.4\% \\ \hline
		14							     & 1			   & 2.2\% \\ \hline
		10-15	      					 & 3		 & 6.7\% \\ \hline
		15							     & 13		 & 28.9\% \\ \hline
		15-20							 & 1		 & 2.2\% \\ \hline
		17								 & 2		 & 4.4\% \\ \hline
		18								 & 3		 & 6.7\% \\ \hline
		18,5							 & 1		 & 2.2\% \\ \hline
		19								 & 1		 & 2.2\% \\ \hline
		20								 & 6		 & 13.3\% \\ \hline
		30								 & 1		 & 2.2\% \\ \hline
	\end{tabularx}
\end{table*}
\todoAll{Als Idee: Diese Tabelle etwas kürzen und zusammenfassen. Also z.B. Zeitintervalle [5-10), [10-15), [15], (15-20], (20-25], (25-30]. Dennis Fragen!}

\begin{table*}
	\caption{Verteilung der Antworten zur Frage "`Hast du geschlafen?"' .}~\label{tab:sleepstatus}
	
	\setlength\tabcolsep{3pt}
	\renewcommand{\arraystretch}{1.4}% for the vertical padding
	\begin{tabularx}{\textwidth}{ | x || r | r | }
		\hline
		Schlafmodus					& Absolutwerte 	& Prozentwerte \\ \hline\hline
		geschlafen 					& 9 			& 20.0\% \\ \hline
		gedöst/kurz vor eingeschlafen	& 12			& 26.7\% \\ \hline
		meditiert					& 3			& 6.7\% \\ \hline
		nicht geschlafen			& 21 			& 46.7\% \\ \hline
	\end{tabularx}
\end{table*}
