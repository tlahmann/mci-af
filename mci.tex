\documentclass[a4paper, 11pt]{article}
\usepackage{fullpage} % changes the margin
\usepackage{url} % enable urls in document

\begin{document}
\noindent
\large\textbf{Anwendungsfach Mensch Computer Interaktion} \hfill Betreuer: Dennis Wolf \\
\normalsize Böhm, Sabrina; Porta, Luca; Lahmann, Tobias \hfill Date: 26.11.2018 \\

% \section*{Abstract}
% Hier sollte ein Abstract stehen, vielleicht reichen die folgenden Sections aber auch schon ohne einen 'formellen' Abstract.

\section*{Problemstellung}
In unterschiedlichsten Fällen können Nutzer auf die Erledigung einer Aufgabe in einer virtuellen Umgebung nur schwer vorbereitet werden, wenn diese  plötzlich oder ohne Überleitung gestellt wird. So können Nutzer einer virtuellen oder augmentierten Realität (VR/AR) ... hier mehr Text einfügen.

\section*{Related Work}
\cite{yumiko2017VisAttention, seppelt2017attend, yeh2001cueReliability, bonanni2005attention, tonnis2005attention, green1995hazard}

\section*{Lösungsansatz}


\section*{Implementierung}
HTC Vive\footnote{~HTC Vive~\url{https://www.vive.com}}

Auf Softwareseite soll eine digitale Umgebung mittels Unity 3D\footnote{~Unity3D~\url{https://unity3d.com}} erstellt werden.

\section*{Testaufbau}

\section*{Zeitplan}
Die Daten beziehen sich auf den Zeitpunkt zu dem der jeweilige Schritt abgeschlossen sein sollte.
\begin{itemize}
    \item \textbf{Dezember 2018} Recherche bestehender Forschung
    \item \textbf{Dezember 2018} Digitale Umgebung in Unity3D erstellen und mit erster Testphase validieren
    \item \textbf{Dezember 2018} Erste Studie durchführen
    \item \textbf{Januar 2019} Auswertung der durchgeführten Studie.
    \item \textbf{März 2019} Entwurf der zweiten Studie
    \item \textbf{April 2019} Erweiterte Umgebung in Unity3D erstellen und für den zweiten Studiendurchlauf vorbereiten
    \item \textbf{Mai 2019} Durchführung der zweiten Studie für erweiterte Ergebnisse
    \item \textbf{Juni 2019} Auswertung der Ergebnisse von zweiter Studie
    \item \textbf{September 2019} Dokumentation der Ergebnisse und des Vorgehens
\end{itemize}

\begin{thebibliography}{9}
\bibitem{yumiko2017VisAttention} Shinohara, Yumiko, et al. "Visual Attention During Simulated Autonomous Driving in the US and Japan." Proceedings of the 9th International Conference on Automotive User Interfaces and Interactive Vehicular Applications. ACM, 2017.
\bibitem{seppelt2017attend} Seppelt, Bobbie, et al. "Differentiating Cognitive Load Using a Modified Version of AttenD." Proceedings of the 9th International Conference on Automotive User Interfaces and Interactive Vehicular Applications. ACM, 2017.
\bibitem{yeh2001cueReliability} Yeh, Michelle, and Christopher D. Wickens. "Display signaling in augmented reality: Effects of cue reliability and image realism on attention allocation and trust calibration." Human Factors 43.3 (2001): 355-365.
\bibitem{bonanni2005attention} Bonanni, Leonardo, Chia-Hsun Lee, and Ted Selker. "Attention-based design of augmented reality interfaces." CHI'05 extended abstracts on Human factors in computing systems. ACM, 2005.
\bibitem{tonnis2005attention} Tonnis, Marcus, et al. "Experimental evaluation of an augmented reality visualization for directing a car driver's attention." Mixed and Augmented Reality, 2005. Proceedings. Fourth IEEE and ACM International Symposium on. IEEE, 2005.
\bibitem{green1995hazard} Green, Paul. "A driver interface for a road hazard warning system: Development and preliminary evaluation." Proceedings of the Second World Congress on Intelligent Transportation Systems. Vol. 4. 1995.
\end{thebibliography}

\end{document}
