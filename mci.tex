\documentclass[a4paper, 11pt]{article}
\usepackage{fullpage} % changes the margin
\usepackage{url} % enable urls in document
\usepackage{ntheorem}
\theoremseparator{:}
\newtheorem{hyp}{Hypothese}

\newcommand{\projectName}{mci-af}

\begin{document}
\begin{center}
	\textbf{\LARGE{\projectName}}\\
    \textbf{\large{Anwendungsfach Mensch Computer Interaktion}}\\
	\vspace{7mm}
    \textbf{\large{Böhm, Sabrina; Porta, Luca; Lahmann, Tobias}}\\
	\textbf{\large{Betreuer: Dennis Wolf}}\\
	\today
\end{center}

\section*{Abstract}
Die Untersuchung der Heranführung einer Person an ein Problem, wenn diese im vorfeld keine Informationen über die Aufgabe hat. % Introduction. In one sentence, what’s the topic?
Wir untersuchen die unterschiedlichen Arten jemanden aus einer abgelenkten Situation in eine Bewusste zu überführen und wie diese Person in diesem Vorgang unterstützt werden kann. % State the problem you tackle
Vorbereitungen auf Aufgaben und noch spezieller die Lenkung der Aufmerksamkeit wurde noch ungenügend untersucht. % Summarize (in one sentence) why nobody else has adequately answered the research question yet
Wir untersuchen die Problemstellung im kontext von VR und nutzen dies um dadurch unterschiedliche Parameter des 'Aufweckens' sowie Designprinzipien zu untersuchen. % Explain, in one sentence, how you tackled the research question
In einer between-subject Studie wurden die Unterschiede der möglichen Designs untersucht. % In one sentence, how did you go about doing the research that follows from your big idea
Aktuell können unsere Vermutungen noch nicht bestätigt werden. % As a single sentence, what’s the key impact of your research?

\section*{Problemstellung}
In unterschiedlichsten Fällen können Nutzer auf die Erledigung einer Aufgabe in einer virtuellen Umgebung nur schwer vorbereitet werden, wenn diese  plötzlich oder ohne Überleitung gestellt wird. So können Nutzern einer virtuellen oder augmentierten Realität (VR/AR) beim Wechsel der Umgebung, oder beim Wechsel in die digitale Umgebung, Informationen fehlen, welche notwendig sind um sich schnell, zuverlässig und ohne potenzielle Fehlerquellen an diese zu gewöhnen. 

Das Projekt "\projectName" soll es einem Nutzer ermöglichen alle relevanten Informationen innerhalb kürzester Zeit aufzunehmen. Des Weiteren soll untersucht werden auf welche Art und Weise dieser Vorgang zuverlässig durchgeführt werden kann.

Wir stellen grundlegend die Hypothese~\ref{hyp:aufwachen} auf:
\begin{hyp}[H\ref{hyp:aufwachen}]\label{hyp:aufwachen}
	Menschen die langsam geweckt werden können sich schneller auf eine gestelle Aufgabe einstellen, als Menschen, die schnell geweckt werden.
\end{hyp}

\section*{Related Work}
Räumliche, aufmerksamkeitssensitive Darstellungen sind effektiv~\cite{bonanni2005attention}. Exogene Hinweise können dem Nutzer helfen sich auch in bekannten Umgebungen zurechtzufinden~\cite{bonanni2005attention}.

Es existieren unterschiedliche Herangehensweisen um Fahrer in Autos über eine auftretende Gefahrensituation zu informieren. Hierbei wurden textuelle Informationen den grafischen vorgezogen.~\cite{green1995hazard}

Kulturelle Unterschiede bewirken, dass sich Fahrer im Straßenverkehr auf unterschiedliche Dinge konzentrieren und im Anschluss an unterschiedliche Details erinnern~\cite{yumiko2017VisAttention}.

Zur Vorbereitung auf die Objekte oder Vorgänge in der Umgebung von Menschen können 3D Marker verwendet werden, die in die Richtung des Objekts oder Geschehens weisen. Eine 3D Darstellung ist nach Chittaro und Burigat mindestens genauso effektiv wie eine 2D darstellung. Sie bietet jedoch den Vorteil, dass Nutzer auch in der dritten Dimension, der Höhe, auf wichtige Punkte hingewiesen werden können~\cite{chittaro20043d}.

\section*{Lösungsansatz}
In einer virtuellen Umgebung werden Probanden dazu aufgefordert sich zu entspannen und, nach Möglichkeit, ohne Ablenkung zu verweilen. Im Anschluss werden die Nutzer auf unterschiedliche Arten aus dieser Ruhephase begleitet und ihnen eine Aufgabe gestellt. 

Eine Gruppe der Nutzer wird 'sanft geweckt' indem die virtuelle Umgebung langsam und bedacht erhellt wird, zudem kann dieser Vorgang mit unterschiedlichen anderen Sinnen unterstützt werden. Die Aufgabenstellung erscheint iterativ und wird mit jedem Schritt Anspruchsvoller, bis die volle Schwierigkeit erreicht wurde. 
Die Andere Gruppe der Nutzer wird hingegen 'schnell geweckt'. Hierbei wird die Helligkeit der Virtuellen Umgebung plötzlich erhöht. Auch hier können unterschiedliche Sinne angesprochen werden um möglicherweise den Stressfaktor beim Probanden zu erhöhen. Die Nutzer werden direkt nach dem Aufwachen mit der vollen Schwierigkeit einer Aufgabe konfrontiert und müssen diese lösen.

% Um unterscheidungen anzustellen wie 

Im nächsten Schritt des Projekts soll untersucht werden auf welche Arten die Nutzer der beiden Gruppen in dem Erledigen von Aufgaben unterstützt werden können. Hierbei werden vor allem gestalterische sowie aufmerksamkeitssteuernde Aspekte untersucht. 

Alle Nutzer der virtuellen Umgebung können hier dahingehend untersucht werden, mit welchen Fehlerraten sowie welcher Geschwindigkeit die gestellten Aufgaben erledigt werden. Nachfolgend können standardisierte Fragebögen herangezogen werden um das Befinden und die eigene Einschätzung sowie die Zuversichtlichkeit des Nutzers zu untersuchen.

\section*{Implementierung}
Auf Softwareseite soll eine digitale Umgebung mittels Unity 3D\footnote{~Unity3D~\url{https://unity3d.com}} erstellt werden. Das gewählte Interface zur Untersuchung der beschriebenen Problemstellung ist die HTC Vive\footnote{~HTC Vive~\url{https://www.vive.com}} mit den zugehörigen Controllern. Die Eingabemethoden zur Lösung der gestellten Aufgaben werden mit Zeigeoperationen realisiert. Bewegung innerhalb der digitalen Umgebung ist, bis auf Kopfbewegungen, nicht vorgesehen. Eine Implementierung zum nachverfolgen der Augenbewegung von Probanden wird nicht vollständig übernommen, sondern auf bestehende Realisierungen zurückberufen.

Fragebögen werden über das Limesurvey online Fragebogen Tool des Instituts für Medieninformatik gestellt und beantwortet. 

\subsubsection*{Aufwachen}
Probanden können auf unterschiedliche Arten aufgeweckt werden
\begin{itemize}
    \item \textbf{Töne} Alarm-Sounds oder Musik~[cite needed]
    \item \textbf{Licht} Helle Umgebungen wecken Personen auf~[cite needed]
    \item \textbf{Temperatur} Eine kalte Umgebung verursacht erhöhte Aufmerksamkeit~[cite needed]
    \item \textbf{Bewegung} Jemand der geschüttelt wird wacht ebenfalls auf~[cite needed]
\end{itemize}

\section*{Testaufbau}
Sitzend werden Probanden erst in einen entspannten Zustand versetzt. In diesem sollen sie möglichst ohne Ablenkung verweilen bis sich eine Gelassenheit einstellt. Diese kann von entspanntem Sitzen bis hin zum Schlaf führen, eine genaue Zeitspanne hierfür kann zwischen Probanden variieren und muss in Tests bestimmt werden.

Nachfolgend wir der Teilnehmer aus diesem Zustand geholt und mit einer Aufgabenstellung konfrontiert. Während der Erledigung der Aufgabe werden unterschiedliche Parameter (Blickrichtung, Zeit und Fehlerrate der Erledigung der Aufgabe) aufgezeichnet und später ausgewertet.

\section*{Zeitplan}
Die Daten beziehen sich auf den Zeitpunkt zu dem der jeweilige Schritt abgeschlossen sein sollte.
\begin{itemize}
    \item \textbf{Dezember 2018} Recherche bestehender Forschung
    \item \textbf{Dezember 2018} Digitale Umgebung in Unity3D erstellen und mit erster Testphase validieren
    \item \textbf{Dezember 2018} Erste Studie durchführen
    \item \textbf{Januar 2019} Auswertung der durchgeführten Studie.
    \item \textbf{März 2019} Entwurf der zweiten Studie
    \item \textbf{April 2019} Erweiterte Umgebung in Unity3D erstellen und für den zweiten Studiendurchlauf vorbereiten
    \item \textbf{Mai 2019} Durchführung der zweiten Studie für erweiterte Ergebnisse
    \item \textbf{Juni 2019} Auswertung der Ergebnisse von zweiter Studie
    \item \textbf{September 2019} Dokumentation der Ergebnisse und des Vorgehens
\end{itemize}

\begin{thebibliography}{9}
\bibitem{yumiko2017VisAttention} Shinohara, Yumiko, et al. "Visual Attention During Simulated Autonomous Driving in the US and Japan." Proceedings of the 9th International Conference on Automotive User Interfaces and Interactive Vehicular Applications. ACM, 2017.
\bibitem{bonanni2005attention} Bonanni, Leonardo, Chia-Hsun Lee, and Ted Selker. "Attention-based design of augmented reality interfaces." CHI'05 extended abstracts on Human factors in computing systems. ACM, 2005.
\bibitem{green1995hazard} Green, Paul. "A driver interface for a road hazard warning system: Development and preliminary evaluation." Proceedings of the Second World Congress on Intelligent Transportation Systems. Vol. 4. 1995.
\bibitem{chittaro20043d} Chittaro, Luca, and Stefano Burigat. "3D location-pointing as a navigation aid in Virtual Environments." Proceedings of the working conference on Advanced visual interfaces. ACM, 2004.
\bibitem{wilkinson1971performance} Wilkinson, Robert T., and M. Stretton. "Performance after awakening at different times of night." Psychonomic Science 23.4 (1971): 283-285.
\end{thebibliography}

\end{document}
