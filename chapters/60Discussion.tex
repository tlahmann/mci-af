\chapter{Diskussion}
% -- hier ganz gute Beispiele zum Formulieren \url{https://www.scribbr.de/aufbau-und-gliederung/diskussion-beispiel/}
% --\\

Im Rahmen der Studie des Projekts \projectName \,haben die Ergebnisse unserer Untersuchung gezeigt, dass die Art und Weise des Aufwachens die anschließende Aufgabenbewältigung nicht signifikant beeinflusst.
Anhand der Ergebnisse und deren Interpretation gehen wir aber davon aus, dass weitere Untersuchungen durchaus signifikante Unterscheide hervorbringen könnten.

\section{Allgemeines}\todoAll{Anders nennen...}

Die Studienteilnehmer waren zu über 70\% männlich, was sich auf den Anteil der männlichen und weiblichen Studierenden in den Studiengängen zurückführen lässt. Diese Verteilung zeigt allerdings keine statistisch signifikanten Unterschiede zwischen den männlichen und weiblichen Teilnehmern.
Durch die Durchführung der Studie an der Uni haben wir eine relativ kleine Bandbreite von Teilnehmern abgedeckt. Dies betrifft sowohl den Studiengang, wie auch das Alter der Teilnehmer. Außerdem weisen alle Teilnehmer dadurch einen höheren Bildungsstand auf. 
Um den Bogen zum Durchschnitt der Gesellschaft zu spannen wäre es interessant die Studie mit mehr Teilnehmern zu wiederholen um Faktoren, wie Alter und Bildung besser vergleichen zu können.
Außerdem ist ein interessanter Faktor, der in dieser Studie nicht untersucht werden konnte, die Erfahrung mit Medien und Geräten im allgemeinen. 
Durch die moderne Gesellschaft werden Nutzer heute mit neuen Technologien überschüttet. Außerdem haben Viele standardgemäß ein Smartphone, wodurch die Erfahrung mit Spielen, Sozialen Netzwerken und technischen Vorgängen weiter gefördert wird.

Auf der anderen Seite haben viele der Teilnehmer möglicherweise noch keine oder nur wenig Erfahrung mit dem Autofahren allgemein. Wir gehen außerdem davon aus, dass die Teilnehmer keine oder nur sehr wenig Erfahrung mit autonomen Fahrzeugen gemacht haben. Durch diese beiden Punkte argumentieren wir ebenfalls, dass die Studie in Zukunft, sobald Autonomes Fahren etabliert ist, mit einer größeren Anzahl an Probanden andere Ergebnisse liefern könnte. 

Da VR und AR zur heutigen Zeit noch kein etablierter Bestandteil des Alltags sind, war die Altersgruppe allerdings auch nicht nachteilig, da manche Probanden explizit selbst auch mit VR und AR an der Universität arbeiten und sich so gut in unseren Studienkontext einfühlen konnten.
Zusätzlich erleichterte uns gerade die Erfahrung der Teilnehmer mit Technik eine relativ Sorgenfreie Studiendurchführung. Dies ist der Tatsache geschuldet, dass die Teilnehmer weniger Probleme im Umgang mit der von uns gewählten Technik hatten und sich schnell mit unserer Studienumgebung und der Handhabung vertraut gemacht haben.

Die AR/VR Erfahrung der Teilnehmer spielte für uns keine gesonderte Rolle, wir fassten diesen Punkt jedoch mit auf und stellen fest, dass unsere Probanden sehr unterschiedliche Erfahrungen hatten, wie in Abbildung~\ref{fig:expVr} und~\ref{fig:expAr} gesehen werden kann. Es ist zu bemerken, dass die Erfahrung mit AR hierbei allerdings noch geringer war, was eine Nutzung von Head-Up Displays in Autos als weiteren spannenden Untersuchungspunkt für zukünftige Projekte hervorhebt. 
Bei der Durchführung der Studie war auffällig, dass die Probanden, die zum ersten Mal VR nutzten sehr neugierig waren, viel mit dem Equipment herumgespielt haben und nur schwer zur Ruhe kamen.

\section{Studiendurchführung und Studienumgebung}

Auch mit bereits gegebener VR-Erfahrung war für die Interaktion eine kurze Lernphase erforderlich.
Unsere Beobachtungen zeigten aber, dass erfahrenen Nutzer die Interaktionen automatisch ausprobierten und schnell verstanden. 
Da wir als Interaktionsmöglichkeit ein verweilen der Pointer-Funktionalität auf Interaktionsobjekten gewählt haben ergab sich hier auch für VR-Erfahrene eine kleine Umstellung. Wir haben diese Funktionalität gewählt um ein "`Auffüllen"' zu simulieren. Der Nutzer sollte also mittels des Controllers die Objekte anregen und so auswählen.
Hierbei dauerte es eine halbe Sekunde (500ms) vom Start der Interaktion bis das Objekt vom System als gewählt registriert wurde. Wir versuchten mit dieser Vorgehensweise die Möglichkeit offen zu lassen, dass Nutzer sich umentscheiden können, falls sie eine initial getroffene Entscheidung nicht mehr für die richtige halten. 
Unsere Ergebnisse zeigen, dass dies allerdings nicht vorgekommen ist, wodurch der Zeitraum entweder verkürzt oder gänzlich entfernt werden könnte. 
Außerdem teilten uns Teilnehmer mit, dass haptisches Feedback über den Controller für den Nutzer noch hilfreich gewesen wäre. Dies resultiert aus der Gewohnheit der Nutzer aus anderen VR Untersuchungen oder Spielen, sowie aus der Annahme, dass das Interagieren mit Objekten auch in der realen Welt ein haptisches Feedback liefert.

Auf Grund der kurzen Ruhephase konnte man nicht davon ausgehen, dass alle Teilnehmer einschlafen beziehungsweise zur Ruhe kommen. So zeigt auch die Tabelle~\ref{tab:sleepstatus}, dass die Teilnehmer mit einer deutlichen Tendenz nicht geschlafen haben. 
Ein Grund hierfür könnte sein, dass es für manche Menschen schwer ist in einer unbekannten Umgebung und vor anderen Personen einzuschlafen. Unsere Ergebnisse zeigen aber auch, dass die Nutzer durchaus zur ruhe gekommen sind, mit einem Anteil von 20\% der schlafenden und 26,7\% der dösenden Probanden. 
Ein anderer Grund hierfür könnte auch die Studienumgebung sein, denn in machen Fällen waren Umgebungsgeräusche, wie das Klacken von Computertastaturen, Musik oder auch Gespräche aus den Nebenräumen wahrnehmbar. Wir haben dies bereits in der Implementierungsphase der VR-Umgebung bemerkt und uns daher entschieden die Nutzer mit einem Headset auszustatten und über dieses Musik abzuspielen.
Für manche Teilnehmer war die Wahl der Musik störend und verhinderte auch das komplette Entspannen beziehungsweise Einschlafen. 
Andere empfanden die Musik als entspannend und wohltuend. 
Hier könnte man eventuell spezifischer auf die Teilnehmer eingehen und jeweils eine eigene "`Schlafmusik"' wählen lassen, beziehungsweise sogar gänzlich weglassen.

Uns wurde nach der Studie oft mitgeteilt, dass der Stuhl sehr bequem sei sich aber nicht zum Schlafen mit einer VR Brille auf dem Kopf eignet. 
Dies war oft der Fall, wenn Probanden die Halteelemente der HMD zwischen dem Kopf und dem Stuhl hatten und dadurch eigentlich auf der VR-Brille lagen und den Kopf möglicherweise nicht bequem ablegen konnten. 
Nach unserer Einschätzung könnte hier die Weiterentwicklung der Geräte hilfreich sein. Kleinere Geräte, mit noch höherem Tragekomfort müssen hierfür allerdings erst noch entwickelt werden.

Dem Entspannen hinderlich waren bei manchen Probanden auch äußere Einflüsse wie Kaffee oder Aufregung, da das Ziel der Studie im Voraus nicht mitgeteilt wurde. Das Einschlafvermögen war demnach wie erwartet sehr personenabhängig.
Bei Teilnehmern die zum ersten mal mit VR in Kontakt getreten sind wurde deutlich, dass die Faszination und die Neugier in den Vordergrund getreten sind und so im Vergleich zu den VR-Erfahrenen weniger schnell Entspannung eingetreten ist. 
Hier wäre eine längere Instruktionsphase eventuell vorteilhaft, um den Probanden ein gutes Gefühl im Umgang mit VR zu ermöglichen. 
Dazu wäre während der Studiendurchführung Zeit gewesen, allerdings haben wir die Nutzer nicht explizit dazu ermutigt sich mit allem Vertraut zu machen. Hier wäre ein besseres Design der VR-Umgebung diesem Problem entgegengekommen.
Uns wurde öfter mitgeteilt, dass mit einer längeren Ruhephase das Einschlafen eingetreten wäre. Dies haben wir bei der Planung unserer Studie allerdings verworfen.~\todoAll{Hier ein Grund ausdenken, warum 15 minuten gut gewält ist (Zeit und Geld sind keine guten Gründe...)}
Manche Nutzer haben uns auch mitgeteilt, dass sie auch nach noch längerer Ruhephase auf dem Stuhl und mit der Brille nicht schlafen hätten können, was wiederum auf die Bequemlichkeit der Brille slebst zurückzuführen ist.

Die Urzeit in der wir die Studie durchgeführt haben, spannte sich von 10 Uhr morgens bis 18 Uhr abends. Hier wäre ein kleinerer Zeitraum ein Faktor, der ein besser vergleichbares Ergebnis liefert. 
Allerdings sollte erwähnt werden, dass zu jeder Tageszeit Teilnehmer existierten, die eingeschlafen sind und Teilnehmer denen es nach eigenen Angaben zu jeder Uhrzeit schwer gefallen wäre.

\section{Aufgaben}

\todoAll{Warum gab es nicht so viele Fehler? (Oder so viele Fehler?) im Bezug auch zur Zeit sehen}
Die Anzahl der Fehler hängt stark von der Bearbeitungszeit ab (vermute ich -> nachschauen!?). Um so länger die Teilnehmer für eine Aufgabe gebraucht haben, desto besser waren die Ergebnisse. So machten Nutzer, die die Zahlordnungsaufgabe langsamer gemacht haben, weniger Fehler. Die Farbenaufgabe wurde meist komplett richtig oder komplett falsch gemacht, da die Nutzer teilweise die Aufgabe genau andersherum verstanden hatten und wählten so nicht die Farbe, die die Schrift anzeigte sondern die Farbe in der die Schrift dargestellt war.

\todoAll{Sind die Aufgaben nicht passend für die Studie?}
Manche Probanden sagten im Anschluss an die Studie, dass sie die Aufgaben zu einfach fanden. Wir wählten jedoch bewusst einfache Aufgaben, die explizites Können abfragen, wie zB. das Räumliche Denken. Auffällig war auch, dass viele Teilnehmer wie erwartet kleine Fehler machten, die sie nach eigener Einschätzung im komplett Wachen Zustand nicht machen würden. 

\todoAll{Hätte man mit der Aufweckdauer spielen können (länger/kürzer Licht)}
\todoAll{Welche Töne sind effektiver als andere?}
\todoAll{Ist eine Kombination der Parameter sinnvoll?}
\todoAll{Wie viele Aufgaben hätten es sein können?}
\todoAll{Warum haben wir nicht die Aufgaben alle gleichzeitig gestartet? Also quasi in einem Halbkreis alle nebeneinander}

%Implementierung etc
\todoAll{Was sind die Probleme bei der Implementierung?}

%Fragebogen, Zukunfsaussicht etc
\todoAll{SAM ergebnisse auswerten}
\todoAll{Ist VR die richtige Herangehensweise?}
Da VR und AR immer mehr in das Alltagsleben in Zukunft integriert werden sollen, ist es ein interessantes Thema Alltagssituationen wie Schlafen oder gewisse Aufgaben zu erledigen im VR/AR Kontext zu untersuchen. Bei manchen Probanden haben wir gehört, dass sie es sich gut vorstellen könnten eine AR Brille dauerhaft zu tragen, falls dies sich von der Größe und vom Gewicht nicht mehr von normalen Sehstärkebrille unterscheidet. Zu dem jetzigen Zeitpunkt ist dies jedoch noch nicht möglich.
\todoAll{Wann Sollte ein Mensch beim Autofahren geweckt werden? Das vielleicht in schlussfolgerung rein?}
Im Kontext des autonomen Fahrens sollte man beim Aufwecken des Fahrers auf jeden Fall einen gewissen Abstand an Zeit bis hin zum Eingreifen einplanen, da deutlich wurde, dass der Mensch erst vorbereitet werden muss. Durch zahlreiche Studien wurde dieses veränderte Verhalten direkt nach dem Aufwachen auch schon belegt.

\todoAll{weitere ERGEBNISSE INTERPRETIEREN (alle) + eventuelle Ursachen und Folgen}
\todoAll{neue Erkenntnisse auflisten}
\todoAll{beschreiben ob Erwartungen erfüllt worden sind, eigene Feststellungen und auch Literatur}
\todoAll{Beschränkungen der Forschung darlegen: es war nicht Ziel er Arbeit blabla oder es würde den Rahmen der Arbeit sprechen blabla}
\todoAll{ist die Validität der Forschung belegt?}
\todoAll{konkrete Vorschläge für weitere Forschung wo unsre Arbeit als Orientierung dient, keine zu vagen Vorschläge}

