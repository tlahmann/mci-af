\chapter{Diskussion}

\todoAll{Warum sind die Leute nicht eingeschlafen?}
In anbetracht der kurzen Ruhephase konnte man nicht davon ausgehen, dass alle Teilnehmer einschlafen bzw. zur Ruhe kommen. Es ist im Allgemeinen auch schwer in einer fremden Umgebung und vor fremden Personen einzuschlafen. In Anbetracht dessen haben wir die Teilnehmer so gewählt, dass uns alle Teilnehmer persönlich kennen und so eine vertrauensvollere Basis zum Entspannen geschaffen. Trotz des persönlichen Kennens kann es durchaus sein, dass manche Teilnehmer trotzdem in Anwesenheit anderer in einem Raum nicht schlafen können. Teilweise konnten die Nutzer auch aus Lärmgründen vom Nebenraum nicht einschlafen. Öfter haben wir auch nach der Studie gehört, dass der Stuhl sehr bequem sei aber nicht zum Schlafen mit einer VR Brille auf dem Kopf geeignet ist. Das Entspannen kam bei manchen Probanden auch durch äußere Einflüsse wie Kaffee oder persönliches aufgeregt sein nicht zu Stande, da die Probanden vorher meist nicht wussten, dass sie gleich an einer Studie teilnehmen werden, in der sie runterfahren sollen. 
Für manche Teilnehmer war die Musik, die wir bewusst ruhig und eintönig gewählt hatten, störend und verhinderte auch das komplette Entspannen bzw. Einschlafen. Andere hingegen haben die Musik als entspannend und wohltuend empfunden. Hier könnte man eventuell personifizierter Vorgehen und jeden eine eigene 'Schlafmusik' wählen lassen, bzw. für manche wäre das Weglasssen der Musik am besten gewesen.
\todoAll{Waren die leute zu sehr abgelenkt von VR?}
Bei Teilnehmern die zum ersten mal mit VR in Kontakt getreten sind wurde deutlich, dass die Faszination und die Neugier in den Vordergrund getreten sind und so im Vergleich zu den VR-Erfahrenen eine weniger entspannte Atmosphäre zu stande kam. Hier wäre eine längere Ausprobierphase eventeuell von Vorteil gewesen, um den Probanden ein gutes Gefühl im Umgang mit VR zu ermöglichen. Dazu war jedoch nicht genügend Zeit vorhanden.
\todoAll{Waren die Uhrzeit gut gewählt?}
Die Durchführung der Studie wieß auch probandenbezogene Ungleichheiten wie 'gerade einen Kaffee getrunken' oder 'noch gar nicht richtig wach sein' auf. Die Uhrzeitspanne in der wir die Studie durchgeführt haben, hat sich zwischen 10 uhr morgens und 18 Uhr abends abgespielt. Da wir möglichst müde Probanden für unsere Studie vorgesehen hatten, wäre es sinnvoll gewesen die Probanden spät abends herzubestellen, was wiederrum ein Problem dargestellt hat, da die Menschen um diese Uhrzeit normalerweise nicht mehr an der Uni sind.
\todoAll{Ist die Altersgruppe gut gewählt?}
Wir haben großen Wert drauf gelegt eine große Bandbreite an Studiengängen und einen großen Altersabschnitt abzudecken. Dadurch, dass wir nur Studenten oder Dokotoranden für unsere Studie hernagezogen haben, haben wir jedoch nur eine Altersspanne zwischen 20 und 30 Jahren erfassen können. Da VR und AR noch nicht normaler Bestandteil des Alltaglebens sind war die Altersgruppe unsererseits gut gewählt, da manche Probanden explizit selbst auch mit VR und AR an der Universität arbeiteten und sich gut in unserer Studienkontext einfühlen konnten. Andererseits haben wir auch Probanden komplett anderer Studiengänge teilnehmen lassen, die meist ihre erste VR Erfahrung machten, so haben wir also beide Extrema abgedeckt. Noch etwas ältere Nutzer, die zwischen 30 und 50 Jahren sind, wären eine interessante weitere Gruppe, die man testen hätte können und bei denen der Unterschied zwischen erster VR Erfahrung und Erfahrenen spannend gewesen wäre. 
\todoAll{Welche Unterschiede existieren zwischen VR-Erfahrung und keine Erfahrung? (Oder sogar 'Digital Natives' und anderen Nutzern? Was wäre, wenn man das Prinzip auf meine Oma anwendet?)}
Es war auffällig, dass die Probanden, die zum ersten Mal VR nutzten sehr neugierig waren und viel mit dem Equipment herumgespielt haben und nur schwer zur Ruhe kamen. Auch für die Interaktion war zuerst ein Lernen erforderlich, wohingegen bei erfahrenen Nutzern die Interaktionen automatisch geschehen sind. Wir haben als Interaktionsmöglichkeit ein längeres (als normal zur Bedieung verwendetes) Tätigen des Triggerknopfes gewählt, was auch für die VR Erfahrenen eine Umstellung war. Dies wählten wir bewusst so aus, dass man in dieser halben Sekunde noch die Möglichkeit hatte, um sich umzusentscheiden, falls man doch eine andere Bubble auswählen wollte. Als weiteres Feedback haben wir auch die Anmerkung bekommen, dass haptisches Feedback durch die Controller angenehm gewesen wäre und nicht nur visuelles Feedback durch platzen der Bubbles.
%Aufgaben etc
\todoAll{Warum gab es nicht so viele Fehler? (Oder so viele Fehler?) im Bezug auch zur Zeit sehen}
Die Anzahl der Fehler hängt stark von der Bearbeitungszeit ab (vermute ich -> nachschauen!?). Um so länger die Teilnehmer für eine Aufgabe gebraucht haben, desto besser waren die Ergebnisse. So machten Nutzer, die die Zahlordnungsaufgabe langsamer gemacht haben, weniger Fehler. Die Farbenaufgabe wurde meist komplett richtig oder komplett falsch gemacht, da die Nutzer teilweise die Aufgabe genau andersherum verstanden hatten und wählten so nicht die Farbe, die die Schrift anzeigte sondern die Farbe in der die Schrift dargestellt war.
\todoAll{Sind die Aufgaben nicht passend für die Studie?}
Manche Probanden sagten im Anschluss an die Studie, dass sie die Aufgaben zu einfach fanden. Wir wählten jedoch bewusst einfache Aufgaben, die explizites Können abfragen, wie zB. das Räumliche Denken. Auffällig war auch, dass viele Teilnehmer wie erwartet kleine Fehler machten, die sie nach eigener Einschätzung im komplett Wachen Zustand nicht machen würden. 
\todoAll{Hätte die Ruhephase länger sein sollen?}
Öfter hörten wir den Kommentar 'Wäre noch etwas länger Zeit gewesen um einschlafen zu können, wäre ich eingeschlafen', was wir auch  vermuteten, jedoch hätte man erheblich mehr Kostenaufwand und auch personalen Aufwand gehabt, die Probanden länger schlafen bzw. ruhen zu lassen. Das Einschlafvermögen ist auch sehr personenabhängig. Manche Nutzer haben uns auch mitgeteit, dass sie auch nach noch längerer Ruhephase auf dem Stuhl und mit der Brille nicht schlafen hätten können. 
\todoAll{Hätte man mit der Aufweckdauer spielen können (länger/kürzer Licht)}
\todoAll{Welche Töne sind effektiver als andere?}
\todoAll{Ist eine Kombination der Parameter sinnvoll?}
\todoAll{Wie viele Aufgaben hätten es sein können?}
\todoAll{Warum haben wir nicht die Aufgaben alle gleichzeitig gestartet? Also quasi in einem Halbkreis alle nebeneinander}
%Implementierung etc
\todoAll{Was sind die Probleme bei der Implementierung?}
%Fragebogen, Zukunfsaussicht etc
\todoAll{Was sind die SAM ergebnisse?}
\todoAll{Welche verbesserungen könnten gemacht werden?}
\todoAll{Ist VR die richtige Herangehensweise?}
Da VR und AR immer mehr in das Alltagsleben in Zukunft integriert werden sollen, ist es ein interessantes Thema Alltagssituationen wie Schlafen oder gewisse Aufgaben zu erledigen im VR/AR Kontext zu untersuchen. Bei manchen Probanden haben wir gehört, dass sie es sich gut vorstellen könnten eine AR Brille dauerhaft zu tragen, falls dies sich von der Größe und vom Gewicht nicht mehr von normalen Sehstärkebrille unterscheidet. Zu dem jetzigen Zeitpunkt ist dies jedoch noch nicht möglich.
\todoAll{Wann Sollte ein Mensch beim Autofahren geweckt werden? Das vielleicht in schlussfolgerung rein?}
Im Kontext des autonomen Fahrens sollte man beim Aufwecken des Fahrers auf jeden Fall einen gewissen Abstand an Zeit bis hin zum Eingreifen einplanen, da deutlich wurde, dass der Mensch erst vorbereitet werden muss. Durch zahlreiche Studien wurde dieses veränderte Verhalten direkt nach dem Aufwachen auch schon belegt.

