\chapter{Einführung}
\todoLuc{Kapitel 1 komplett lesen, ggf verbessern oder korrigieren}
Virtuelle und augmentierte Umgebungen begleiten uns bereits seit einigen Jahren. Die Anwendungsgebiete sind vielfältig und reichen von der Unterstützung von Demenzpatienten~\cite{flynn2003developing, hayhurst2018augmented} über die Unterhaltungsindustrie~\cite{hughes2005mixed} bis in alltägliche Bereiche, wie das Autofahren oder Büro- und Fabrikarbeiten~\cite{medenica2011augmented, caudell1992augmented}. 
Die Head Mounted Displays (HMD), also die Geräte, welche die Schnittstelle zwischen realer und virtueller Welt bilden, sind zum jetzigen Zeitpunkt noch enorm groß und eignen sich daher wenig für den permanenten Einsatz. Technologiefirmen arbeiten daran, die Hardware klein, leicht und kompakt zu gestalten~\cite{shibata2002head, sugihara2012head} und so den täglichen Einsatz zu ermöglichen.
Aufgrund der aktuellen Einschränkungen könnten Benutzer den Einsatz der Geräte im Alltag noch ablehnen, werden jedoch in Zukunft diese Systeme kleiner und leichter, oder sogar permanent mit dem menschlichen Körper verbunden, wird es möglich sein, jeden Bereich des Lebens zu augmentieren. 
\todoTob{hier schon über das Aufwachen und Abliefern Thema labern oder erst später?}
