\chapter{Einführung}

This is the introduction

\section{Motivation}

Virtuelle und Augmentierte Umgebungen begleiten uns bereits seit einigen Jahren.  Die Produkte, welche dabei zum Einsatz kommen, sind zum jetzigen Zeitpunkt aufgrund der unpraktischen Größe nicht für das alltägliche Umfeld des Durchschnittsbürgers geeignet und angesichts der hohen Preise auch nicht sonderlich weit verbreitet.
Werden in Zukunft diese Systeme kleiner, leichter, oder sogar permanent mit dem menschlichen Körper verbunden, wird es möglich sein, jeden Bereich des Lebens zu augmentieren. AR/VR soll ein stetiger Begleiter sein und dem Nutzer auch in anspruchsvollen oder unerwarteten Situationen unter die Arme greifen. Eine solche unerwartete Situation könnte beispielsweise direkt nach dem Aufwachen aus dem Schlaf der Fall sein. So können in autonomen Autos Aufgaben vom Fahrer übernommen, im Nachtdienst eines Sicherheitsunternehmens kritische Vorgänge überwacht, oder am Morgen Herausforderungen vom Benutzer verlangt werden, welche ein hohes Maß an Aufmerksamkeit erfordern.
Wir möchten herausfinden wie schnell und effizient ein Benutzer auf diese Aufgaben Vorbereitet werden kann. Vor allem die Frage, zu welchem Zeitpunkt der Benutzer geweckt wird gilt es hierbei zu erforschen.
