\chapter{Einführung}
\todoTob{Wusste nicht wohin mit diesem TODO also hier her: haben wir alles aus limesurvey was wir nutzen können genutzt?}
Virtuelle und Augmentierte Umgebungen begleiten uns bereits seit einigen Jahren.  Die Produkte, welche dabei zum Einsatz kommen, sind zum jetzigen Zeitpunkt aufgrund der unpraktischen Größe nicht für das alltägliche Umfeld des Durchschnittsbürgers geeignet und angesichts der hohen Preise auch nicht sonderlich weit verbreitet.
Werden in Zukunft diese Systeme kleiner, leichter, oder sogar permanent mit dem menschlichen Körper verbunden, wird es möglich sein, jeden Bereich des Lebens zu augmentieren. 

\section{Motivation}

Virtual Reality (VR) wird eingesetzt, um ein breites Spektrum an Erfahrungen zu ermöglichen. In der Forschung hat VR eine lange Tradition als Werkzeug zur Erforschung komplexer Phänomene (unterschiedliche Perspektiven und Ängste) und zunehmend als Technologie zur Erforschung neuer Erfahrungen und Kooperationen.
AR/VR soll ein stetiger Begleiter sein und dem Nutzer auch in anspruchsvollen oder unerwarteten Situationen unter die Arme greifen.
Eine solche unerwartete Situation könnte beispielsweise direkt nach dem Aufwachen aus dem Schlaf oder aus einer sehr entspannten Ruhephase der Fall sein. So können in autonomen Autos Aufgaben vom Fahrer übernommen, im Nachtdienst eines Sicherheitsunternehmens kritische Vorgänge überwacht, oder am Morgen Herausforderungen vom Benutzer verlangt werden, welche ein hohes Maß an Aufmerksamkeit erfordern. 
In unterschiedlichsten Fällen können Nutzer auf die Erledigung einer Aufgabe in einer virtuellen Umgebung nur schwer vorbereitet werden, wenn diese  plötzlich oder ohne Überleitung gestellt wird.
So können Nutzern einer virtuellen oder augmentierten Realität (VR/AR) beim Wechsel der Umgebung, oder beim Wechsel in die digitale Umgebung, Informationen fehlen, welche notwendig sind um sich schnell, zuverlässig und ohne potenzielle Fehlerquellen an diese zu gewöhnen.\\
Wir möchten herausfinden wie ein Benutzer auf bevorstehenden Aufgaben vorbereitet werden kann und wie gut er diese Aufgaben nach einem schlafähnlichen Zustand bewältigen kann. Vor allem die Frage, durch welche Parameter der Benutzer am besten geweckt werden sollte gilt es hierbei zu erforschen. Wichtig ist hierbei vorallem die Qualität der erbrachten Leistung.\\
Das Projekt \projectName \  soll es einem Nutzer ermöglichen in einem solchen Szenario alle relevanten Informationen innerhalb kürzester Zeit aufzunehmen und entsprechend der Aufgabe zu reagieren. Des Weiteren soll untersucht werden auf welche Art und Weise dieser Vorgang zuverlässig durchgeführt werden kann. Hierbei spielen viele Faktoren eine Rolle wie beispielweise die Umgebung, der Zustand des Probanden und der zum Wecken gewählte Parameter.\\