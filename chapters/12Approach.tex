\section{Herangehensweise}\label{sec:approach}  

Für die Untersuchung der beschriebenen Problemstellung nutzen wir eine vollständig virtuelle Umgebung (VR), erstellt mit Unity 3D\footnote{~Unity3D~\url{https://unity3d.com}}. Innerhalb dieser Umgebung haben wir die größte Kontrolle über die zu untersuchenden Parameter. Außerdem können wir die Bewegungen der Benutzer, alle benötigten Zeiten und andere Werte aufzeichnen und für die Auswertung heranziehen. Weiter ermöglicht uns diese Technologie ein schnelles Prototyping und eine schnelle Anpassung an Ideen und Diskussionsergebnisse während der Entwicklung.

Um eine optimale Umgebung zu schaffen nutzen wir einen Gaming-Stuhl mit Armlehnen und einer verstellbaren Rückenlehne. Dieser lässt sich, ähnlich wie ein Autositz in eine liegende Position verstellen, wodurch die Teilnehmer der Studie eigenständig eine für sie angenehme Position wählen können, wie in Abbildung~\ref{fig:chair_backrest} gezeigt ist. Wir haben uns dafür entschieden eine beruhigende Umgebung zu erschaffen, wie sie auftreten könnte, wenn sich der Fahrer eines autonomen Fahrzeugs dazu entschließt ein Nickerchen zu machen.
\todoSab{Zeilenumbrüche sind hier komisch}


Fragebögen für demografische Fragen und die Erfahrung mit VR und AR werden über das Limesurvey Online-Fragebogen Tool der Universität Ulm\footnote{\url{https://surveys.informatik.uni-ulm.de/limesurvey/
}} gestellt und beantwortet.
Das HMD unserer Wahl ist die HTC Vive Pro\footnote{\url{https://www.vive.com/}}, da für diese viele einsteigerfreundliche Tutorials im Internet vorhanden sind und die Entwicklung der virtuellen Umgebung mit den Steam VR-Plugins für Unity 3D nahtlos funktioniert. Die Daten, die dabei erhoben werden speichern wir im \textit{.csv} Format, da dies eine einfache Auswertung und Verarbeitung mittels gängiger Statistikprogrammen ermöglicht. Eine Deserialisierung der erhobenen Daten erfolgt mit der Skriptsprache Python unter Version 3.5\footnote{\url{https://www.python.org/}} und eine spätere Auswertung mittels IBMs SPSS (Version 24)\footnote{\url{https://www.ibm.com/analytics/spss-statistics-software}}. Weiter nutzen wir auch R\footnote{\url{https://www.r-project.org/}} Skripte zur statistischen Auswertung sowie zur Visualisierung der Ergebnisse.
