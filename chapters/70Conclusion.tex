\chapter{Schlussfolgerung}

\todoAll{Schlussfolgerung Schreiben}
\todoAll{Wann wecken?}
\todoAll{Wie schnell wecken?}
\todoAll{Welche darstellung?}
\todoAll{Wie vorbereiten?}
\todoAll{Auf die Hypothesen eingehen}

In unserer Arbeit nutzen wir eine in virtueller Realität entwickelte Umgebung um zu untersuchen in welcher Art und Weise Menschen aus einer Ruhephase geweckt werden können.
Wir interessieren und im Besonderen dafür, welche Unterschiede zwischen den beiden Parametern \textit{Licht} und \textit{Ton} existieren.
Unsere Annahme besteht darin, dass unterschiede zwischen den Gruppen bestehen und eine der beiden sich als schneller in der Aufgabenbearbeitung herausstellt.

Beachtet man die in Kapitel~\ref{cha:approach} aufgelisteten Hypothesen, so wird klar, unsere Annahmen weder bestätigt noch widerlegt werden können. 
Unsere Ergebnisse zeigen, dass sowohl die Gruppe der mit Licht geweckten, wie auch die Gruppe der mit Ton geweckten Probanden große Unterschiede in der Bearbeitungszeit sowie der Fehlerrate haben.
Zwar lässt sich gelegentlich ein Trend im Bezug auf Hypothese~\ref{hyp:lichtErfolgreicher} wahrnehmen, jedoch fällt dieser nicht signifikant aus. 

Hypothese~\ref{hyp:lichtSchneller} wird ebenfalls nicht durch die Werte bestätigt, da keine gravierenden Tendenzen zu beobachten sind. Auch die Gegenhypothese lässt sich hiermit aber nicht bestätigen, weshalb wir allerdings auf zukünftige Forschung gespannt sind.

Eine unserer Aufgaben zeigt, dass Menschen die mit einem akustischen Signal geweckt werden tendenziell etwas schneller sind und auch weniger Fehler machen als Probanden, welche einem Lichtreiz zum Wecken ausgesetzt wurden.
Dies spricht dementsprechend für die Gegenhypothese zu Hypothese~\ref{hyp:langKurzErfolgreicher}.

Auch in Bezug auf die Nullhypothese~\ref{hyp:langKurzSchneller} und~\ref{hyp:langKurzErfolgreicher} lässt sich nicht herauslesen, dass die Gruppe, welche mit Licht geweckt wurde, bedeutsam besser oder schlechter als die Audio-Gruppe abgeschnitten hat.

Den einzigen auszumachenden, wenn jedoch nicht besonders signifikanten, Unterschied erkennt man beim Ordnen der Zahlen. 
Hier zeigen die Daten, dass die Fade 5 Gruppe deutlichere Extrema im Bezug auf die Geschwindigkeit aufwiesen. 
Sowohl der schnellste als auch der langsamste gehörten dieser Gruppe an. 
Die Fade 20 Gruppe hingegen bewegt sich deutlicher im Mittelfeld der Grafik ~\ref{fig:orderingMistakeTimeScatterplot}.

Obwohl keinerlei bedeutende Signifikanz in den Daten auszumachen ist, lässt sich doch der ein oder andere Trend erkennen, wodurch unsere Studie durchaus ein geeigneter Nährboden für zukünftige Forschungen sein kann. 
Unsere Forschung gewährt darüber hinaus einige Ansätze für die Realisierung, sowie Daten auf welche aufbauende Forschungen fußen können. 
Aus diesem Grund währen weitere Forschungen interessant, welche sowohl eine höhere Anzahl an Probanden abdecken würden, als auch Probanden, welche einer erweiterten Altersklasse beziehungsweise einer breiteren Bildungsschicht angehören, beinhalten würden. 
Des Weiteren würde eine längere Schlafdauer womöglich zu deutlicheren Ergebnisunterschieden führen, weshalb dies auch ein interessanter Punkt für zukünftige Forschung sein könnte.

\todoAll{Korrekturlesen}
