\chapter{Schlussfolgerung}

\todoAll{Schlussfolgerung Schreiben}
\todoAll{Wann wecken?}
\todoAll{Wie schnell wecken?}
\todoAll{Welche darstellung?}
\todoAll{Wie vorbereiten?}
\todoAll{Auf die Hypothesen eingehen}

Beachtet man die in Kapitel~\ref{cha:approach} aufgelisteten Hypothesen, so wird klar, dass keine der im voraus definierten Hypothesen komplett bestätigt wird. 
Zwar lässt sich beim Ordnen der Zahlen tatsächlich ein leichter Trend im Bezug auf Hypothese~\ref{hyp:lichtErfolgreicher} wahrnehmen, jedoch fällt diese nicht signifikant aus. 
Ansonsten lässt sich diese Hypothese jedoch nicht anhand er gesammelten Werte bestätigen. 
Hypothese~\ref{hyp:lichtSchneller} wird ebenfalls nicht durch die Werte bestätigt, da keine gravierenden Tendenzen zu beobachten sind.
Beim Stroop Test geht der Trend sogar eher in die Richtung, dass Menschen die mit einem akustischen Signal geweckt werden tendenziell eher etwas schneller sind und auch weniger Fehler machen als Probanden, welche einem Lichtreiz zum Wecken ausgesetzt wurden.
Dies spricht dementsprechend für die Gegenhypothese zu Hypothese~\ref{hyp:langKurzErfolgreicher}.

Auch in Bezug auf Hypothese~\ref{hyp:langKurzErfolgreicher} und~\ref{hyp:langKurzSchneller} lässt sich nicht herauslesen, dass eine der beiden Gruppen, welche mit Licht geweckt wurden bedeutsam besser oder schlechter als die andere Gruppe abgeschnitten hat. 
Den einzigen auszumachenden, wenn jedoch nicht besonders signifikanten, Unterschied erkennt man beim Ordnen der Zahlen. 
Hier zeigen die Daten, dass die Fade 5 Gruppe deutlichere Extrema im Bezug auf die Geschwindigkeit aufwiesen. 
Sowohl der schnellste als auch der langsamste gehörten dieser Gruppe an. 
Die Fade 20 Gruppe hingegen bewegt sich deutlicher im Mittelfeld der Grafik ~\ref{fig:orderingMistakeTimeScatterplot}.

\todoAll{Korrekturlesen}
