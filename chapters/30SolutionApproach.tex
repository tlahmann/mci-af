\chapter{Lösungsansatz}

Im Folgenden werden unsere Herangehensweisen an das Problem beschrieben. Wir untersuchen den Übergang zwischen einem schläfrigen Zustand, ohne kognitive Beanspruchung, und einem wachen Zustand, in dem Aufgaben von der Testperson übernommen werden können. Hierzu bietet sich aus zeitlichen Gründen eine Untersuchung mittels virtueller Realität an. Wir betrachten außerdem eine Reihe von Aufgaben, die in VR durchgeführt werden und einen Bezug auf reale Situationen im Kontext Autofahren aufweisen.

\section{Idee}

Wir möchten die Teilnehmer der Studie dazu bringen einen Zustand zu erreichen in dem das fehlerfreie Erledigen von Aufgaben eine gewisse geistige Anstrengung aufweist. Dies ist vergleichbar damit, dass im Straßenverkehr das autonome Fahren dem Fahrer erlaubt seine Aufmerksamkeit von der Straße zu nehmen, seinen Sitz zu kippen und die Augen zu schließen. Sollte der Fahrer daraufhin von seinem Fahrzeug aufgefordert werden in eine Situation einzugreifen und eine Entscheidung zu treffen, wie Beispielsweise einen schwierigen Abschnitt der Strecke selbst zu Fahren, oder eine Entscheidung zu treffen, muss das Fahrzeug den Fahrer aufwecken und informieren. In diesem Zustand könnten Informationen nur schwer aufgenommen werden und eine Aufgabe, die in diesem Zeitraum gestellt wird könnte so mit geringerem Erfolg erledigt werden, als wenn der Fahrer voll aufnahmefähig ist.

\todoTob{Dieses Paper könnte interessant sein, warum Entscheidungen von Menschen getroffen werden sollten... \cite{awad2018moral}.Für die Interessierten: 'Moral Machine' bei google eingeben -tl -> Sabrina laß und folgendes kam raus: -sb}
-- Szenario, autonomes Fahren, wenn es keinen Ausweg gibt bei einem Unfall eine Fahrspur zu wählen die keine Opfer fordert, AI soll entscheiden, dass bspw. ein Reh umzufahren besser als ein Mensch umzufahren ist -> ethische Probs. Dann wird hauptsächlich auf Länderunterschiede eingegangen alles eher lame. Was man vllt brauchen kann:--\\
Bis jetzt hat man noch so eine wichtige Entscheidung wie ob jemand stirbt oder wer im Falle eines unumgehlichen Todes stirbt eine AI entscheiden lassen. Hier treffen viele ethische Konflikte aufeinander. Was wichtig ist, dass jegliche programmierbare Entscheidung den Tod beteiligter verhindern soll. Man kann nicht sagen, dass alle menschlichen Fahrer richtig entscheiden in allen Situationen, jedoch kann es nicht festelegt werden, wie eine Maschine in einem kritischen Fall reagieren soll, ohne eine komplizierte ethische Diskussion.
Bevor man autonomen Autos erlauben kann ethische Entscheidungen zu treffen, muss man ein globales Gespräch führen, um Präferenzen gegenüber den Unternehmen, die moralische Algorithmen entwickeln, und gegenüber den politischen Entscheidungsträgern, die sie regulieren werden, zum Ausdruck zu bringen. 


Wir untersuchen in dieser Arbeit nicht die Auswirkungen des Schlafens, oder des Mangels an Schlaf auf das Gehirn, sondern viel mehr die Überführung von einem trägen oder schläfrigen Zustand in einen Wachen. 

Um eine möglichst kontrollierbare Testumgebung mit vergleichbaren Ergebnissen zu haben nutzen wir VR in einem separaten Raum der Universität Ulm. Hier werden die Teilnehmer gebeten mit dem VR-HMD in einer initialen Phase zu entspannen. Das Ziel ist es die beschriebene Situation der geringeren Aufnahmefähigkeit und Schläfrigkeit zu erzeugen, wie er auch nach dem Schlafen auftreten kann. 

Im Anschluss an die Ruhephase werden die Nutzer auf unterschiedliche Arten "`aufgeweckt' und ihnen werden drei Aufgaben gestellt. Die von uns untersuchten Arten des Weckens sind Licht und Ton. Wobei die Einstellung des Lichts noch in zwei weitere Gruppen unterteilt ist. Für Gruppe eins wird das licht innerhalb von 5 Sekunden von 0\% Intensität auf 100\% erhöht, bei Gruppe zwei geschieht dies Über einen Zeitraum von 20 Sekunden. 
Allen drei Gruppen werden im Anschluss die Aufgaben iterativ präsentiert.

Auf dieser Grundlage formulieren und untersuchen wir die folgenden Hypothese auf:

\begin{hyp}[H\ref{hyp:lichtSchneller}]\label{hyp:lichtSchneller}
	Menschen, die mit Licht geweckt werden, können sich in kürzerer Zeit auf eine gestellte Aufgabe einstellen, als Menschen, die mit Ton geweckt werden.
\end{hyp}

\begin{hyp}[H\ref{hyp:lichtErfolgreicher}]\label{hyp:lichtErfolgreicher}
	Menschen die mit Licht geweckt werden, können eine gestellte Aufgabe mit weniger Fehlern erledigen, als Menschen, die mit Ton geweckt werden.
\end{hyp}

\begin{hyp}[H\ref{hyp:langKurzSchneller}]\label{hyp:langKurzSchneller}
	Menschen die langsam geweckt werden können sich in kürzerer Zeit auf eine gestellte Aufgabe einstellen, als Menschen, die abrupt aus dem Schlaf gerissen werden.
\end{hyp}

und 

\begin{hyp}[H\ref{hyp:langKurzErfolgreicher}]\label{hyp:langKurzErfolgreicher}
	Menschen die langsam geweckt werden können eine gestellte Aufgabe mit weniger Fehlern erledigen, als Menschen, die abrupt aus dem Schlaf gerissen werden.
\end{hyp}

\section{Zeitplan}
Die Daten beziehen sich auf den Zeitpunkt zu dem der jeweilige Schritt abgeschlossen sein sollte.
\begin{itemize}
	\item \textbf{Dezember 2018} Recherche bestehender Forschung
	\item \textbf{Januar - Mai 2019} Beginnen mit der Implementierung und Modellierung der digitalen Umgebung in Unity3D und Vorbereitung für den Studiendurchlauf
	\item \textbf{Juni 2019} Erste Studie durchführen
	\item \textbf{Juni 2019} Abschluss der Studie und Erweiterung auf Zehn weitere Probanden
	\item \textbf{Juli 2019} Einführung auf eine weitere 15-köpfige Testgruppe, welche mit Ton geweckt wird und Implementierung und Entwurf dieser neuen angepassten Studie
	\item \textbf{August 2019} Abschluss der gesamten Studie
	\item \textbf{September 2019} Auswertung der Ergebnisse
	\item \textbf{Oktober 2019} Erstellen der Ausarbeitung und Präsentation
\end{itemize}

\section{Testaufbau}
Sitzend werden Probanden erst in einen entspannten Zustand versetzt. In diesem verweilen sie möglichst ohne Ablenkung, bis sich eine Gelassenheit oder Trägheit einstellt. Diese kann von entspanntem Sitzen bis hin zum Schlaf führen, eine genaue Zeitspanne hierfür kann zwischen Probanden variieren und muss in Tests bestimmt werden.

Nachfolgend wird der Teilnehmer aus diesem Zustand geleitet und mit einer Aufgabe konfrontiert. Während der Erledigung dieser werden unterschiedliche Parameter aufgezeichnet und später ausgewertet. Die erfassten Parameter sind die folgenden:

\begin{itemize}
	\item Zeit in der eine Aufgabe erledigt wird
	\item Fehlerrate bei der Erledigung der Aufgabe
	\item Blickrichtung
\end{itemize}\todoTob{Redundant mit der Implementierung... wo soll das am ehesten hin?}

Die erste Studie umfasst ungefähr 30 Minuten, hierfür werden die Teilnehmer mit fünf Euro entlohnt. Der Ablauf der Studie Umfasst folgende Punkte:

\begin{enumerate}
	\item \textbf{5 Minuten} Vorbereitung und Einführung in den Studienablauf inklusive der Bedienung der VR-Umgebung
	\item \textbf{15 Minuten} Beruhigungsphase bis hin zum Schlafen
	\item \textbf{3-5 Minuten} Aufgaben lösen
	\item \textbf{5 Minuten} Fragebögen beantworten
\end{enumerate}

Es handelt sich um eine between subject Studie. Im ersten Durchlauf erfassen wir die genannten Parameter unter der Betrachtung der Zeit in der die virtuelle Umgebung erhellt wird. Die genauen Zeiten werden in einer Testphase während des Implementierens eingegrenzt.

Zudem werden die Probanden entweder mit Ton oder durch einen Lichtreiz "`geweckt"'. Auch dieser Parameter wird in einer Testphase experimentell angenähert. Nähere Informationen zum genauen Studienablauf werden im Abschnitt der Studiendurchführung erörtert.

\subsection{Studienumgebung}
\todoLuc{Raum und Licht ergänzen oder umschrieben falls notwenig, mehr zum Abschnitt}
Um die Studie durchführen zu können, bauten wir unser Equipment in einem kleinen abgeschotteten Hinterzimmer eines Computerlabors auf. Dadurch dass die Probanden keine stehenden oder gar laufenden Bewegungen vollziehen mussten, reichte ein Bereich, in dem sich der Stuhl geradeso drehen konnte. Der Raum war mit einem Fenster versehen und so kam immer Tageslicht ins Zimmer herein, was die Probanden jedoch durch das Tragen der VR Brille während der Durchführung der Studie nicht wahrnehmen konnten, sondern nur in dem Anfangsgespräch und beim anschließenden Fragebogenausfüllen. Da im Nebenzimmer ebenfalls Studien durchgeführt wurden, waren Störgeräusche nicht immer zu verhindern. So kam es vor, dass manche Probanden ohne jegliche Störgeräusche die Studie absolvieren konnten, wohingegen bei einigen anderen verschiedenste, relativ leise, dennoch hörbare Störungen auftraten. Wir achteten darauf, dass die räumlichen Begebenheiten bei jedem Teilnehmer der Studie gleich blieben, darunter Aspekte wie Art der Begrüßung und des Vorworts oder Aufbau des Zimmers, darunter Stuhllokalisation, Lichteinstellungen und die Aufklärung, dass wir als Studienleiter ebenfalls durchgehend im Raum anwesend waren. 