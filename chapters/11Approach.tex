\section{Herangehensweise}\label{sec:approach}  

Im ersten Abschnitt des Projekts betrieben wir Recherche zu den Themen Schlafen, Aufwachen, VR/AR sowie zum Bereich Aufgaben bewältigen. Nachdem reichlich Recherche betrieben wurde stellten wir Hypothesen auf, die hauptsächlich die Parameter des Weckens, sowie auch die effektive Aufgabenbewältigung betreffen. Mit diesen Hypothesen sind wir in die nächste Phase eingestiegen. 
Wir erstellten eine virtuelle Umgebung mittels Unity 3D\footnote{~Unity3D~\url{https://unity3d.com}} um eine entspannende Atmosphäre zu erschaffen. Um den Probanden eine entspannte physische Atmosphäre zu bieten wurde für die Studie ein bequemer Bürostuhl mit verstellbarer Lehne genutzt. Nachdem wir die erste Studie durchgeführt hatten, änderten wir den 'Aufweckparameter' und führten eine zweite Studie mit neuen Probanden durch. \todoSab{Hier welche Zeitform? -> Präsens}
\todoSab{Hier sollte eher so etwas stehen, wie: "Welche methoden haben wir gewählt und warum?" "Welche Technologien haben wir verwendet und warum?" "Wie haben wir mögliche Probleme einer Studie bewältigt und welche Daten wollen wir überhaupt erheben". Also die grundlegende Herangehensweise. Wie das Projekt abgelaufen ist soll in 'Studiendurchführung'. Aber nicht zu viel, weil es da ja eigentlich ausführlich beschrieben wird.}
