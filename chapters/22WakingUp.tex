\section{Aufwachen}\label{sec:relatedWork.aufwachen}

Bis zu 2 Stunden nach dem Aufwachen kann die subjektive Aufmerksamkeit und die kognitive Leistungsfähigkeit noch beeinträchtigt sein~\cite{jewett1999time}. Ein Grund für die Schlafträgheit nach dem Aufwachen könnte die automatische Herunterregulierung der Körpertemperatur sein, die auftritt, wenn der Körper schläft~\cite{dinges1990you}. Die damit einhergehende geringere geistige Leistungsfähigkeit lässt sich in Experimenten messen~\cite{dinges1990you}. 
Im Umkehrschluss kann angenommen werden, dass die Erhöhung der Körpertemperatur dem schlaffen Gefühl entgegenwirkt~\cite{jewett1999time}.
Jewett et. al. fanden heraus, dass aber weder die Helligkeit der Umgebung noch andere Aktivitäten, die kurz nach dem Aufwachen erledigt wurden (Essen, duschen, etc.) signifikant die Aufmerksamkeit noch die Schlafträgheit oder deren Abbau beeinflussten~\cite{jewett1999time}.

In einer Situation von Müdigkeit, die direkt nach dem Aufwachen einsetzt und erst über die Zeit abgebaut wird konnten Probanden einer Studie noch einfache soziale Interaktion durchführen~\cite{dinges1990you}. 
Die funktionale Deafferenzierung, wie sie von Broughton genannt wurde~\cite{broughton1968sleep} um die niedrigen Hirnaktivitäten nach dem Aufwachen zu beschreiben, erschweren die Aufbringung der mentalen Kapazitäten für komplexe Aufgaben nach dem Erwachen. 
Daher müssen, in all den Situationen, welche eine erhöhte Leistungsfähigkeit benötigen, die unumgänglichen Effekte der Schlafträgheit im Vorfeld beachtet werden. 
Diejenigen, die eine Aufgabe erledigen sollen, können hilfreiche Tools zur Unterstützung gegeben werden~\cite{ferrara2000sleep}\todoTob{Satz umschreiben, nicht verständlich}. 
Aktuell könnten alarmierende Faktoren\todoTob{was sind alamierende faktoren?} verwendet werden um diese Ziele\todoTob{welche ziele?} zu erreichen, aber weitere Forschung muss bestätigen welcher der Faktoren am effizientesten ist~\cite{ferrara2000sleep}.

\cite{online:muedesGehirn}

In den ersten Minuten nach dem Aufwachen können Probanden selbst einfache Aufgaben, wie zum Beispiel kleine Rechnungen, nicht effektiv umsetzen. Die Länge der Schlafdauer hat dabei keine Auswirkung auf das Ergebnis. Man spricht auch vom 'toten Punkt nach dem Aufwachen'. Nach einem einwöchigen Untersuchen der Probanden, konnte man erkennen, dass die Nutzer eine Minute nach dem Aufwachen gerade einmal 65 Prozent ihrer sonstigen Leistungsfähigkeit erreichten. Nach 26 Stunden ohne Schlaf schnitten sie mit ca 85 Prozent ihrer Maximalpunktzahl signifikant besser ab. Forscher gingen bislang davon aus, dass wichtige Aufgaben, wie zum Beispiel Bereitschaftsdienst, nach Schlafentzug zu erledigen deutlich gefährlicher ist, als direkt nach dem Aufwachen, jedoch kann dies hier nicht bestätigt werden.
Schlafentzug kann ähnliche Auswirkungen wie Alkoholkonsum haben.
Die Leistungsfähigkeit nach dem Aufwachen normalisiert sich zwischen 20 bis 30 Minuten vollständig, jedoch können manche Einschränkung aber auch bis zu einer Stunde anhalten. \\

\cite{online:uebermuedetesHirn}

In dieser Studie wurden nur motorische Fähigkeiten und die dabei beteiligten Regionen des Hirns untersucht. Bei Müdigkeit setzen gewisse Teile des Denkorgans ab, wobei andere Abteile versuchen normal weiter zu arbeiten. Die Müdigkeit führt zu kleine Ausfällen in gewissen Hirnregionen. An kleinen Tierversuchen wurde gemessen, dass die Hirnrinde bei starker Müdigkeit unabhängig von der Umgebung in einen inaktiven Zustand fallen können und die Tiere wach erschienen, jedoch nach langem Wachhalten signifikant unterschiedliches Verhalten zum Normalzustand aufwiesen. Trotz Schwerpunkt auf motorische Untersuchungen, wird davon ausgegangen, dass diese Ergebnisse auf Wahrnehmung und Denkvermögen teilweise übertragen werden kann.\\

\cite{online:muede}

Müdigkeiten kann Schlaftrunkenheit verursachen, was die Hauptursache für Beeinträchtigungen nachdem Aufstehen morgens ist. Der Mensch kann schwer von einem aktiven in einen ruhenden Zustand zu wechseln und auch andersrum. Am morgen können diese beiden Aktionen nicht unterschieden werden und so kommt es zu körperlichen Unstimmigkeiten und der Mensch ist von der Müdigkeit beeinträchtigt. Wieder wurden Matheaufgaben an Nutzern, die gerade aufgewacht sind getestet und wieder stellt man fest, dass die Probanden deutlich schlechter als in einem wachen Zustand, der nach einer Stunde nach dem Aufwachen gemessen wurde, abschnitten. Es wird vermutet, dass das Gehirn zwischen den beiden Modi gefangen ist und die abnehmende Leistungsfähigkeit darauf zurückzuführen ist.\\

\cite{online:streiche}

Wenn man sich an Ereignisse erinnert, dann sind es meist Erinnerungen in denen man vollkommend wach und zurechnungsfähig war. In einem müden Zustand kann unser Gehirn schwieriger Vorkommnisse und Details abspeichern. Diese Erkenntnis kann zu einigen Problemen führen, wie beispielsweise bei einem Autounfall der morgens passiert oder Verbrechen wo ein Augenzeuge wichtig ist. Durch ein Experiment wurden über 100 Probanden Bilder von einem Diebstahl in unterschiedlichen Momenten gezeigt. Die Gruppen gliederten sich in 1) abends und schlafen, 2) abends und die Nacht durchmachen, 3) morgens und geschlafen und 4) morgens und die Nacht durchgemacht. Die Ergebnisse nach einem Fragebogen wiesen auf, dass die Probanden, die Fotos nach der durchwachten Nacht gesehen hatten, bei ihren Erinnerungen deutlich häufiger falsch lagen als ihre ausgeschlafenen Kollegen. Hatten sie dagegen das Foto vor der schlaflosen Nacht gesehen, war ihre Erinnerung daran sehr viel besser. Nach Ansicht der Forscher zeigt dies, dass es eine große Rolle spielen kann, ob ein Zeuge zum Zeitpunkt seiner Beobachtung ausgeschlafen war oder nicht. Durch Schlafmangel könne also weniger Details aufgenommen werden und falsche Erinnerungen aufzunehmen wird gefordert, was beispielweise in der Kriminalistik schwere Folgen haben kann.\\


-\cite{dinges1990you}\\
Es ist ein Phänomen, dass man beim Aufwachen mehr müde und träge ist, als abends vor dem zu Bett gehen. Diese Beeinträchtigung ist bescheiden und kurzlebig aufgrund des allmählichen Erwachens oder eines langsamen Übergangs aus dem hypnopompen Zustand. Bei abruptem Erwachen spielt es auch keine Rolle, ob man aus dem 'Nachtschlaf' oder einem Nap erwacht. Die Probleme von hypnopompen Disorientierung und Verwirrung lassen sich mit eine Vielzahl von Schlagwörtern wie Schlafträgheit, prödormitale Schlaflosigkeit oder auch Schlafdrunkenheit in Verbindung bringen. Die Hypnagogie bezeichnet dabei einen Bewusstseinszustand, der beim Einschlafen auftreten kann. Eine Person im hypnagogen Zustand kann in manchen Fällen visuelle, auditive und taktile Halluzinationen erleben, ohne sich bewegen zu können. Obwohl der Person bewusst ist, dass sie halluziniert, kann sie meist nicht darauf reagieren.\\
Die Tiefe von Schlaf hängt immer von der Zeit ohne Schlaf davor ab. Ein Porband der 2 Tage schlaflos war, schläft viel tiefer als jemand der viel schlaf genossen hat.\\%komisches paper

-\cite{wilkinson1971performance}\\
Es wurde die Performance von Probanden getestet, die mitten in der Nacht geweckt wurden und innerhalb von 4 Minuten nach dem Erwachen gewisse Tests machen mussten. 11 Minuten gingen diese Tests und sie haben die Themen Reaktionszeit, Berechnung und muskulären Koordination und Stabilität beinhaltet. In allen drei Tests lag die Leistung unter dem normalen Niveau, welches man am Tag haben würde. Es wird vorgeshclagen, dass man mindestens 15 Minuten warten muss, nach direktem Wecken, um solide Leistungen von Probanden zu erhalten. Zwei Faktoren beeinflussen diese Messungen noch: Schlaftiefe und der zirkadiane Rhythmus der physiologischen und verhaltensbezogenen Aktivität (hat sein Tiefpunkt zwoschen 3 und 5 Uhr nachts).
%mir fällt auf dass viele paper echt das gleiche schreiben und ich langsam nichts mehr mit schlafen lesen kann ;) greez gehen raus an meine crew <3
