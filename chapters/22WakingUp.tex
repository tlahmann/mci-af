\section{Aufwachen}\label{sec:relatedWork.aufwachen}

Bis zu 2 Stunden nach dem Aufwachen kann die subjektive Aufmerksamkeit und die kognitive Leistungsfähigkeit noch beeinträchtigt sein~\cite{jewett1999time, online:muedesGehirn}. 
Die Länge der Schlafdauer hat dabei keine Auswirkung auf das Ergebnis. Man spricht auch vom 'toten Punkt nach dem Aufwachen'~\cite{online:muedesGehirn}. 

\subsubsection{Leistungsfähigkeit}

Ein Grund für die Schlafträgheit nach dem Aufwachen könnte die automatische Herunterregulierung der Körpertemperatur sein, die auftritt, wenn der Körper schläft~\cite{dinges1990you}. Die damit einhergehende geringere geistige Leistungsfähigkeit lässt sich in Experimenten messen~\cite{dinges1990you,wilkinson1971performance, online:muedesGehirn, online:muede, online:uebermuedetesHirn}. 
So konnte in einem Experiment feststellt werden, dass die Probanden eine Minute nach dem Aufwachen gerade einmal 65 Prozent ihrer üblichen Leistungsfähigkeit erreichten~\cite{online:muedesGehirn}. Nach 26 Stunden ohne Schlaf schnitten sie mit circa 85 Prozent ihrer Maximalpunktzahl signifikant besser ab~\cite{online:muedesGehirn}. 
Weiter heißt es, dass der menschliche Körper nur schwer von einem aktiven in einen ruhenden Zustand wechseln kann. Es wird vermutet, dass das Gehirn zwischen den beiden Zuständen gefangen ist und die abnehmende Leistungsfähigkeit darauf zurückzuführen ist~\cite{online:muede}.
Um dem entgegenzuwirken wird vorgeschlagen, dass mindestens 15 Minuten nach dem Aufwachen vergehen sollten um solide Leistungen von Probanden zu erhalten~\cite{wilkinson1971performance}.
Auch die motorischen Fähigkeiten lassen in einem müden Zustand nach dem Aufwachen nach. So konnte in Tierversuchen festgestellt werden, dass bestimmte Bereiche des Gehirns aussetzen, wenn sich Müdigkeit einstellt~\cite{online:uebermuedetesHirn}.

Im Umkehrschluss zur Trägheit durch Kälte kann angenommen werden, dass die Erhöhung der Körpertemperatur dem schlaffen Gefühl entgegenwirkt~\cite{jewett1999time}.
Jewett et al. fanden heraus, dass aber weder die Helligkeit der Umgebung noch andere Aktivitäten, die kurz nach dem Aufwachen erledigt wurden (essen, duschen, etc.) signifikant die Aufmerksamkeit noch die Schlafträgheit oder deren Abbau beeinflussten~\cite{jewett1999time}.

In einer Situation von Müdigkeit, die direkt nach dem Aufwachen einsetzt und erst über die Zeit abgebaut wird, konnten Probanden einer Studie noch einfache soziale Interaktion durchführen~\cite{dinges1990you}. 
Die funktionale Deafferenzierung, wie sie von Broughton genannt wurde~\cite{broughton1968sleep} um die niedrigen Hirnaktivitäten nach dem Aufwachen zu beschreiben, erschweren die Aufbringung der mentalen Kapazitäten für komplexe Aufgaben nach dem Erwachen.

Unterschiedliche Forscher gehen nun davon aus, dass die Erledigung wichtiger Aufgaben, wie zum Beispiel Bereitschaftsdienst, nach dem Aufwachen deutlich gefährlicher ist, als nach Schlafentzug.
Daher müssen, in all den Situationen, welche eine erhöhte Leistungsfähigkeit benötigen, die unumgänglichen Effekte der Schlafträgheit im Vorfeld beachtet werden. 
Für diejenigen, die eine Aufgabe erledigen sollen, können hilfreiche Tools zur Unterstützung herangezogen werden~\cite{ferrara2000sleep}. 
Es könnten alarmierende Faktoren wie körperliche Betätigung, Außenlärm, helles Licht und kaltes Wasser eingesetzt werden um die gewünschten Zustände des Nutzers zu erreichen, aber weitere Forschung muss bestätigen welcher der Faktoren am effizientesten ist~\cite{ferrara2000sleep}.
\todoAll{und deshlab setzen wir hier an blabla überleitung zum experiment}
Die Leistungsfähigkeit nach dem Aufwachen normalisiert sich zwischen 20 bis 30 Minuten vollständig, jedoch können manche Einschränkungen auch bis zu einer Stunde anhalten~\cite{online:muedesGehirn}.

\subsubsection{Erinnerungen}

Wenn man sich an Ereignisse erinnern soll, dann funktioniert es meist besser bei solchen, in denen man vollkommen wach und zurechnungsfähig war. In einem müden Zustand kann das Gehirn schwieriger Vorkommnisse und Details abspeichern und ordnen~\cite{online:streiche}. 
Diese Erkenntnis kann zu einigen Problemen führen. So kann beispielsweise bei einem Autounfall, der morgens passiert oder Verbrechen in denen ein Augenzeuge befragt wird, die Erinnerung an den genauen Hergang getrübt sein~\cite{online:streiche}. 
Durch ein Experiment wurden über 100 Probanden Bilder von einem Diebstahl in unterschiedlichen Momenten gezeigt. 
Die Gruppen dieses Experiments gliederten sich in diejenigen, die geschlafen haben und diejenigen, die eine Nacht lang wach blieben. Eine weitere Unterteilung ist die nach der Uhrzeit der Befragung in den jeweiligen Gruppen. Es wurden in beiden Gruppen weitere Unterteilungen gemacht nach der Befragung am morgen und der Befragung am Abend~\cite{online:streiche}. 
Die Ergebnisse zeigen hierbei, dass die Probanden, die Fotos nach der durchgemachten Nacht sahen, bei ihren Erinnerungen deutlich häufiger falsch lagen als die ausgeschlafenen Teilnehmer~\cite{online:streiche}.
Hatten sie dagegen das Foto vor der schlaflosen Nacht gesehen, war ihre Erinnerung daran sehr viel besser~\cite{online:streiche}. 
Nach Ansicht der Forscher zeigt dies, dass es eine große Rolle spielen kann, ob ein Zeuge zum Zeitpunkt seiner Beobachtung ausgeschlafen war oder nicht. Durch Schlafmangel können somit weniger Details aufgenommen werden~\cite{online:streiche}.
