\section{Motivation}\label{sec:motivation}

Betrachtet man die Entwicklung von virtueller (VR) und augmentierter Realität (AR), dann kann man schnell feststellen, dass diese Technologien eine ganze Bandbreite an unterstützenden Funktionen anbieten~\cite{hughes2005mixed,medenica2011augmented,hayhurst2018augmented}. Die Geräte, welche dies realisieren können bereits in naher Zukunft um ein Vielfaches kleiner und damit auch handlicher produziert werden~\cite{shibata2002head} und dadurch als fester Bestandteil unseres Alltags in vielen Lebenssituationen helfen.
Genauso gut kann eine feste Integration in alltägliche Gegenstände unseren Alltag durch eine zusätzliche, digitale Komponente erweitern. So können beispielsweise bereits heute Smart-Mirrors unseren Alltag anschaulich darstellen und die wichtigsten Informationen präsentieren. 
In Auto helfen Head-up-Displays dabei Informationen, wie die aktuelle Geschwindigkeit und Routenführung auf die Innenseite der Windschutzscheibe zu projizieren. Diese sollen dem Fahrer ermöglichen die Informationen zu erhalten, ohne den Blick von der Straße nehmen zu müssen.
Weiter ist die aktuelle Forschung dabei autonomes Fahren der Stufe 3 zu etablieren~\cite{rodel2014towards}. Dieses Level erfordert es, dass ein Fahrer noch in der Lage ist in den Prozess einzugreifen, sollte dies vom Fahrzeug verlangt werden. Es erlaubt aber auch, dass der Fahrer seine Aufmerksamkeit vollständig von der Straße nimmt und sich anderen Dingen zuwendet.
Sollte der Fahrer in einer solchen Situation aufgefordert werden eine Entscheidung zu treffen, oder auch das Führen des Fahrzeugs wieder in die eigenen Hände zu nehmen ist es unverzichtbar ihn über die aktuelle Situation in Kenntnis zu setzen und auf die bevorstehende Aufgabe vorzubereiten. 
Die angesprochenen Technologien spielen dabei eine wichtige Rolle um Informationen darzustellen. AR/VR soll ein stetiger Begleiter sein und dem Nutzer auch in anspruchsvollen oder unerwarteten Situationen unter die Arme greifen.

Auch in anderen Situationen kann die Überführung von einer Ruhe- oder Schlafsituation in eine Situation, die ein hohes Maß an Aufmerksamkeit fordert verlangt sein. Es könnten auch im Nachtdienst eines Sicherheitsunternehmens kritische Vorgänge überwacht werden, wenn ein Sicherheitssystem einen Alarm auslöst. Auch kann am Morgen eine Herausforderung auf einen Benutzer im privaten Haushalt warten, welche ein hohes Maß an Aufmerksamkeit erfordert. 
Außerdem können Nutzern einer virtuellen oder augmentierten Realität (VR/AR) beim Wechsel der Umgebung, oder beim Wechsel in die digitale Umgebung, Informationen fehlen, welche notwendig sind um sich schnell, zuverlässig und ohne potenzielle Fehlerquellen an diese zu gewöhnen \cite{knibbe2018dream}.

Wir untersuchen wie ein Benutzer auf bevorstehenden Aufgaben vorbereitet werden kann und wie gut er diese Aufgaben nach einem schlafähnlichen Zustand bewältigen kann. 
Vor allem die Frage, durch welche Parameter der Benutzer am besten geweckt bzw. aus seinem Ruhezustand geholt werden sollte, wird hierbei erforscht. Wichtig ist dabei vor allem die Qualität der erbrachten Leistung.\\
Das Projekt \projectName \,soll es einem Nutzer ermöglichen in einem solchen Szenario alle relevanten Informationen innerhalb kürzester Zeit aufzunehmen und entsprechend der Aufgabe zu reagieren. Des Weiteren soll untersucht werden, auf welche Art und Weise dieser Vorgang zuverlässig durchgeführt werden kann. Hierbei spielen viele Faktoren eine Rolle wie beispielsweise die Umgebung, der Zustand des Probanden und der zum Wecken gewählte Parameter.
