\chapter{Implementierung}
Auf Softwareseite soll eine digitale Umgebung mittels Unity 3D\footnote{~Unity3D~\url{https://unity3d.com}} erstellt werden. Das gewählte Interface zur Untersuchung der beschriebenen Problemstellung ist die HTC Vive\footnote{~HTC Vive~\url{https://www.vive.com}} mit den zugehörigen Controllern. Bewegung innerhalb der digitalen Umgebung ist, bis auf Kopfbewegungen, nicht vorgesehen, da die Probanden in einem sitzenden Zustand untersucht werden. Die Eingabemethoden zur Bewältigung der gestellten Aufgaben werden mit Zeigeoperationen innerhalb der virtuellen Umgebung realisiert. Eine Implementierung zum Nachverfolgen der Augenbewegung von Probanden (Eye-Tracking) wird nicht selbst durchgeführt, sondern auf bestehende Implementierungen zurückgegriffen. 

Fragebögen werden über das Limesurvey online Fragebogen Tool der Universität Ulm\footnote{~Limesurvey~\url{https://surveys.informatik.uni-ulm.de/limesurvey/
}} gestellt und beantwortet. 
