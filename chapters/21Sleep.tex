\section{Schlafen}\label{sec:relatedWork.schlafen}

Die Schlafforschung beschäftigt sich bereits seit einigen Jahren mit den Phasen des Schlafes. So ist bekannt, dass gesunder Schlaf aus sich wiederholenden Zyklen von jeweils zwei Phasen besteht~\cite{broughton1968sleep}.
Diese Phasen sind der \textit{Tiefschlaf} und die \textit{Rapid Eye Movement} Phase (REM-Phase), welche durch gelegentliches Erwachen der Testpersonen unterbrochen werden~\cite{broughton1968sleep}. Der Übergang der beiden Phasen kann allerdings auch als eigenständige Phase von \textit{leichtem Schlaf} bezeichnet werden. 
In der Tiefschlafphase taucht der Körper üblicherweise ab. Das Gehirn sendet Hormone in den Kreislauf, um die Körperzellen regenerieren zu lassen. Des Weiteren sinkt auch der Blutdruck.
Durchschnittlich dauert solch ein Schlafphasen-Zyklus etwa 90 Minuten~\cite{broughton1968sleep}. Dabei nimmt der Anteil des Tiefschlafs mit zunehmender Dauer des Schlafes ab.
Vor allem in der REM-Phase ist die Gehirn- sowie die Traumaktivität der Probanden erhöht~\cite{gackenbach1991herrscher, broughton1968sleep}. Testpersonen, welche in dieser Phase geweckt werden oder aufwachen, können durch die erhöhte Hirnaktivität schneller Informationen aufnehmen und Aufgaben erledigen. Sie können sich daher besser orientieren und machen weniger Fehler~\cite{aschoff1985perception}. 

\subsubsection{Unfreiwilliges Einschlafen}

Bei langanhaltenden, gleichbleibenden Arbeitsvorgängen kann es schnell passieren, dass man müde wird~\cite{dinges1985assessing,kraemer2000time}. So kann beim Führen eines Kraftfahrzeugs Sekundenschlaf eintreten~\cite{ruhle2008sekundenschlaf, muttray2010videoanalyse,mccartt2000factors}, der die Leistungsfähigkeit des Fahrers deutlich senkt~\cite{boyle2008driver}. Je länger sich dabei der Ablauf nicht ändert, desto größer werden die Symptome von Schlafmangel und desto gravierender werden seine Auswirkungen~\cite{boyle2008driver,mccartt2000factors}.
Diese Probleme können einerseits durch kurze, geplante Nickerchen oder Schlafphasen gelindert werden, erhöht allerdings auch das Risiko, dass die Person bei abruptem Erwachen Schwierigkeiten hat, Informationen aufzunehmen oder Aufgaben zu erfüllen~\cite{dinges1985assessing}. 
Ein längerer Schlafentzug erhöhte die Menge an Tiefschlaf in den Tagschlafepisoden, was mit einer größeren Abnahme der kognitiven Leistung danach verbunden ist~\cite{dinges1985assessing}. Die Manipulation von zunehmenden Schlafmangel führte zu einem tieferen Schlaf, was signifikante Leistungseinbußen hervorruft~\cite{dinges1985assessing}. Dies ist für die kognitive Leistung am dramatischsten. Die direkt nach dem schlafen auftretende Verwirrung wird \textit{Schlafträgheit} genannt.

\subsubsection{Der Zeitpunkt des Einschlafens}

Abhängig von der Tageszeit existieren Unterschiede in der Performanz, der Selbsteinschätzung und anderer psychologischer Parameter bei voll ausgeschlafenen Probanden~\cite{kraemer2000time}.
Kraemer et al. haben in einer Studie analysiert, wie die Variation der Tageszeit mit verschiedenen Aufmerksamkeitsindikatoren zu verknüpfen sind~\cite{kraemer2000time}. 
Nach einer mit ausreichend Schlaf erfüllten Nacht wurden Probanden von 7 Uhr bis 23 Uhr alle zwei Stunden mit Tests konfrontiert, die verschiedene Kernthemen untersuchen, darunter Reaktionszeit, Pupillometrie\footnote{Mit Pupillometrie werden die diagnostischen Messverfahren der jeweiligen Pupillengrößen und Lichtreaktionen, sowie Vergleichsmessungen zwischen dem rechten und linken Auge bezeichnet~\cite{sachsenweger1975neuroophthalmologie}.} und Visualisierung.
Augenmerk dieser Studie lag auf der Beobachtung von tageszeitliche Schwankungen. Nach der Analyse wurden in fast allen untersuchten Parametern Unterschiede deutlich. 
Im Bezug auf die Selbsteinschätzung und die damit verbundenen Tests, wurde ein Höchstmaß der Wachsamkeit zwischen 11:00 Uhr und 15:00 Uhr gemessen~\cite{kraemer2000time}. Dies legt nahe, dass nach dem aufwachen eine gewisse Zeit vergehen sollte bevor eine Person mit anspruchsvollen Aufgaben vertraut werden kann.

Die Aufmerksamkeit und Aufnahmefähigkeit hat also direkt etwas mit der Tageszeit zu tun.
Mit einigen Schwankungen, bleibt die Aufmerksamkeit vom frühen Morgen an vergleichsweise hoch und nimmt ab Nachmittag stetig ab~\cite{kraemer2000time}.
Phänomene wie unter anderem Wachsamkeit und Leistung sind keine stabilen Eigenschaften, sondern variieren über einen gesamten Tag und sogar über die Lebensdauer von Menschen hinweg. 
Es kommt stark auf den Tageszeitpunkt und auf die Merkmale einzelner Personen an~\cite{kraemer2000time}.

\subsubsection{Die Grenze zwischen Wach­zu­stand und Schlaf}

Der Übergang zwischen Schlafen und dem Wachzustand kann ebenfalls als eigenständige Schlafphase beschrieben werden. Es wird als langsamer, komplexer Prozess beschrieben, der gewisse Zeit braucht um abgeschlossen zu sein~\cite{ferrara2000sleep}. 
Während dieser Übergangszeit, die einige Merkmale sowohl mit dem Wach- als auch mit dem Schlafzustand teilt, ist eine klare Trennung zwischen verschiedenen psychologischen, kognitiven und verhaltensbezogenen Parametern erkennbar~\cite{ferrara2000sleep}. 
Ein Nutzer kann in einer solchen Situation möglicherweise noch in der Lage sein, einfache Interaktionen durchzuführen, jedoch treten bei komplexeren Aufgaben Schwächen auf, da die Gehirnaktivität beim Erwachen nicht vollständig präsent ist. 
Aus diesen Gründen müssen in Situationen, die eine hochqualifizierte Leistung unmittelbar nach dem Erwachen erfordern, die unvermeidlichen negativen Auswirkungen im Voraus berücksichtigt werden~\cite{ferrara2000sleep}.
