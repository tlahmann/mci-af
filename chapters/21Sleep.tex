\section{Schlafen}\label{sec:relatedWork.schlafen}

- \cite{broughton1968sleep}\\
Wichtige soziale und meidizinische Probleme wie Schlafstörungen (Alpträume, nächtliches Bettnässen oder Schlafterror) sind Probleme die an bestimmten Punkten während der Schlafphase auftreten können. Hier wurden Daten erhoben, die die Hypothese stützen, dass solche Aktionen unabhängig von typischen Perioden der Traumaktivität auftreten.\\
Eine durschnittliche Nacht eines Erwachsenen besteht aus einer Reihe von Zyklen, welche alle aus einer nicht-REM Phase, einer REM Phase und einer Phase mit Schlaf, unterbrochen von gelegentlichem Erwachen, besteht. Die ersten Zyklendurchläufe weisen häufig Tiefschlafaktivitäten auf, während sich im Laufe der Durchläufe die Proportionen der REM Phasen verändern. Die REM Phasen werden länger und die Nicht-REM Phasen, insbesondere der Tiefschlaf nimmt ab. Die Dauer der Schlafzyklen nimmt mit zunehmendem Alter zu, von 45 bis 55 Minuten bei einem Säugling bis 75 bis 90 Minuten bei jungen Erwachsenen.\\
Schlafstörungen wie oben beschrieben treten also nicht in der REM Phase sondern in de Tiefschlafphase auf.\\%weiß nicht ob das paper so viel bringt für uns

- \cite{ferrara2000sleep}\\
Der Schlaf-Wach-Übergang wird als neuer Zustand zwischen Schlafen und dem Wach sein, da keine exakte Grenze gezogen werden kann. Es wird als langsamer komplexer Prozess beschrieben, der gewisse Zeit braucht um abgeschlossen zu werden. Während dieser Übergangszeit, die einige Merkmale sowohl mit dem Wach- als auch mit dem Schlafzustand teilt, ist eine klare Trennung zwischen verschiedenen Parametern (physiologisch, kognitiv und verhaltensbezogen) erkennbar, da sie unterschiedliche Veränderungsraten vom Schlafmuster zum Wachmuster aufweisen. Ein Nutzer kann in einer solchen Situation möglicherweise noch in der Lage sein, einfache Interaktionen durchzuführen, jedoch bei komplexeren Aufgaben treten Schwächen auf, da die Gehirnreaktivität beim Erwachen nicht vollständig present ist. Aus diesen Gründen müssen in solchen Situationen, die eine hochqualifizierte Leistung unmittelbar nach einem abrupten Erwachen erfordern (z.B. medizinisches Notfallmanagement), die unvermeidlichen negativen Auswirkungen im Voraus berücksichtigt werden.Durch Hilfewerkzeuge können solche ungewollten Effekte ausgeglichen werden.\\


- \cite{dinges1985assessing}:\\
Schlafentzug verursacht tiefere Kurzschlaf-Phasen~\cite{dinges1985assessing}. Sollte eine optimale Performance in Aufgaben benötigt werden sollten Kurzschlaf-Phasen vermieden werden~\cite{dinges1985assessing}. Schlummern sowie Nickerchen sollten gemacht werden bevor ein gravierender Schlafentzug eintritt~\cite{dinges1985assessing}.\\
Bei stupiden langanhaltenden Arbeitsvorgängen kann es schnell passieren, dass man müde wird. Umso länger man im gleichbleibenden Trott ist wird der Schlafmangel immer größer. Diese Probleme kann man einerseits durch Nickerchen/schlafen lindern, aber es erhöht das Risiko, dass die Person bei abruptem Erwachen Schwierigkeiten hat, zu funktionieren. Mit unterschiedlichen Schlafentzugszeiten (6, 18, 30, 42 oder 54) wurde eine Studie durchgeführt, um zu testen wie gut die Probanden auf Ereignisse reagieren können. Nach den verschiedenen zeiten war es den Nutzern gestattet 2 Stunden zu schlafen. Schlafentzug erhöhte die Menge an Tiefschlaf in den Nickerchen, was mit einer größeren Abnahme der kognitiven Leistung nach dem Nickerchen verbunden war.\\
Die Manipulation von zunehmenden Schlafmangel führte zu einem tieferen Schlaf, was siginifikante Leistungseinbußen hervorruf. Dies war für die kognitive Leistung am dramatischsten. Die direkt nach dem schlafen auftretende Veriwrrung wird Schlafträgheit genannt. Probanden die 42 bis 54 Stunden lang schlaflos waren wurde diese Schlafträgheit durch tiefgreifende Desorientierung, Unfähigkeit und Verwirrung gekennzeichnet.
\texttt{sabrina fazit: paper is schwer verständlich für mich und Hauptaussagen stehen oben}\\

- \cite{kraemer2000time}:\\
Abhängig von der Tageszeit existieren Unterschiede in der Performance, so wie der Selbsteinschätzung und anderer psychologischer Parameter bei voll ausgeschlafenen Probanden (12 Stunden Schlaf)~\cite{kraemer2000time}.\\

In dieser Studie wurde an verschiedenen Tageszeiten analysiert, wie die Variation der Zeit mit verschiedenen Aufmerksamkeitsindikatoren zu verknüpfen sind. Nach einer mit ausreichend Schlaf erfüllten Nacht wurden Probanden von 7:00 Uhr bis 23 Uhr alle zwei Stunden mit Tests konfrontiert, die verschiedene Kernthemen untersuchen, darunter Reaktionszeit, Pupillometrie, Visualisierung etc. . Alle Probanden waren gesund und hatten einen normalen tageszyklischen Rythmus, der durch Temperatur- und Speichelmessungen gemessen wurde. Augenmerk dieser Studie lag auf der Beobachtung von tageszeitliche Schwankungen. Nach der Analyse wurden in fast allen untersuchten Parametern Unterschiede deutlich. Die Beobachtungsaspekte zeigten bei den physische Parameter wie beispielsweise die Pupillometrie Spiitzenwerte der Wachsamkeit, welche unmittelbar nach dem Aufstehen (um 7:00 Uhr) und wieder um 21:00 Uhr aufwieß. Im Bezug auf die Selbsteinschätzung und die damit verbundenen Tests, wurde ein Höchmaß der Wachsamkeit zwischen 11:00 Uhr und 15:00 Uhr gemessen. Unterschiedliche Aspekte der Aufmerksamkeit folgen also verschiedenen Tageszeiten. Es wird diskutiert, dass diese Ergebnisse auf tagesrythmische und homöostatische Faktoren zurückzuführen sind.\\
Phasische Aufmerksamkeit entsteht, wenn eine Person auf einen plötzlichen, eher alarmierenden Reiz reagiert.\\
Mit einigen Schwankungen, bleibt die Aufmerksamkeit bis zum frühen Morgen vergleichsweise hoch und nimmt ab Nachmittag (15:00 Uhr) stetig ab.\\
Phänomene wie Wachsamkeit, Schläfrigkeit, Leistung, etc. sind keine stabilen Eigenschaften, sondern variieren während der gesamten Lebensdauer. Es kommt stark auf den Tageszeitpunkt und auf die Eigenschaften einzelner Personen (Alter, Schlafverhalten, etc.) an.\\

-\cite{jewett1999time}\\
Wie wir einschlafen: regionale und zeitliche Unterschiede in der elektroenzephalographischen Synchronisation bei Schlafbeginn:
es wurde angenommen, dass das Gehirn spezifische und vorhersagbare Muster von räumlichen und zeitlichen Unterschieden während des Schlafbeginns zeigt.
Es wurden vollständig elektroenzephalographische (EEG) Aufzeichnungen von 40 gesunden Probanden gemacht, um räumliche und zeitliche Veränderungen der EEG-Aktivität vom Schlaf-Wach Übergang zu untersuchen.... OK hier wirklich nur Bio Gelaber was ich nicht verstehe....\\

- \cite{mccartt2000factors}\\
Faktoren, die mit dem Einschlafen am Steuer bei Fernfahrern von Lastkraftwagen verbunden sind: \texttt{nicht sehr viel was ich sinnvoll finde nur Faktoren die wichtig sind:(was man aber schon 100 mal aufgeschriebem hat)} Schlaflosigkeit, Schlafstörungen, zu lange Arbeitszeiten, etc.