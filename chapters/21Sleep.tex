\section{Schlafen}\label{sec:relatedWork.schlafen}

- \cite{dinges1985assessing}:\\

Schlafentzug verursacht tiefere Kurzschlaf-Phasen~\cite{dinges1985assessing}. Sollte eine optimale Performance in Aufgaben benötigt werden sollten Kurzschlaf-Phasen vermieden werden~\cite{dinges1985assessing}. Schlummern sowie Nickerchen sollten gemacht werden bevor ein gravierender Schlafentzug eintritt~\cite{dinges1985assessing}.\\
Bei stupiden langanhaltenden Arbeitsvorgängen kann es schnell passieren, dass man müde wird. Umso länger man im gleichbleibenden Trott ist wird der Schlafmangel immer größer. Diese Probleme kann man einerseits durch Nickerchen/schlafen lindern, aber es erhöht das Risiko, dass die Person bei abruptem Erwachen Schwierigkeiten hat, zu funktionieren. Mit unterschiedlichen Schlafentzugszeiten (6, 18, 30, 42 oder 54) wurde eine Studie durchgeführt, um zu testen wie gut die Probanden auf Ereignisse reagieren können. Nach den verschiedenen zeiten war es den Nutzern gestattet 2 Stunden zu schlafen. Schlafentzug erhöhte die Menge an Tiefschlaf in den Nickerchen, was mit einer größeren Abnahme der kognitiven Leistung nach dem Nickerchen verbunden war.\\
Die Manipulation von zunehmenden Schlafmangel führte zu einem tieferen Schlaf, was siginifikante Leistungseinbußen hervorruf. Dies war für die kognitive Leistung am dramatischsten. Die direkt nach dem schlafen auftretende Veriwrrung wird Schlafträgheit genannt. Probanden die 42 bis 54 Stunden lang schlaflos waren wurde diese Schlafträgheit durch tiefgreifende Desorientierung, Unfähigkeit und Verwirrung gekennzeichnet.
\texttt{sabrina fazit: paper is schwer verständlich für mich und Hauptaussagen stehen oben}\\

- \cite{kraemer2000time}:\\

Abhängig von der Tageszeit existieren Unterschiede in der Performance, so wie der Selbsteinschätzung und anderer psychologischer Parameter bei voll ausgeschlafenen Probanden (12 Stunden Schlaf)~\cite{kraemer2000time}.\\

In dieser Studie wurde an verschiedenen Tageszeiten analysiert, wie die Variation der Zeit mit verschiedenen Aufmerksamkeitsindikatoren zu verknüpfen sind. Nach einer mit ausreichend Schlaf erfüllten Nacht wurden Probanden von 7:00 Uhr bis 23 Uhr alle zwei Stunden mit Tests konfrontiert, die verschiedene Kernthemen untersuchen, darunter Reaktionszeit, Pupillometrie, Visualisierung etc. . Alle Probanden waren gesund und hatten einen normalen tageszyklischen Rythmus, der durch Temperatur- und Speichelmessungen gemessen wurde. Augenmerk dieser Studie lag auf der Beobachtung von tageszeitliche Schwankungen. Nach der Analyse wurden in fast allen untersuchten Parametern Unterschiede deutlich. Die Beobachtungsaspekte zeigten bei den physische Parameter wie beispielsweise die Pupillometrie Spiitzenwerte der Wachsamkeit, welche unmittelbar nach dem Aufstehen (um 7:00 Uhr) und wieder um 21:00 Uhr aufwieß. Im Bezug auf die Selbsteinschätzung und die damit verbundenen Tests, wurde ein Höchmaß der Wachsamkeit zwischen 11:00 Uhr und 15:00 Uhr gemessen. Unterschiedliche Aspekte der Aufmerksamkeit folgen also verschiedenen Tageszeiten. Es wird diskutiert, dass diese Ergebnisse auf tagesrythmische und homöostatische Faktoren zurückzuführen sind.\\
Phasische Aufmerksamkeit entsteht, wenn eine Person auf einen plötzlichen, eher alarmierenden Reiz reagiert.\\
Mit einigen Schwankungen, bleibt die Aufmerksamkeit bis zum frühen Morgen vergleichsweise hoch und nimmt ab Nachmittag (15:00 Uhr) stetig ab.\\
Phänomene wie Wachsamkeit, Schläfrigkeit, Leistung, etc. sind keine stabilen Eigenschaften, sondern variieren während der gesamten Lebensdauer. Es kommt stark auf den Tageszeitpunkt und auf die Eigenschaften einzelner Personen (Alter, Schlafverhalten, etc.) an.

