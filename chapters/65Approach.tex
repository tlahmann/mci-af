\section{Herangehensweise}

\todoTob{SAM ergebnisse bewerten}
\todoSab{Ist VR die richtige Herangehensweise?}
Da VR und AR immer mehr in das Alltagsleben in Zukunft integriert werden sollen, ist es ein interessantes Thema Alltagssituationen wie Schlafen oder gewisse Aufgaben zu erledigen im VR/AR Kontext zu untersuchen. Bei manchen Probanden haben wir gehört, dass sie es sich gut vorstellen könnten eine AR Brille dauerhaft zu tragen, falls dies sich von der Größe und vom Gewicht nicht mehr von normalen Sehstärkebrille unterscheidet. Zu dem jetzigen Zeitpunkt ist dies jedoch noch nicht möglich.
\todoSab{Wann Sollte ein Mensch beim Autofahren geweckt werden? Das vielleicht in schlussfolgerung rein?}
Im Kontext des autonomen Fahrens sollte man beim Aufwecken des Fahrers auf jeden Fall einen gewissen Abstand an Zeit bis hin zum Eingreifen einplanen, da deutlich wurde, dass der Mensch erst vorbereitet werden muss. Durch zahlreiche Studien wurde dieses veränderte Verhalten direkt nach dem Aufwachen auch schon belegt.
