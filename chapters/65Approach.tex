\section{Herangehensweise}

% Ist VR die richtige Herangehensweise?
Durch VR und AR können spezielle Situationen simuliert werden, die man einem Nutzer ohne diese Erweiterungen nicht darbieten könnte. Deshalb war es für uns die beste Möglichkeit ein konkretes Ruheszenario aufzubauen.

% \todoAll{SAM und RSME ergebnisse woanders rein} -> ich finde es hier eigentlich nicht schlecht -tl
% SAM Ergebnisse bewerten
Dadurch, dass der SAM Fragebogen einmal vor und einmal nach der Ruhephase ausgefüllt werden musste, haben wir einen direkten Vergleich. In Abbildung~\ref{fig:samResults} sieht, dass die Werte aller Gruppen gleichbleibend bis etwas niedriger werden. Tendenziell gehen die Werte von Pleasure, Arousal und auch Dominance nach der Ruhephase herunter, da die Probanden nach unserer Studie nach eigenen Angaben ziemlich müde wurden. Diese Müdigkeit trat auch bei nicht eingeschlafenen Personen auf, da ihr Körper durch das Ausruhen runtergefahren ist. 

% RSME 
Die RSME Werte der Probanden fielen sehr deutlich aus. Bis auf einen Ausreißer wurden die zu bearbeitenden Aufgaben als sehr einfach eingestuft und für mit wenig Aufwand bearbeitbar befunden. Das ist darauf zurückzuführen, dass unsere Aufgaben bewusst einfach gehalten wurden und wir entsprechende direkte Aspekte abgefragt haben, wie beispielweise das räumliche Denken in Aufgabe drei. Nach einigen Angaben der Studienteilnehmern waren die Aufgaben dadurch, dass man sie aus einer Ruhephase gerissen hatte, schwieriger zu lösen als wenn man ihnen die Aufgaben während der Arbeitszeit gestellt hätte, was zu erwarten war. Als Anregung für Zukunftsprojekte könnten diese Aufgaben schwieriger gestalten werden, um mehr 'Nachdenken' der Nutzer zu provozieren.
