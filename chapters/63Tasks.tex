\section{Aufgaben}

\todoAll{Warum gab es nicht so viele Fehler? (Oder so viele Fehler?) im Bezug auch zur Zeit sehen}
Die Anzahl der Fehler hängt stark von der Bearbeitungszeit ab (vermute ich -> nachschauen!?). 
Um so länger die Teilnehmer für eine Aufgabe gebraucht haben, desto besser waren die Ergebnisse. 
So machten Nutzer, die die Zahlordnungsaufgabe langsamer gemacht haben, weniger Fehler. 
Die Farbenaufgabe wurde meist komplett richtig oder komplett falsch gemacht, da die Nutzer teilweise die Aufgabe genau andersherum verstanden hatten und wählten so nicht die Farbe, die die Schrift anzeigte sondern die Farbe in der die Schrift dargestellt war.

\todoTob{Sind die Aufgaben nicht passend für die Studie?}
Manche Probanden sagten im Anschluss an die Studie, dass sie die Aufgaben zu einfach fanden. 
Wir wählten jedoch bewusst einfache Aufgaben, die explizites Können abfragen, wie zB. das Räumliche Denken. 
Auffällig war auch, dass viele Teilnehmer wie erwartet kleine Fehler machten, die sie nach eigener Einschätzung im komplett Wachen Zustand nicht machen würden. 

\todoAll{Diskutieren warum die Nurtzer eine bestimmte Zeit für eine Aufgabe gebraucht haben. Ich finde bspw. interessant, dass die Aufgabe 1 so ein Zick-Zack macht...}
\todoLuc{Hätte man mit der Aufweckdauer spielen können (länger/kürzer Licht)}
% Welche Töne sind effektiver als andere?
Für die Teilnehmer der Studie, die mit einem Alarm-Ton geweckt wurden, haben wir den Alarm-Ton eines Automobilherstellers gewählt, wie er in aktuellen Modellen auch verwendet wird.
Dieser Ton wurde von uns im Vorfeld nicht ausreichend evaluiert, was manche Aussagen der Teilnehmer widerspiegeln.
Diese sagen, dass der Ton entweder erschreckend, oder sogar unangenehm war.
Hierbei spielt die persönliche Präferenz von Menschen auch eine große Rolle. So kann argumentiert werden, dass manche auch morgens einen lauten, schrillen Weck-Ton benötigen um aufzuwachen, während andere sogenannte Licht-Wecker bevorzugen. Eine Möglichkeit wäre, ähnlich der persönlichen Auswahl der Ambient-Musik, diesen Ton auch frei wählbar zu gestalten. 

% Ist eine Kombination der Parameter sinnvoll?
Im Zuge dessen könnte auch die Auswahl der Art des Weckens durch den Benutzer ein interessanter Untersuchungspunkt. Wir haben uns gegen diese Herangehensweise entschieden, da wir vergleichbare Ergebnisse und vor Allem vergleichbare Gruppengrößen angestrebt haben.

\todoLuc{Wie viele Aufgaben hätten es sein können?}
