\section{Aufgaben}

% Warum gab es nicht so viele Fehler? (Oder so viele Fehler?) im Bezug auch zur Zeit sehen
Betrachtet man Abbildung~\ref{fig:orderingMistakeTimeScatterplot}, so fällt auf, dass alle Gruppen in etwa gleich bei der Verteilung der Fehler gewesen sind. 
Ein leichter Trend lässt sich erkennen insoweit, dass die Fehlerrate mit zunehmender Zeit abnimmt. 
Die Schlussfolgerung aus der Visualisierung könnte also sein, dass Teilnehmer, die länger für die Erledigung einer Aufgabe gebraucht haben auch erfolgreicher waren. Allerdings lassen sich in dieser Aufteilung und mit unserer Herangehensweise keine statistisch signifikanten unterschiede erkennen. 
Die Probanden haben also nicht besser oder schlechte abgeschnitten, unabhängig davon wie sie aus der Ruhephase geholt wurden. 
Die gleichen Tendenzen lassen sich hier auch für die zweite und dritte Aufgabe erkennen. So sind die Abbildungen~\ref{fig:matchingMistakeTimeScatterplot} und~\ref{fig:countingMistakeTimeScatterplot} auf die gleiche Art zu interpretieren.
Die statistische Auswertung liefert aber auch hier keine statistisch signifikanten Unterschiede zwischen den Gruppen.

Für die Bearbeitung der Aufgabe eins und zwei existieren Ausreißer, die widerspiegeln, dass die Probanden die Aufgabe in exakt umgekehrter Weise durchgeführt haben, wie von uns vorgesehen. 
Dabei ist Interessant, dass die Gruppe, die mit einem Alarm-Ton geweckt wurde die erste Aufgabe eher missverstanden hat und beinahe alle Gruppen Ausreißer in Aufgabe zwei aufweisen.
Hier könnte die ungeschickte Formulierung der Aufgabenbeschreibung, also des Textes am Anfang der Aufgabe, aber auch andere Faktoren, wie Müdigkeit oder Verwirrung der ausschlaggebende Faktor der Ergebnisse, wie wir sie haben, sein. 
Eine genaue Aussage über die Gründe lässt sich mit unseren Ergebnissen leider nicht treffen.

% Sind die Aufgaben nicht passend für die Studie?
Manche Probanden sagten im Anschluss an die Studie, dass sie die Aufgaben zu einfach fanden. 
Wir wählten jedoch bewusst einfache Aufgaben, die explizites Können abfragen, wie zB. das Räumliche Denken. 
Auffällig war auch, dass viele Teilnehmer wie erwartet kleine Fehler machten, die sie nach eigener Einschätzung im komplett Wachen Zustand nicht machen würden. 

\todoLuc{Wie viele Aufgaben hätten es sein können?}

\todoAll{Diskutieren warum die Nurtzer eine bestimmte Zeit für eine Aufgabe gebraucht haben. Ich finde bspw. interessant, dass die Aufgabe 1 so ein Zick-Zack macht...}
\todoLuc{Hätte man mit der Aufweckdauer spielen können (länger/kürzer Licht)}

% Welche Töne sind effektiver als andere?
Für die Teilnehmer der Studie, die mit einem Alarm-Ton geweckt wurden, haben wir den Alarm-Ton eines Automobilherstellers gewählt, wie er in aktuellen Modellen auch verwendet wird.
Dieser Ton wurde von uns im Vorfeld nicht ausreichend evaluiert, was manche Aussagen der Teilnehmer widerspiegeln.
Diese sagen, dass der Ton entweder erschreckend, oder sogar unangenehm war.
Hierbei spielt die persönliche Präferenz von Menschen auch eine große Rolle. So kann argumentiert werden, dass manche auch morgens einen lauten, schrillen Weck-Ton benötigen um aufzuwachen, während andere sogenannte Licht-Wecker bevorzugen. Eine Möglichkeit wäre, ähnlich der persönlichen Auswahl der Ambient-Musik, diesen Ton auch frei wählbar zu gestalten. 

% Ist eine Kombination der Parameter sinnvoll?
Im Zuge dessen könnte auch die Auswahl der Art des Weckens durch den Benutzer ein interessanter Untersuchungspunkt. Wir haben uns gegen diese Herangehensweise entschieden, da wir vergleichbare Ergebnisse und vor Allem vergleichbare Gruppengrößen angestrebt haben.
