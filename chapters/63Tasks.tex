\section{Aufgaben}

% Warum gab es nicht so viele Fehler? (Oder so viele Fehler?) im Bezug auch zur Zeit sehen
Betrachtet man Abbildung~\ref{fig:orderingMistakeTimeScatterplot}, so fällt auf, dass alle Gruppen in etwa gleich bei der Verteilung der Fehler gewesen sind. 
Ein leichter Trend lässt sich erkennen insoweit, dass die Fehlerrate mit zunehmender Zeit abnimmt. 
Die Schlussfolgerung aus der Visualisierung könnte also sein, dass Teilnehmer, die länger für die Erledigung einer Aufgabe gebraucht haben auch erfolgreicher waren. Allerdings lassen sich in dieser Aufteilung und mit unserer Herangehensweise keine statistisch signifikanten unterschiede erkennen. 
Die Probanden haben also nicht besser oder schlechte abgeschnitten, unabhängig davon wie sie aus der Ruhephase geholt wurden. 
Die gleichen Tendenzen lassen sich hier auch für die zweite und dritte Aufgabe erkennen. So sind die Abbildungen~\ref{fig:matchingMistakeTimeScatterplot} und~\ref{fig:countingMistakeTimeScatterplot} auf die gleiche Art zu interpretieren.
Die statistische Auswertung liefert aber auch hier keine statistisch signifikanten Unterschiede zwischen den Gruppen.

Für die Bearbeitung der Aufgabe eins und zwei existieren Ausreißer, die widerspiegeln, dass die Probanden die Aufgabe in exakt umgekehrter Weise durchgeführt haben, wie von uns vorgesehen. 
Dabei ist Interessant, dass die Gruppe, die mit einem Alarm-Ton geweckt wurde die erste Aufgabe eher missverstanden hat und beinahe alle Gruppen Ausreißer in Aufgabe zwei aufweisen.
Hier könnte die ungeschickte Formulierung der Aufgabenbeschreibung, also des Textes am Anfang der Aufgabe, aber auch andere Faktoren, wie Müdigkeit oder Verwirrung der ausschlaggebende Faktor unserer Ergebnisse sein. 
Eine genaue Aussage über die Gründe lässt sich mit unseren Ergebnissen leider nicht treffen, könnte aber in zukünftigen Forschungsprojekten weiter untersucht werden.

% Diskutieren warum die Nutzer eine bestimmte Zeit für eine Aufgabe gebraucht haben. Ich finde bspw. interessant, dass die Aufgabe 1 so ein Zick-Zack macht...
Bei der Erledigung von Aufgabe eins kann ein interessantes Phänomen gesehen werden. Hierbei zeigt die Abbildung~\ref{fig:timeTask1} das Verhalten der Benutzer beim Bearbeiten der einzelnen Stufen. 
Im ersten Schritt gehen wir davon aus, dass der Teilnehmer eine bestimmte Zeit braucht um alle gegebenen Objekte zu überfliegen und sich einen ersten Eindruck zu machen. 
Der Teilnehmer schaut sich im Zuge dessen möglicherweise jedes Objekt an und überprüft im mentalen Modell, ob eine Zahl größer oder kleiner ist. 
Hat der Teilnehmer dann alle Objekte betrachtet wird das Objekt mit der kleinsten Zahl, entsprechend der Aufgabenstellung, ausgewählt. 
Die nächste Zahl in der Reihenfolge könnte zu diesem Zeitpunkt noch im Kurzzeitgedächtnis des Probanden gespeichert sein und wird entsprechend schneller ausgewählt. 
Nach diesen ersten zwei ausgewählten Objekten fängt dieser Prozess von neuem an und wiederholt sich zwei weitere Male bis nur noch vier bis fünf Objekte verbleiben. 
Diese zwei Wiederholungen haben allerdings in allen Gruppen eine geringere durchschnittliche Bearbeitungszeit, was auf die sinkende  Anzahl von Objekten zurückgeführt werden kann.
Die letzten fünf Objekte könnten hier der "`Magic Number Seven"'~\cite{saaty2003magic} entsprechen, also der Annahme, dass $7\pm 2$ Objekte im Kurzzeitgedächtnis gespeichert werden können.
Auch das sogenannte "`Subitizing"'~\cite{mandler1982subitizing} könnte eine Rolle spielen, also das Erkennen von fünf gleichen Objekten.
Weitere Forschung kann hier die Anzahl der Objekte variieren um diese Phänomene zu untersuchen.

% Sind die Aufgaben nicht passend für die Studie?
Abgesehen von den Missverständnissen mancher Teilnehmer kann es durchaus sein, dass zu wenige Fehler passiert sind bei der Erledigung der Aufgaben um valide Aussagen über unterschiede Treffen zu können.
Die Aussagen mancher Probanden zeigen, dass die Aufgaben nicht anspruchsvoll genug waren um wirkliche kognitive Anstrengung hervorzurufen. 
Wir wählten für unsere Zwecke bewusst kurze, weniger anstrengende Aufgaben, die explizite Fähigkeiten, wie räumliches Denken oder Vergleichsprozesse abfragen. Dies geschah, da wir für die hier untersuchten Grundlagen zu wenige Referenzen auffinden konnten. 
Als eine Verbesserung hierzu kann die Aufgabenstellung beispielsweise zeitlich begrenzt werden. Dies würde ebenfalls der von uns definierten Aufgabenstellung entgegenkommen, dass Fahrer eines autonomen Fahrzeugs möglichst schnell eine Situation einschätzen sollen. Andererseits wollen wir auch argumentieren, dass der Zeitpunkt, wann der Fahrer auf eine Aufgabe vorbereitet werden sollte möglichst nicht zu knapp vor der Aufgabe selbst sein sollte. 
Auffällig war auch, dass viele Teilnehmer kleine Fehler machten, die sie nach eigener Einschätzung im komplett Wachen Zustand nicht machen würden. 

% \todoLuc{Wie viele Aufgaben hätten es sein können?}

% \todoLuc{Hätte man mit der Aufweckdauer spielen können (länger/kürzer Licht)}

% Welche Töne sind effektiver als andere?
Für die Teilnehmer der Studie, die mit einem Alarm-Ton geweckt wurden, haben wir den Alarm-Ton eines Automobilherstellers gewählt, wie er in aktuellen Modellen auch verwendet wird.
Dieser Ton wurde von uns im Vorfeld nicht ausreichend evaluiert, was manche Aussagen der Teilnehmer widerspiegeln.
Diese sagen, dass der Ton entweder erschreckend, oder sogar unangenehm war.
Hierbei spielt die persönliche Präferenz von Menschen auch eine große Rolle. So kann argumentiert werden, dass manche auch morgens einen lauten, schrillen Weck-Ton benötigen um aufzuwachen, während andere sogenannte Licht-Wecker bevorzugen. Eine Möglichkeit wäre, ähnlich der persönlichen Auswahl der Ambient-Musik, diesen Ton auch frei wählbar zu gestalten. 

% Ist eine Kombination der Parameter sinnvoll?
Im Zuge dessen könnte auch die Auswahl der Art des Weckens durch den Benutzer ein interessanter Untersuchungspunkt. Wir haben uns gegen diese Herangehensweise entschieden, da wir vergleichbare Ergebnisse und vor Allem vergleichbare Gruppengrößen angestrebt haben.
