\section{Aufgaben}

\todoAll{Warum gab es nicht so viele Fehler? (Oder so viele Fehler?) im Bezug auch zur Zeit sehen}
Die Anzahl der Fehler hängt stark von der Bearbeitungszeit ab (vermute ich -> nachschauen!?). Um so länger die Teilnehmer für eine Aufgabe gebraucht haben, desto besser waren die Ergebnisse. So machten Nutzer, die die Zahlordnungsaufgabe langsamer gemacht haben, weniger Fehler. Die Farbenaufgabe wurde meist komplett richtig oder komplett falsch gemacht, da die Nutzer teilweise die Aufgabe genau andersherum verstanden hatten und wählten so nicht die Farbe, die die Schrift anzeigte sondern die Farbe in der die Schrift dargestellt war.

\todoTob{Sind die Aufgaben nicht passend für die Studie?}
Manche Probanden sagten im Anschluss an die Studie, dass sie die Aufgaben zu einfach fanden. Wir wählten jedoch bewusst einfache Aufgaben, die explizites Können abfragen, wie zB. das Räumliche Denken. Auffällig war auch, dass viele Teilnehmer wie erwartet kleine Fehler machten, die sie nach eigener Einschätzung im komplett Wachen Zustand nicht machen würden. 

\todoAll{Diskutieren warum die Nurtzer eine bestimmte Zeit für eine Aufgabe gebraucht haben. Ich finde bspw. interessant, dass die Aufgabe 1 so ein Zick-Zack macht...}
\todoLuc{Hätte man mit der Aufweckdauer spielen können (länger/kürzer Licht)}
\todoTob{Welche Töne sind effektiver als andere?}
\todoLuc{Ist eine Kombination der Parameter sinnvoll?}
\todoLuc{Wie viele Aufgaben hätten es sein können?}
