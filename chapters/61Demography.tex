\section{Demografie}

Die Studienteilnehmer waren zu über 70\% männlich, was sich auf den Anteil der männlichen und weiblichen Studierenden in den Studiengängen zurückführen lässt. Diese Verteilung zeigt allerdings keine statistisch signifikanten Unterschiede zwischen den männlichen und weiblichen Teilnehmern.
Da die Durchführung der Studie an der Uni stattgefunden hat haben wir eine relativ kleine Bandbreite von Teilnehmern abgedeckt. Dies betrifft sowohl den Studiengang, wie auch das Alter der Teilnehmer. Außerdem weisen alle Teilnehmer dadurch einen höheren Bildungsstand auf. 
Um den Bogen zum Durchschnitt der Gesellschaft zu spannen wäre es interessant die Studie mit mehr Teilnehmern zu wiederholen um Faktoren, wie Alter und Bildung besser vergleichen zu können.
Außerdem ist ein interessanter Faktor, der in dieser Studie nicht untersucht werden konnte, die Erfahrung mit Medien und Geräten im allgemeinen. 
Durch die moderne Gesellschaft werden Nutzer heute mit neuen Technologien überschüttet. Außerdem haben viele Menschen standardgemäß ein Smartphone, wodurch die Erfahrung mit Spielen, sozialen Netzwerken und technischen Vorgängen weiter gefördert wird.

Auf der anderen Seite haben viele der Teilnehmer möglicherweise noch keine oder nur wenig Erfahrung mit dem Autofahren allgemein. Wir gehen außerdem davon aus, dass die Teilnehmer keine oder nur sehr wenig Erfahrung mit autonomen Fahrzeugen gemacht haben. Durch diese beiden Punkte argumentieren wir ebenfalls, dass die Studie in Zukunft, sobald autonomes Fahren etabliert ist, mit einer größeren Anzahl an Probanden andere Ergebnisse liefern könnte. 

Da VR und AR zur heutigen Zeit noch kein etablierter Bestandteil des Alltags sind, war die Altersgruppe allerdings auch nicht nachteilig, da manche Probanden explizit selbst auch mit VR und AR an der Universität arbeiten und sich so gut in unseren Studienkontext einfühlen konnten.
Zusätzlich erleichterte uns gerade die Erfahrung der Teilnehmer mit Technik eine relativ sorgenfreie Studiendurchführung. Dies ist der Tatsache geschuldet, dass die Teilnehmer weniger Probleme im Umgang mit der von uns gewählten Technik hatten und sich schnell mit unserer Studienumgebung und der Handhabung vertraut gemacht haben.

Die AR/VR Erfahrung der Teilnehmer spielte für uns keine gesonderte Rolle, wir fassten diesen Punkt jedoch mit auf und stellen fest, dass unsere Probanden sehr unterschiedliche Erfahrungen hatten, wie in Abbildung~\ref{fig:expVr} und~\ref{fig:expAr} gesehen werden kann. Es ist zu bemerken, dass die Erfahrung mit AR hierbei allerdings noch geringer war, was eine Nutzung von Head-Up Displays in Autos als weiteren spannenden Untersuchungspunkt für zukünftige Projekte hervorhebt. 
Bei der Durchführung der Studie war auffällig, dass die Probanden, die zum ersten Mal VR nutzten sehr neugierig waren, viel mit dem Equipment herumgespielt haben und nur schwierig zur Ruhe kamen.