\chapter{Verwandte Forschung}\label{sec:relatedWork}
Um einen Überblick über die Thematik zu erlangen untersuchten wir die bestehende verwandte Forschung. Dies haben wir in einzelne Teilgebiete unterteilt. Zuerst betrachteten wir den Schlafzustand, aus welchem heraus wir unsere Untersuchungen durchführen wollten. Anschließend das unmittelbar auf den Schlaf folgende Aufwachen und als dritten Abschnitt das Erledigen von anfallenden Aufgaben nach den ersten zwei Phasen.

Maybe read:

\url{https://www.spiegel.de/wissenschaft/mensch/muedes-gehirn-der-tote-punkt-nach-dem-aufwachen-a-394613.html}
\url{https://www.spiegel.de/wissenschaft/mensch/uebermuedetes-hirn-nervenzellen-melden-sich-zur-mini-siesta-ab-a-759453.html}
\url{https://www.futurezone.de/science/article215969833/Muede-Kein-Wunder-dein-Hirn-braucht-ewig-um-wach-zu-werden.html}
\url{https://www.focus.de/wissen/experten/nadja_podbregar/schlafmangel-foerdert-falsche-erinnerungen-warum-uebernaechtigte-augenzeugen-unzuverlaessiger-sind_id_4012940.html}
