\chapter{Verwandte Forschung}\label{sec:relatedWork}
\todoAll{Mehr paper etc finden!!}
Um einen Überblick über die Thematik zu erlangen untersuchten wir die bestehende verwandte Forschung. Dies haben wir in einzelne Teilgebiete unterteilt. Zuerst betrachten wir den Schlafzustand, aus welchem heraus wir unsere Untersuchungen durchführen wollen. Anschließend das unmittelbar auf den Schlaf folgende Aufwachen und was damit einher geht. Als dritten Abschnitt schauen wir das Erledigen von anfallenden Aufgaben an und wie der Mensch damit umgeht. Zuletzt fokussieren wir uns auf den VR Kontext und untersuchen vorallem wie der Nutzer mit der Grenze aus Realität und virtueller Realität umgeht.\\

\todoTob{die Einfügen in die Bib?}

 -\url{https://www.spiegel.de/wissenschaft/mensch/muedes-gehirn-der-tote-punkt-nach-dem-aufwachen-a-394613.html} :
 
 In den ersten Minuten nach dem Aufwachen können Probanden selbst einfache Aufgaben, wie zum Beispiel kleine Rechnungen, nicht effektiv umsetzen. Die Länge der Schlafdauer hat dabei keine Auswirkung auf das Ergebniss. Man spricht auch vom 'toten Punkt nach dem Aufwachen'. Nach einem einwöchigen Untersuchen der Probanden, konnte man erkennen, dass die Nutzer eine Minute nach dem Aufwachen gerade einmal 65 Prozent ihrer sonstigen Leistungsfähigkeit erreichten. Nach 26 Stunden ohne Schlaf schnitten sie mit ca 85 Prozent ihrer Maximalpunktzahl signifikant besser ab. Forscher gingen bislang davon aus, dass wichtige Aufgaben, wie zum Beispiel Bereitschaftsdienst, nach Schlafentzug zu erledigen deutlich gefährlicher ist, als direkt nach dem Aufwachen, jedoch kann dies hier nicht bestätigt werden.
 Schlafentzug kann ähnliche Auswirkungen wie Alkoholkonsum haben.
 Die Leistungsfähigkeit nach dem Aufwachen normalisiert sich zwischen 20 bis 30 Minuten vollständig, jedoch können manche Einschränkung aber auch bis zu einer Stunde anhalten. \\

-\url{https://www.spiegel.de/wissenschaft/mensch/uebermuedetes-hirn-nervenzellen-melden-sich-zur-mini-siesta-ab-a-759453.html}

In dieser Studie wurden nur motorische Fähigkeiten und die dabei beteiligten Regionen des Hirns untersucht. Bei Müdigkeit setzen gewisse Teile des Denkorgans ab, wobei andere Abteile versuchen normal weiter zu arbeitem. Die Müdigkeit führt zu kleine Ausfällen in gewissenn Hirnregionen. An kleinen Tierversuchen wurde gemessen, dass die Hirnrinde bei starker Müdigkeit unabhängig von der Umgebung in einen inaktiven Zustand fallen können und die Tiere wach erschienen, jedoch nach langem Wachhalten signifikant unterschiedliches Verhalten zum Normalzustand aufwiesen. Trotz Schwerpunkt auf motorische Untersuchungen, wird davon ausgegangen, dass diese Ergebnisse auf Wahrnehmung und Denkvermögen teilweise übertragen werden kann.\\

-\url{https://www.futurezone.de/science/article215969833/Muede-Kein-Wunder-dein-Hirn-braucht-ewig-um-wach-zu-werden.html}

Müdigkeiten kann Schlaftrunkenheit verursachen, was die Hauptursache für Beeinträchtigungen nachdem Aufstehen morgens ist. Der Mensch kann schwer von einem aktiven in einen ruhenden Zustand zu wechseln und auch andersrum. Am morgen können diese beiden Aktionen nicht unterschieden werden und so kommt es zu körperlichen Unstimmigkeiten und der Mensch ist von der Müdigkeit beeinträchtigt. Wieder wurden Matheaufgaben an Nutzern, die gerade aufgewacht sind getestet und wieder stellt man fest, dass die Probanden deutlich schlechter als in einem wachen Zustand, der nach einer Stunde nach dem Aufwachen gemessen wurde, abschnitten. Es wird vermutet, dass das Gehirn zwischen den beiden Modi gefangen ist und die abnehmende Leistungsfähigkeit darauf zurückzuführen ist.\\

-\url{https://www.focus.de/wissen/experten/nadja_podbregar/schlafmangel-foerdert-falsche-erinnerungen-warum-uebernaechtigte-augenzeugen-unzuverlaessiger-sind_id_4012940.html}

Wenn man sich an Ereignisse erinnert, dann sind es meist Erinnerungen in denen man vollkommend wach und zurechnungsfähig war. In einem müden Zustand kann unser Gehirn schwieriger Vorkommnisse und Details abspeichern. Diese Erkenntnis kann zu einigen Problemen führen, wie beispielsweise bei einem Autounfall der morgens passiert oder Verbrechen wo ein Augenzeuge wichtig ist. Durch ein Experiment wurden über 100 Probanden Bilder von einem Diebstahl in unterschiedlichen Momenten gezeigt. Die Gruppen gleiderten sich in 1) abends und schlafen, 2) abends und die Nacht durchmachen, 3) morgens und geschlafen und 4) morgens und die Nacht durchgemacht. Die Ergebnisse nach einem Fragebogen wiesen auf, dass die Probanden, die Fotos nach der durchwachten Nacht gesehen hatten, bei ihren Erinnerungen deutlich häufiger falsch lagen als ihre ausgeschlafenen Kollegen. Hatten sie dagegen das Foto vor der schlaflosen Nacht gesehen, war ihre Erinnerung daran sehr viel besser. Nach Ansicht der Forscher zeigt dies, dass es eine große Rolle spielen kann, ob ein Zeuge zum Zeitpunkt seiner Beobachtung ausgeschlafen war oder nicht. Durch Schlafmangel könne also weniger Details aufgenommen werden und falsche Erinnerungen aufzunehmen wird gefordert, was beispielweise in der Kriminalistik schwere Folgen haben kann.