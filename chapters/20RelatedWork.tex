\chapter{Verwandte Forschung}\label{sec:relatedWork}
\todoAll{Mehr paper etc finden!!}
Um einen Überblick über die Thematik zu erlangen untersuchten wir die bestehende verwandte Forschung. Dies haben wir in einzelne Teilgebiete unterteilt. Zuerst betrachten wir den Schlafzustand, aus welchem heraus wir unsere Untersuchungen durchführen wollen. Anschließend das unmittelbar auf den Schlaf folgende Aufwachen und was damit einher geht. Als dritten Abschnitt schauen wir das Erledigen von anfallenden Aufgaben an und wie der Mensch damit umgeht. Zuletzt fokussieren wir uns auf den VR Kontext und untersuchen vorallem wie der Nutzer mit der Grenze aus Realität und virtueller Realität umgeht.\\

\todoTob{hab es gereadet wie soll man die Einfügen in die Bib?}

 -\url{https://www.spiegel.de/wissenschaft/mensch/muedes-gehirn-der-tote-punkt-nach-dem-aufwachen-a-394613.html} :
 
 In den ersten Minuten nach dem Aufwachen können Probanden selbst einfache Aufgaben, wie zum Beispiel kleine Rechnungen, nicht effektiv umsetzen. Die Länge der Schlafdauer hat dabei keine Auswirkung auf das Ergebniss. Man spricht auch vom 'toten Punkt nach dem Aufwachen'. Nach einem einwöchigen Untersuchen der Probanden, konnte man erkennen, dass die Nutzer eine Minute nach dem Aufwachen gerade einmal 65 Prozent ihrer sonstigen Leistungsfähigkeit erreichten. Nach 26 Stunden ohne Schlaf schnitten sie mit ca 85 Prozent ihrer Maximalpunktzahl signifikant besser ab. Forscher gingen bislang davon aus, dass wichtige Aufgaben, wie zum Beispiel Bereitschaftsdienst, nach Schlafentzug zu erledigen deutlich gefährlicher ist, als direkt nach dem Aufwachen, jedoch kann dies hier nicht bestätigt werden.
 Schlafentzug kann ähnliche Auswirkungen wie Alkoholkonsum haben.
 Die Leistungsfähigkeit nach dem Aufwachen normalisiert sich zwischen 20 bis 30 Minuten vollständig, jedoch können manche Einschränkung aber auch bis zu einer Stunde anhalten. \\

\url{https://www.spiegel.de/wissenschaft/mensch/uebermuedetes-hirn-nervenzellen-melden-sich-zur-mini-siesta-ab-a-759453.html}


\url{https://www.futurezone.de/science/article215969833/Muede-Kein-Wunder-dein-Hirn-braucht-ewig-um-wach-zu-werden.html}
\url{https://www.focus.de/wissen/experten/nadja_podbregar/schlafmangel-foerdert-falsche-erinnerungen-warum-uebernaechtigte-augenzeugen-unzuverlaessiger-sind_id_4012940.html}
