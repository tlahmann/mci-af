\chapter{Verwandte Forschung}\label{sec:relatedWork}
\todoAll{Mehr paper etc finden!!}
Um einen Überblick über die Thematik zu erlangen untersuchten wir die bestehende verwandte Forschung. Dies haben wir in einzelne Teilgebiete unterteilt. Zuerst betrachten wir den Schlafzustand, aus welchem heraus wir unsere Untersuchungen durchführen wollen. Anschließend das unmittelbar auf den Schlaf folgende Aufwachen und was damit einher geht. Als dritten Abschnitt schauen wir das Erledigen von anfallenden Aufgaben an und wie der Mensch damit umgeht. Zuletzt fokussieren wir uns auf den VR Kontext und untersuchen vorallem wie der Nutzer mit der Grenze aus Realität und virtueller Realität umgeht.

Maybe read:

\url{https://www.spiegel.de/wissenschaft/mensch/muedes-gehirn-der-tote-punkt-nach-dem-aufwachen-a-394613.html}
\url{https://www.spiegel.de/wissenschaft/mensch/uebermuedetes-hirn-nervenzellen-melden-sich-zur-mini-siesta-ab-a-759453.html}
\url{https://www.futurezone.de/science/article215969833/Muede-Kein-Wunder-dein-Hirn-braucht-ewig-um-wach-zu-werden.html}
\url{https://www.focus.de/wissen/experten/nadja_podbregar/schlafmangel-foerdert-falsche-erinnerungen-warum-uebernaechtigte-augenzeugen-unzuverlaessiger-sind_id_4012940.html}
