\section{Übergänge in und aus der Virtuellen Realität}\label{sec:relatedWork.vr}

\todoTob{Lesen und mgwl umschreiben}

Die Studie \texttt{The Dream is Collapsing} im Bereich Virtual Reality (VR) hat die Erfahrungen der Nutzer mit Immersion, Präsenz, Simulatorkrankheiten und Lerneffekten untersucht. Die momentane Erfahrungen, VR zu verlassen und in die Realität zurückzukehren, sind noch nicht gut erörtert.\\
Der Akt des Ein- und Ausstiegs aus VR also der Moment des An- und Ausziehens des Headsets spielt eine wichtige Rolle in der gesamten Benutzererfahrung, erhält aber wenig Aufmerksamkeit. Eine Erkenntnis ist, dass der 'Moment des Ausstiegs' eine ungenutzte Gelegenheit sein kann bestimmte Auswirkungen beim Nutzer herzvorzurufen. Die Konstruktion für diesen Moment könnte zu vielen Zwecken genutzt werden. Zum Beispiel könnten Designer jede Überraschung beim Entfernen des Headsets verringern, indem sie den Benutzer auf alle Änderungen in der realen Umgebung aufmerksam machen bzw. darauf eingehen, um den Übergang möglichst 'unauffällig' zu gestalten. So könnte sich beispielsweise eine VR-Anwendung an das verblassende Licht in der Realität anpassen.
Andernfalls könnten gewünschte 'harte cuts' für bestimmte Szenarien konstruiert werden, wie zum Beispiel Horror Spiele oder das Erzwingen bestimmter Angstsituationen.\\
In der Studie werden vier Gruppen gebildet, die einzelne Bereiche abdecken: Gaming, Illusionen, Wahrnehmungsverfälschung und kognitive Aufgaben. \\

\begin{itemize}
	\item Gaming: Ein großer Teil der kommerziellen Bemühungen der VR-Entwicklung richtet sich auf Unterhaltung und Gaming. Gaming verwendet eine große Anzahl von Mechaniken, um immersive Spiele zu entwickeln.
	\item Illusion: In letzter Zeit wurde die Aufmerksamkeit auf VR-Illusionen erhöht. Mit Schwerpunkt auf haptische Illusionen und wandelnde Illusionen. Der Moment des Ausstiegs aus Illusionen ist besonders interessant, da er dem Moment entspricht, in dem die Nutzer erkennen, dass sie einer Illusion ausgesetzt waren und sich schnell innerhalb der realen Umgebung neu orientieren müssen.
	\item Wahrnehmungsverfälschung: Es gibt Veränderungen in der Wahrnehmung, die sich daraus ergeben, dass Menschen unterschiedliche körperliche Merkmale haben, wie bspw. das Alter oder die Herkunft. Ein wichtiger Aspekt stellt auch die Körpergröße dar. In diesem Fall angelehnt an den Proteus-Effekt, bei dem Teilnehmer unterschiedlicher Größe unterschiedliche Reaktionen auf Reize aufzeigen. Der Fokus liegt hier auf der Erforschung der Umgebung und der Erledigung einer Aufgabe aus einer anderen Höhenperspektive. Der Moment des Ausstiegs beinhaltet sowohl eine Körperneuausrichtung, als auch eine perspektivische Neuausrichtung und Reflexion.
	\item Kognitive Aufgaben: Virtuelle Umgebungen werden unter Anderem als Lehrmittel verwendet und gewisse Literatur hat den Lerneffekt von VR untersucht. Die Hervorhebung des Lernens und der Bildung stellt eine kognitive Belastung für den Nutzer dar. Der Zeitpunkt des Ausstiegs, in welchem diese Last abgebaut wird, kann zu einer anderen Benutzererfahrung beim Verlassen der virtuellen Umgebung führen.
\end{itemize}

Durch eine thematischen Analyse kristalisierten sich fünf Aspekte hervor: Raum, Kontrolle, Zeit, Sozialität und sensorische Anpassung begutachtet.\\
Ein auffälliger Punkt ist die räumliche Desorientierung unabhängig von der Komplexität der VR-Szene.\\ 
 Die Teilnehmer beschreiben einen Übergang vom VR-Verlassen hin zur Realität, zum Beispiel zuerst mental und dann physisch.
Bisher war die Erfahrung von VR ausschließlich innerhalb des VR-Headsets gebunden, aber der Moment des Ausstiegs kann eine Gelegenheit darstellen, diese strikte Grenze zwischen virtueller und realer Welt in Frage zu stellen.\\
Teilnehmer beschreiben Veränderungen in ihrem Kontrollgefühl als sie den Übergang zwischen VR und Realität vollzogen haben. Nutzer fühlen sich unter Anderem erleichtert nachdem sie die VR Brille abgenommen haben und meinten dass diese 'Welt' vertrauensvoller ist. Es hat mit Sicherheit zu tun, die dadurch entsteht, dass man weiß, dass das Headset entfernt werden kann, um sofort in die reale Umgebung zurückzukehren.\\
Zudem fühlen sich Probanden schnell desorientiert durch Unterschiede der VR zur Realität, die noch nicht umgesetzt werden können, wie zum Beispiel nicht statische Objekte und Umgebungen. Der globale Orientierungssinn kann also durch statische Ausrichtung der VR Elemente beibehalten werden.
Bis heute existiert eine Lücke zwischen dem menschlichen Verständnis von VR-Erfahrungen im Moment des Ausstiegs aus VR, der wirklichen Erfahrung in VR und dem Auftreten von VR-Nachwirkungen. \cite{knibbe2018dream}\\


-\cite{bonanni2005attention}\\
AR kann im Bezug auf Aufgabenbewältigung herangezogen werden. 
Die Objekte und Oberflächen einer Umgebung können mit digitalen Schnittstellen überlagert werden, um sie einfacher und sicherer für anstehende Aufgaben zu gestalten. Sobald Informationen überall im Raum projiziert werden können, ist es wichtig, die Informationen so zu gestalten, dass die Aufmerksamkeit der Benutzer optimal genutzt wird und keine negativen Auswirkungen durch beispielsweise Überlagerung gewisser Elemente und Überforderung des Probanden hervozurufen. Pilotstudien und Nutzerauswertungen zeigen, dass räumliche, aufmerksamkeitsstarke Projektionen am nützlichsten waren. Exogene Hinweise können für den Nutzer auch in vertrauten Umgebungen nützlich sein. Unter Berücksichtigung der Position eines Nutzers und seiner Leistung können Schnittstellen bereitgestellt werden, die die Aufgaben unterstützen und nicht stören. Multimodale erweiterte Interaktionen können eine Vielzahl von Aktivitäten verbessern, einschließlich verfahrenstechnischer Aktivitäten.\\

- \cite{chittaro20043d}\\
3D-Lokalisierung als Navigationshilfe in virtuellen Umgebungen: Die Navigationsunterstützung durch die Benutzeroberflächen von Virtual Environments ist oft unzureichend und meist zu komplex, insbesondere bei großen VE. Ein schlecht aufgebautes Nutzerinterface führt dazu, dass Menschen desorientiert werden und sich verlaufen. Diese Probleme treten immer häufiger in großen/ komplexen virtuellen Umgebungen auf und wollen eigentlich vermieden werden. Als Navigationshilfe sollte ein einfach zu steurendes Werkzeug genügen mit wenig Funktionen, so werden Fehler in der Interaktion vermieden. Zudem führen unnötig angezeigte Informationen ebenso zu Verwirrung der Probanden. Auf diese solltet verzichtet werden und man sollte stattdessen durch kleine Interaktionen oder 'Nachfragen' Zugriff zu anderen Features erhalten. \\
In der Studie wurden 4 Gruppen aufgestellt wobei die erste einen 3D Pfeil zur Hilfe in der virtuellen Umgebung hatte, die zweite Gruppe einen 3D Pfeil, die dritte Gruppe eine 2D-Hilfe, die auf einer Radarmetapher basiert und als letztes die vierte Gruppe, welcher jegliche Hilfe verwährt blieb. Es stellte sich heraus, dass die Gruppe ohne Hilfe deutlich schlechter im Bezug auf die Zeit in der Aufgaben bewältigt wurden abschnitten, jedoch bei den anderen drei Gruppen keine signifikanten Unterschiede aufgetreten sind.
