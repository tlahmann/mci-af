\section{Übergänge in und aus der Virtuellen Realität}\label{sec:relatedWork.vr}

Der Begriff \textit{Immersion} nimmt in virtueller Realität eine Sonderrolle ein. Er beschreibt den Grad mit dem die virtuelle Umgebung als real empfunden wird. Ein hoher Immersionsgrad wird auch als Präsenz bezeichnet~\cite{bowman2007virtual,knibbe2018dream}.

% \subsubsection{Immersion}
% Ganz allgemein könnte man noch etwas zu Vor-/Nachteilen von Immersion schreiben und was das bedeutet. -> bowman2007virtual 
%-> das vllt dennis fragen und erstmal ned machen -sb

\subsubsection{Hilfestellung innerhalb der virtuellen Realität}

Die Navigationsunterstützung durch die Benutzeroberflächen von virtual Environments ist oft unzureichend und meist zu komplex, insbesondere bei großen Umgebungen~\cite{chittaro20043d}. 
Ein schlecht aufgebautes Nutzerinterface führt dazu, dass Menschen desorientiert werden und sich in Menüstrukturen 'verlaufen'~\cite{chittaro20043d,bonanni2005attention}. 
Diese Probleme treten immer häufiger in großen und komplexen virtuellen Umgebungen auf und sollten eigentlich vermieden werden. 
Unnötig angezeigte Informationen führen schnell zur Verwirrung der Nutzer, sowohl im echten Leben (z.B. innerhalb des Autos) als auch in virtuellen Umgebungen~\cite{chittaro20043d}. 
Auf diese kann verzichtet werden, wenn stattdessen durch kleine Interaktionen der Zugriff zu weiteren Features gelenkt wird~\cite{chittaro20043d}. Die Steuerung der Aufmerksamkeit von Nutzern kann in virtuellen und augmentierten Umgebungen durch eine geschickte Gestaltung oder weitere visuelle Hinweise geschehen~\cite{bonanni2005attention}. Es bleibt dabei ein wichtiger Punkt, wie diese Hinweise auszusehen haben, da auch auf kulturelle Unterschiede eingegangen werden muss. 

In einer Studie, durchgeführt von Chittaro et al.~\cite{chittaro20043d}, wurden 4 Gruppen aufgestellt um unterschiedliche Designprinzipien zu untersuchen. 
Der ersten Gruppe wurde ein 3D Pfeil in der virtuellen Umgebung zur Hilfe gezeigt, der zweite Gruppe ein zweidimensionaler Pfeil, der dritten Gruppe eine 2D-Hilfe, ähnlich einer Radarmetapher und der vierte Gruppe keine visuelle Hilfe~\cite{chittaro20043d}. 
Es stellte sich heraus, dass die Gruppe ohne Hilfe deutlich schlechter im Bezug auf die Zeit in der Aufgaben bewältigt wurden abschnitten, jedoch bei den anderen drei Gruppen keine signifikanten Unterschiede aufgetreten sind~\cite{chittaro20043d}.

\subsubsection{Übergänge}

Der Vorgang des Ein- und Ausstiegs in und aus VR, also der Moment des An- und Ausziehens des Headsets, spielt eine wichtige Rolle in der gesamten Benutzererfahrung~\cite{knibbe2018dream}.  Knibbe et al. haben hierzu Untersuchungen im Bereich Virtual Reality angestellt welche Faktoren von besonderer Bedeutung sind.
Eine Erkenntnis der Studie ist, dass der "`Moment des Ausstiegs"' ein Zeitpunkt ist bestimmte Verhaltensweisen oder Gefühle beim Nutzer hervorzurufen~\cite{knibbe2018dream}. Hierzu kann das Verlieren von Kontrolle oder von sensorischen Einflüssen zählen, oder auch die Veränderte Wahrnehmung von Farben und Lichtverhältnissen~\cite{bowman2007virtual,knibbe2018dream}. Nutzer, die in eine virtuelle Umgebung eintauchen oder diese verlassen werden von einem temporal vertrauten Umfeld in eine nicht vertraute Umgebung überführt. Hierbei kann es zu veränderter Wahrnehmung, oder sogar Verwirrung kommen, wie einige Teilnehmer beschrieben. 

Die Gestaltung dieses besonderen Moments könnte zu vielen Zwecken genutzt werden. Zum Beispiel könnten Designer eine unerwartete Situation beim Entfernen des Headsets vermeiden, indem sie den Benutzer auf alle Änderungen in der realen Umgebung aufmerksam machen.
Hierdurch wird der Übergang möglichst "`unauffällig"' gestaltet, was einige der Probleme beheben könnte. 
Dies könnte beispielsweise durch die VR-Anwendung vorgenommen werden, indem die Lichtverhältnisse in VR denen der realen Welt angepasst werden.
Weiter könnten ungewünschte "`harte cuts"' durch das Einspielen von Bildern aus der realen Welt vermieden werden~\cite{knibbe2018dream}.
Für bestimmte Szenarien, in denen diese strikten Übergänge gewünscht sind, könnten sie viel mehr genutzt werden, wie zum Beispiel in Horror Spiele beim Erzwingen bestimmter Angstsituationen~\cite{knibbe2018dream}.

Durch eine thematischen Analyse zeigen sich fünf Aspekte als besonders bedeutend: Raum, Kontrolle, Zeit, gesellschaftlicher Umgang und sensorische Anpassung~\cite{knibbe2018dream}.
Ein auffälliger Punkt ist die räumliche Desorientierung unabhängig von der Komplexität der VR-Szene. Die Teilnehmer beschreiben hier einen Übergang vom VR-Verlassen hin zur Realität~\cite{knibbe2018dream}. So sagten Probanden aus, dass der Übergang für sie zuerst mental und dann physisch passiere~\cite{knibbe2018dream}.
Bisher war die Erfahrung von VR ausschließlich innerhalb des VR-Headsets gebunden, aber der Moment des Ausstiegs kann einen Zeitpunkt darstellen, diese strikte Grenze zwischen virtueller und realer Welt aufzuweichen.

Teilnehmer beschreiben weiter Veränderungen in ihrem Kontrollgefühl als sie den Übergang zwischen VR und Realität vollzogen haben~\cite{knibbe2018dream}. 
Nutzer fühlen sich unter Anderem erleichtert nachdem sie die VR Brille abgenommen haben und meinten, dass die reale Welt vertrauter ist. 
Es hat mit Sicherheit zu tun, die dadurch entsteht, dass man weiß, dass das Headset entfernt werden kann, um sofort in die reale Umgebung zurückzukehren~\cite{knibbe2018dream}.

Zudem fühlen sich Probanden schnell desorientiert durch Unterschiede der virtuellen zur realen Umgebung~\cite{knibbe2018dream}. Der globale Orientierungssinn kann also durch statische Ausrichtung der VR Elemente beibehalten werden~\cite{knibbe2018dream}.
Bis heute existiert eine Lücke zwischen dem menschlichen Verständnis von VR-Erfahrungen im Moment des Ausstiegs aus VR, der wirklichen Erfahrung in VR und dem Auftreten von VR-Nachwirkungen~\cite{knibbe2018dream}.
