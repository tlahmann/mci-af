\section{Zukunftsaussicht}

%\todoSab{beschreiben ob Erwartungen erfüllt worden sind, eigene Feststellungen und auch Literatur}
%\todoSab{Beschränkungen der Forschung darlegen: es war nicht Ziel er Arbeit blabla oder es würde den Rahmen der Arbeit sprengen blabla}
%\todoAll{ist die Validität der Forschung belegt?}

Da VR und AR immer mehr in Zukunft in das Alltagsleben integriert werden sollen, ist es ein interessantes Thema Alltagssituationen wie Schlafen oder gewisse Aufgaben zu erledigen im VR/AR Kontext zu untersuchen. Bei manchen Probanden haben wir gehört, dass sie es sich gut vorstellen könnten eine AR Brille dauerhaft zu tragen, falls dies sich von der Größe und vom Gewicht nicht mehr von normalen Sehstärkebrille unterscheidet. Zu dem jetzigen Zeitpunkt ist dies jedoch noch nicht möglich.

Es war nicht Ziel der Arbeit die Probanden auf eine anstrengende Aufgabe mit zeitlichem Druck vorzubereiten wie es in einem Kontext mit autonomen Fahren hätte sein können, wie beispielsweise die Vorbereitung auf einen Unfall. Die Nutzer sollten lediglich den Übergang erfahren, der zwischen einem Ruhezustand zu einem Zustand indem sie koginitv anspruchsvolle Aufgaben lösen sollten. Hierbei legten wir hauptsächlich wert auf die Aufweckparameter und die Fehleranzahlen und nicht darauf den Probanden in eine Stressituation zu bringen.

Im Kontext des autonomen Fahrens sollte man beim Aufwecken des Fahrers auf jeden Fall einen gewissen Abstand an Zeit bis hin zum Eingreifen einplanen, da deutlich wurde, dass der Mensch erst vorbereitet werden muss. Durch zahlreiche Studien wurde dieses veränderte Verhalten direkt nach dem Aufwachen auch schon belegt. Sobald autonomes Fahren weiter fortgeschritten ist, wäre eine Wiederholung der Studie mit angepassten Variablen und einer angepassten Studienumgebung denkbar. Hierbei könnten die angesprochenen Stress- und Drucksituationen implizit eingebaut und analysiert werden.