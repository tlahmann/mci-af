\section{Studiendurchführung und Studienumgebung}

Auch mit bereits gegebener VR-Erfahrung war für die Interaktion eine kurze Lernphase erforderlich.
Unsere Beobachtungen zeigten aber, dass erfahrenen Nutzer die Interaktionen automatisch ausprobierten und schnell verstanden. 
Da wir als Interaktionsmöglichkeit ein Verweilen der Pointer-Funktionalität auf Interaktionsobjekten gewählt haben, ergab sich hier auch für VR-Erfahrene eine kleine Umstellung. Wir haben diese Funktionalität gewählt um ein "`Auffüllen"' zu simulieren. Der Nutzer sollte also mittels des Controllers die Objekte anregen und so auswählen.
Hierbei dauerte es eine halbe Sekunde (500ms) vom Start der Interaktion bis das Objekt vom System als gewählt registriert wurde. Wir versuchten mit dieser Vorgehensweise die Möglichkeit offen zu lassen, dass Nutzer sich umentscheiden können, falls sie eine initial getroffene Entscheidung nicht mehr für die richtige halten. 
Unsere Ergebnisse zeigen, dass dies allerdings nicht vorgekommen ist, wodurch der Zeitraum entweder verkürzt oder gänzlich entfernt werden könnte. 
Außerdem teilten uns Teilnehmer mit, dass haptisches Feedback über den Controller für den Nutzer noch hilfreich gewesen wäre. Dies resultiert aus der Gewohnheit aus anderen VR Untersuchungen oder Spielen, sowie aus der Annahme, dass das Interagieren mit Objekten auch in der realen Welt ein haptisches Feedback liefert.

Auf Grund der kurzen Ruhephase konnte man nicht davon ausgehen, dass alle Teilnehmer einschlafen beziehungsweise zur Ruhe kommen. So zeigt auch die Tabelle~\ref{tab:sleepstatus}, dass die Teilnehmer mit einer deutlichen Tendenz nicht geschlafen haben. 
Ein Grund hierfür könnte sein, dass es für manche Menschen schwer ist in einer unbekannten Umgebung und vor anderen Personen einzuschlafen. Unsere Ergebnisse zeigen aber auch, dass die Nutzer durchaus zur Ruhe gekommen sind, mit einem Anteil von 20\% der schlafenden und 26,7\% der dösenden Probanden. 
Ein anderer Grund hierfür könnte auch die Studienumgebung sein, denn in manchen Fällen waren Umgebungsgeräusche, wie das Klacken von Computertastaturen, Musik oder auch Gespräche aus den Nebenräumen wahrnehmbar. Wir haben dies bereits in der Implementierungsphase der VR-Umgebung bemerkt und uns daher entschieden die Nutzer mit einem Headset auszustatten und über dieses meditative Musik abzuspielen.
Für manche Teilnehmer war die Wahl der Musik störend und verhinderte auch das komplette Entspannen beziehungsweise Einschlafen. 
Andere empfanden die Musik als entspannend und wohltuend. 
Hier könnte man eventuell spezifischer auf die Teilnehmer eingehen und jeweils eine eigene "`Schlafmusik"' wählen lassen, beziehungsweise sogar gänzlich weglassen.

Uns wurde nach der Studie oft mitgeteilt, dass der Stuhl sehr bequem sei sich aber nicht zum Schlafen mit einer VR Brille auf dem Kopf eignet. 
Dies war oft der Fall, wenn Probanden die Halteelemente der HMD zwischen dem Kopf und dem Stuhl hatten und dadurch eigentlich auf der VR-Brille lagen und den Kopf möglicherweise nicht bequem ablegen konnten. 
Nach unserer Einschätzung könnte hier die Weiterentwicklung der Geräte hilfreich sein. Kleinere Geräte, mit noch höherem Tragekomfort müssen hierfür allerdings erst noch entwickelt werden.

Dem Entspannen hinderlich waren bei manchen Probanden auch äußere Einflüsse wie Kaffee oder Aufregung, da das Ziel der Studie im Voraus nicht mitgeteilt wurde. Das Einschlafvermögen war demnach wie erwartet sehr personenabhängig.
Bei Teilnehmern die zum ersten mal mit VR in Kontakt getreten sind wurde deutlich, dass die Faszination und die Neugier in den Vordergrund getreten sind und so im Vergleich zu den VR-Erfahrenen weniger schnell Entspannung eingetreten ist. 
Hier wäre eine längere Instruktionsphase eventuell vorteilhaft, um den Probanden ein gutes Gefühl im Umgang mit VR zu ermöglichen. 
Dazu wäre während der Studiendurchführung Zeit gewesen, allerdings haben wir die Nutzer nicht explizit dazu ermutigt sich mit allem vertraut zu machen. Diesem Problem wäre hier ein besseres Design der VR-Umgebung entgegengekommen.
Uns wurde öfter mitgeteilt, dass mit einer längeren Ruhephase das Einschlafen eingetreten wäre. Dies haben wir bei der Planung unserer Studie allerdings verworfen und betrachten eine Zeit von 15 Minuten als ausreichend.
Manche Nutzer haben uns auch mitgeteilt, dass sie auch nach noch längerer Ruhephase auf dem Stuhl und mit der Brille nicht schlafen hätten können, was wiederum auf die Bequemlichkeit der Brille selbst zurückzuführen ist.

Die Urzeit in der wir die Studie durchgeführt haben, spannte sich von 10 Uhr morgens bis 18 Uhr abends. Hier wäre ein kleinerer Zeitraum ein Faktor, der ein besser vergleichbares Ergebnis liefert. 
Allerdings sollte erwähnt werden, dass zu jeder Tageszeit Teilnehmer existierten, die eingeschlafen sind und Teilnehmer denen es nach eigenen Angaben zu jeder Uhrzeit schwer gefallen wäre.

\todoSab{Hier (oder irgendwo weiter oben im Text) kann diskutiert werden, dass die Wahrgenommene Schlafdauer zwischen den beiden Gruppen Alarm und Fade 5 statistisch signifikant unterschiedlich war!}
% Zitat aus der Email von Dennis vom 06.09.2019: Das sehr ihr in der ANOVA. Hier ist leider nur die wahrgenommene Schlafdauer signifikant unterschiedlich (p=0.037). In den Tukey Post-Hoc Tests direkt darunter sieht man dann, das genauer genommen nur die 5-Sekunden-Gruppe und die Audio-Gruppe signifikant unterschiedlich waren (p=0.05).
% Interessanterweise war die wahrgenommene Schlafdauer in der 5-Sekunden Gruppe kleiner (M=13.03, siehe "Descriptives") als in der Audio-Gruppe (M=16.43) und der 20-Sekunden-Gruppe (M=16.1).
