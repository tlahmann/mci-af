\section{Implementierung}

% Warum haben wir nicht die Aufgaben alle gleichzeitig gestartet? Also quasi in einem Halbkreis alle nebeneinander
In VR können Informationen in einem größeren Bereich dargestellt werden, ohne dass sich Informationen überlagern.
Auf klassischen Bildschirmen, welche heutzutage schon enorme Ausmaße annehmen, werden Bildschirmbereiche oft in Fenster oder Menüstrukturen unterteilt, die verschoben oder minimiert werden müssen, um darunterliegende Informationen zu sehen. 
In VR haben wir einen beinahe sphärischen raum um einen Nutzer herum, der genutzt werden kann um zu informieren, du unterhalten oder zu steuern. 
In unserer Implementierung haben wir uns auf einen kleinen Bereich des Sichtfelds beschränkt um den Nutzer nicht mit Infos zu überhäufen und um eine klar definierte Blickrichtung für den Probanden, den Stuhl und die Aufgaben zu erhalten.

% Was sind die Probleme bei der Implementierung?
% ich glaube gerade es gab gar keine Probleme. 
Eine verpasste Gelegenheit war, die Bewegung des Controllers nicht aufzuzeichnen. Hierdurch hätten Untersuchungen zu Momenten des Zögerns oder eine unruhige Haltung der Nutzer angestellt werden können. Mit unseren jetzigen Daten können wir nur indirekte Schlussfolgerungen anstellen, ob ein Teilnehmer eine Entscheidung für falsch gehalten und sich im Laufe der Auswahl umentschieden hat.
