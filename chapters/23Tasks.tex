\section{Aufgaben erledigen}\label{sec:relatedWork.aufgaben}

Räumliche, aufmerksamkeitssensitive Darstellungen sind effektiv~\cite{bonanni2005attention}. Exogene Hinweise können dem Nutzer helfen sich auch in unbekannten Umgebungen zurechtzufinden~\cite{bonanni2005attention}.

Es existieren unterschiedliche Herangehensweisen um Fahrer in Autos über eine auftretende Gefahrensituation zu informieren. Hierbei wurden textuelle Informationen den grafischen vorgezogen.~\cite{green1995driver}

Kulturelle Unterschiede bewirken, dass sich Fahrer im Straßenverkehr auf unterschiedliche Dinge konzentrieren und im Anschluss an unterschiedliche Details erinnern~\cite{shinohara2017visual}.

Zur Vorbereitung auf die Objekte oder Vorgänge in der Umgebung von Menschen können 3D Marker verwendet werden, die in die Richtung des Objekts oder Geschehens weisen. Eine 3D Darstellung ist nach Chittaro und Burigat mindestens genauso effektiv wie eine 2D Darstellung. Sie bietet jedoch den Vorteil, dass Nutzer auch in der dritten Dimension, der Höhe, auf wichtige Punkte hingewiesen werden können~\cite{chittaro20043d}.\\

-\cite{aschoff1998human}\\
Wahrnehmung von Zeitintervallen und wie das mit Körpertemperatur und Dauer des Wachseins zusammenhängt: Die menschliche Zeitwahrnehmung kann in zwei verschiedene Klassen eingeteilt werden, die sich in ihrer Interaktion mit dem zirkadianen(=tagesrythmischen) System unterscheiden: Kurze Zeitintervalle im Sekundenbereich (bis zu ca. 2 min) werden nicht von Veränderungen des Schlaf-Wach-Zyklus beeinflusst, jedoch unter Bedingungen der zeitlichen Isolation können Veränderungen aufgezeichnet werden. Die Zeitschätzung wurde bei sieben Probanden wüber einen gewissen zeitraum untersucht, bei dem die Probanden von Zeithinweisen isoliert wurden. Kurze und lange Zeitintervalle werden über verschiedene Mechanismen subjektiv erlebt. Dabei weisen die kurzen Intervalle wenig auffallendes und bei den langen Intervallen (~ 1 Stunde) interessante Feststellungen auf. Die Nutzer mussten 1 Stunde Zeitintervalle einschätzen und es stellte sich heraus, dass die Einschätzungen in Verbindung mit der Zeit in der der Nutzer schon wach ist  korreliert. Zudem gab es bei kleinen Zeitintervallen (~ 2 Minuten) eine negative Korrelation mit der Temperatur und eine positive mit der Beleuchtungsintesität. Bei längeren Zeitintervallen gibt es keine signifikanten Auffälligkeiten im Bezug auf die Lichtintensität.\\

-\cite{devisch2018mini}\\
Minispiele/Lernspiele spielen, um kollektives Lernen über komplexe städtische Prozesse zu erleichtern. Raumplanungsprojekte können als Prozesse des kollektiven Lernens verstanden werden. Planer haben sich Spiele und spielerische Ansätze angeschaut, um die Möglichkeiten diese Prozesse zu unterstützen. In Anbetracht der Tatsache, dass Planungsprojekte langwierig sind und komplex, schlagen diese Aufgabe durch sogenannte ernsthafte Minispiele zu bereichern, von dem jedes ein bestimmtes Lernziel anspricht oder Spiele, die die Spieler in Simulationen realer Umgebungen eintauchen lassen. Es geht darum ernsthafte Minispiele zu entwickeln, die von einem kollektiven Lernmodell umrahmt werden.\\ 
Neben der Vermittlung von Ideen und Werten ist es ein wichtiges Merkmal von Serious Games das Lernen zu erleichtern, ohne dass die Spieler es überhaupt bemerken oder als greifbares Lernen wahrnehmen. Das Lernen ist also in der Spielerfahrung verpackt. Planer spielen meist falsche zu komplexe Spiele. Hingegen der bisherigen Annahmen, ist es effektiver vergleichsweise einfache Minispiele zu spielen, die für präzise Lernziele entwickelt werden und in den Momenten gespielt werden, in denen der kollektive Lernprozess ihre Nutzung erfordert.\\
Als Konsequenz wird argumentiert, dass bewusst gestaltete Minispiele, die bestimmte Lernphasen und Gestaltungsmerkmale abdecken, besser geeignet sind als vollwertige ernsthafte Spiele, die nur auf das Endergebnis des Projekts abzielen. Durch Anpassung der Minispiele können gezielt Fähigkeiten im räumliche Kontext oder auch andere Fähigkeiten abgefragt werden, solange die Spiele so klein wie möglich und leicht verständlich sind und nur gezielte Aspekte abfragen.

-\cite{green1995driver}
Warnungen im Fahrzeug für unfallprovokative Situationen (Unfälle im Vorfeld, neue Ampeln, Polizeifahrzeuge auf Notfahrten usw.): Kandidatenwarnungen wurden für 30 Gefahren entwickelt. Anschließend wurden diese Warnungen von 75 Fahrern bewertet. Bevorzugte Warnhinweise waren nicht immer im Standard-Autobahnzeichensatz der USA enthalten. Im Allgemeinen wurden Textwarnungen gegenüber symbolischen Warnungen leicht bevorzugt. Ziele der Studie waren die Entwicklung von Richtlinien für menschliche Faktoren, Methoden zur Prüfung der Sicherheit und Benutzerfreundlichkeit und ein Modell zur Vorhersage der menschlichen Leistung bei der Verwendung dieser Schnittstellen. Im Allgemeinen war der Text vorne links der von den Fahrern bevorzugte Anhaltspunkt für die Lage der Angabe bei einer Gefahr und führte zu höchster Verständlichkeit.
%auch nicht sehr hilfreich und wenig infos
