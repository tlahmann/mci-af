\section{Aufgaben erledigen}\label{sec:relatedWork.aufgaben}

Die menschliche Wahrnehmung von Zeit kann in zwei verschiedene Klassen eingeteilt werden, die sich in ihrer Interaktion mit dem zirkadianen\footnote{Ein zirkadianer Rhythmus (auch: circadianer Rhythmus) bezeichnet zum Beispiel die Schwankungen von Körperfunktionen, die durch exogene (Tag-Nacht-Wechsel) oder endogene (Hormone) Einflüsse gesteuert werden. Beispiele sind Schwankungen der Herzfrequenz, des Schlaf-Wach-Rhythmus, des Blutdrucks und der Körpertemperatur.} System unterscheiden~\cite{aschoff1998human, aschoff1985perception}.

\subsubsection{Vorbereitung}

Teilnehmer einer Studie, durchgeführt von J. Aschoff~\cite{aschoff1998human}, wurden aufgefordert einen Zeitraum von fünf bis zehn Sekunden abzuschätzen, sobald eine Stunde Zeit vergangen war. Im Rahmen der Studie wurden die Probanden von Zeitmessern oder anderen Hinweisgebern isoliert.
Die Aufgabe war also das Abschätzen von langen sowie von kurzen Zeiträumen. 
Hierbei stellte sich heraus, dass die Teilnehmer kurze Intervalle von bis zu zwei Minuten zuverlässig abschätzen konnten und diese Schätzung nicht von Veränderungen des Schlaf-Wach-Zyklus beeinflusst~\cite{aschoff1998human} wird. 
Bei langen Zeitintervallen stellte sich ein Zusammenhang zwischen der Dauer in der ein Proband bereits wach war mit der Zuverlässigkeit seiner Schätzung heraus~\cite{aschoff1998human}. 
Es liegt also nahe, dass das menschliche Gehirn mit abnehmender Müdigkeit ein besseres Zeitgefühl entwickelt.
Weiter stellte sich heraus, dass die Lichtintensität der Umgebung mit der Zuverlässigkeit der Schätzung von kurzen Zeitintervallen korreliert~\cite{aschoff1998human}.
Es liegt hierbei also nahe, dass die Beleuchtung der Umgebung für die Vorbereitung auf eine Aufgabe ein initial wichtigerer Faktor ist als die Temperatur. 

\subsubsection{Aufgaben}
\todoSab{Lesen ob alles passt}

Für die Vorbereitung der Nutzer in die Erledigung der Aufgaben ist die Gestaltung ebendieser ein wichtiger Faktor. Die Gestaltung und Etablierung einer Aufgabe und die Motivation sind Bestandteil unterschiedlicher Forschungsarbeiten. 
So kann die Aufnahme von Informationen einer gestellten Aufgabe beispielsweise auf unterschiedliche Faktoren wie Kontext und zusätzliche Informationen zurückgeführt werden~\cite{salancik1978social, van2002blueprints, hollnagel2003handbook}. 

\todoTob{Übergang Aufgaben, [12] einbinden}

Als Grundlage für die möglichst erfolgreiche Erledigung von Aufgaben kann auch die Vertrautheit mit gegebenen Situationen und Umgebungen gesehen werden~\cite{scott1966activation}. So stellen sich bei zunehmender Sicherheit mit der gegebenen Umgebung eine Sicherheit und ein erhöhtes maß an Selbstvertrauen bei Benutzern ein, die zusammen die Erledigung dieser Aufgaben erleichtern~\cite{scott1966activation}. 

-\cite{devisch2018mini}\\
Als solche Aufgaben kommen mehrere Arten in Frage. Beispielsweise können Lernspiele helfen das kollektive Lernen über komplexe städtische Prozesse zu erleichtern. Raumplanungsprojekte hingegen können als Prozesse des kollektiven Lernens verstanden werden. Planer haben sich Spiele und spielerische Ansätze angeschaut, um die Möglichkeiten dieser Prozesse zu unterstützen. In Anbetracht der Tatsache, dass Planungsprojekte langwierig  und komplex sind, schlagen Planer vor diese Aufgaben durch sogenannte ernsthafte Minispiele zu bereichern, von dem jedes einzelne ein bestimmtes Lernziel anspricht oder Spiele, die die Spieler in Simulationen realer Umgebungen eintauchen lassen. Es geht darum ernsthafte Minispiele zu entwickeln, die von einem kollektiven Lernmodell umrahmt werden.\\ 
Neben der Vermittlung von Ideen und Werten ist es ein wichtiges Merkmal von Serious Games das Lernen zu erleichtern, ohne dass die Spieler es überhaupt bemerken oder als greifbares Lernen wahrnehmen. Das Lernen ist also in der Spielerfahrung verpackt. Planer spielen meist falsche und zu komplexe Spiele. Hingegen der bisherigen Annahmen, ist es effektiver vergleichsweise einfache Minispiele zu spielen, die für präzise Lernziele entwickelt werden und in den Momenten gespielt werden, in denen der kollektive Lernprozess ihre Nutzung erfordert.\\
Als Konsequenz wird argumentiert, dass bewusst gestaltete Minispiele, die bestimmte Lernphasen und Gestaltungsmerkmale abdecken, besser geeignet sind als vollwertige ernsthafte Spiele, die nur auf das Endergebnis des Projekts abzielen. Durch Anpassung der Minispiele können gezielt Fähigkeiten im räumliche Kontext oder auch andere Fähigkeiten abgefragt werden, solange die Spiele so klein wie möglich und leicht verständlich gehalten werden und nur gezielte Aspekte abfragen.

\subsubsection{Visuelle Hinweise}

\todoAll{Etwas zu visual cues schreiben...}
\todoAll{Korrekturlesen}

Bei der Verwendung von visuellen Hinweisen gibt es eine Reihe an Faktoren, welche es zu beachten gilt. Zum einen sind räumliche, aufmerkasmkeitssensitive Darstellungen sehr effektiv~\cite{bonanni2005attention}. Außerdem können exogene Hinweise dem Nutzer helfen, sich auch in unbekannten Umgebungen zurechtzufinden~\cite{bonanni2005attention}.

Geht es um spezielle Situationen, wie zum Beispiel dem Informieren eines Fahrers in einem Auto über eine Gefahrensituation, existieren unterschiedliche Herangehensweisen um das Informieren zu bewerkstelligen. In solchen Fällen werden jedoch textuelle Informationen den grafischen vorgezogen~\cite{green1995driver}.

-\cite{green1995driver}
In einer Studie wurden Kandidatenwarnungen für 30 unterschiedliche Gefahren designt. Anschließend wurden diese Warnungen von 75 Fahrern bewertet. Dabei kam heraus, dass die bevorzugten Warnhinweise nicht immer im Standard-Autobahnzeichensatz der USA enthalten waren. Im Allgemeinen wurden Textwarnungen gegenüber symbolischen Warnungen leicht bevorzugt. Ziele der Studie waren sowohl die Entwicklung von Richtlinien für menschliche Faktoren und Methoden zur Prüfung der Sicherheit und Benutzerfreundlichkeit, als auch das Entwickeln eines Modells zur Vorhersage der menschlichen Leistung bei der Verwendung dieser Schnittstellen.\todoSab{Was für ein Text?} Im Allgemeinen war der Text vorne links der von den Fahrern bevorzugte Anhaltspunkt für die Lage der Angabe bei einer Gefahr und führte zu höchster Verständlichkeit.

Darüber hinaus ist ein wichtiger Aspekt ebenfalls die Herkunft des Fahrers, da sich kulturelle Unterschied auf die Aufmerksamkeit des Fahrzeugführers auswirken.
Kulturelle Unterschiede bewirken, dass sich Fahrer im Straßenverkehr auf unterschiedliche Dinge konzentrieren und im Anschluss an unterschiedliche Details erinnern~\cite{shinohara2017visual}.

Zur Vorbereitung auf die Objekte oder Vorgänge in der Umgebung von Menschen können ebenfalls 3D Marker verwendet werden, die in die Richtung des Objekts oder Geschehens weisen. Eine 3D Darstellung ist nach Chittaro und Burigat mindestens genauso effektiv wie eine 2D Darstellung. Sie bietet jedoch den Vorteil, dass Nutzer auch in der dritten Dimension, der Höhe, auf wichtige Punkte hingewiesen werden können~\cite{chittaro20043d}.\\
