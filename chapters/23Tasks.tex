\section{Aufgaben erledigen}\label{sec:relatedWork.aufgaben}

Räumliche, aufmerksamkeitssensitive Darstellungen sind effektiv~\cite{bonanni2005attention}. Exogene Hinweise können dem Nutzer helfen sich auch in unbekannten Umgebungen zurechtzufinden~\cite{bonanni2005attention}.

Es existieren unterschiedliche Herangehensweisen um Fahrer in Autos über eine auftretende Gefahrensituation zu informieren. Hierbei wurden textuelle Informationen den grafischen vorgezogen.~\cite{green1995driver}

Kulturelle Unterschiede bewirken, dass sich Fahrer im Straßenverkehr auf unterschiedliche Dinge konzentrieren und im Anschluss an unterschiedliche Details erinnern~\cite{shinohara2017visual}.

Zur Vorbereitung auf die Objekte oder Vorgänge in der Umgebung von Menschen können 3D Marker verwendet werden, die in die Richtung des Objekts oder Geschehens weisen. Eine 3D Darstellung ist nach Chittaro und Burigat mindestens genauso effektiv wie eine 2D Darstellung. Sie bietet jedoch den Vorteil, dass Nutzer auch in der dritten Dimension, der Höhe, auf wichtige Punkte hingewiesen werden können~\cite{chittaro20043d}.
