\section{Aufgaben erledigen}\label{sec:relatedWork.aufgaben}

Die menschliche Wahrnehmung von Zeit kann in zwei verschiedene Klassen eingeteilt werden, die sich in ihrer Interaktion mit dem zirkadianen\footnote{Ein zirkadianer Rhythmus (auch: circadianer Rhythmus) bezeichnet zum Beispiel die Schwankungen von Körperfunktionen, die durch exogene (Tag-Nacht-Wechsel) oder endogene (Hormone) Einflüsse gesteuert werden. Beispiele sind Schwankungen der Herzfrequenz, des Schlaf-Wach-Rhythmus, des Blutdrucks und der Körpertemperatur.} System unterscheiden~\cite{aschoff1998human, aschoff1985perception}.

\subsubsection{Vorbereitung}

Teilnehmer einer Studie, durchgeführt von J. Aschoff~\cite{aschoff1998human}, wurden aufgefordert einen Zeitraum von fünf bis zehn Sekunden abzuschätzen, sobald eine Stunde Zeit vergangen war. Im Rahmen der Studie wurden die Probanden von Zeitmessern oder anderen Hinweisgebern isoliert.
Die Aufgabe war also das Abschätzen von langen sowie von kurzen Zeiträumen. 
Hierbei stellte sich heraus, dass die Teilnehmer kurze Intervalle von bis zu zwei Minuten zuverlässig abschätzen konnten und diese Schätzung nicht von Veränderungen des Schlaf-Wach-Zyklus beeinflusst~\cite{aschoff1998human} wird. 
Bei langen Zeitintervallen stellte sich ein Zusammenhang zwischen der Dauer in der ein Proband bereits wach war mit der Zuverlässigkeit seiner Schätzung heraus~\cite{aschoff1998human}. 
Es liegt also nahe, dass das menschliche Gehirn mit abnehmender Müdigkeit ein besseres Zeitgefühl entwickelt.
Weiter stellte sich heraus, dass die Lichtintensität der Umgebung mit der Zuverlässigkeit der Schätzung von kurzen Zeitintervallen korreliert~\cite{aschoff1998human}.
Es liegt hierbei also nahe, dass die Beleuchtung der Umgebung für die Vorbereitung auf eine Aufgabe ein initial wichtigerer Faktor als die Temperatur ist. \todoAll{wieder auf Studienaufbau verweisen. Bezug muss klar werden bzw. Abgrenzung zur verwandten Forschung}

\subsubsection{Aufgaben}

Für die Vorbereitung der Nutzer in die Erledigung der Aufgaben ist die Gestaltung ebendieser ein wichtiger Faktor. Die Gestaltung und Etablierung einer Aufgabe und die Motivation sind Bestandteil unterschiedlicher Forschungsarbeiten. 
So kann die Aufnahme von Informationen einer gestellten Aufgabe beispielsweise auf unterschiedliche Faktoren wie Kontext und zusätzliche Informationen zurückgeführt werden~\cite{salancik1978social, van2002blueprints, hollnagel2003handbook}. 

Als Grundlage für die möglichst erfolgreiche Erledigung von Aufgaben kann auch die Vertrautheit mit gegebenen Situationen und Umgebungen gesehen werden~\cite{scott1966activation}. So stellen sich bei zunehmender Sicherheit mit der gegebenen Umgebung eine Sicherheit und ein erhöhtes Maß an Selbstvertrauen bei Benutzern ein, die zusammen die Erledigung dieser Aufgaben erleichtern~\cite{scott1966activation}. 

Als solche Aufgaben kommen unterschiedliche Arten in Frage. Beispielsweise können Lernspiele helfen das kollektive Lernen über komplexe Prozesse zu erleichtern~\cite{devisch2018mini, lampert2008gespielte, michael2005serious, ritterfeld2009serious}.
Ein Fokus der Lernspiele ist es Visualisierungen und Abläufe interaktiv an Benutzer heranzutragen und durch die selbst wählbare Lerngeschwindigkeit eine Verbesserung im Verständnis zu erzielen.
Neben der Vermittlung von Ideen und Werten ist es ein wichtiges Merkmal von "`Serious Games"' das Lernen zu erleichtern, ohne dass die Spieler dies als Lernen wahrnehmen~\cite{devisch2018mini, michael2005serious, ritterfeld2009serious}.
Das Lernen ist also in der Spielerfahrung verpackt.
Hingegen der bisherigen Annahmen, ist es effektiver vergleichsweise einfache Minispiele zu spielen, die für präzise Lernziele entwickelt werden und in den Momenten gespielt werden, in denen der Lernprozess ihre Nutzung erfordert~\cite{devisch2018mini}.
Als Konsequenz wird argumentiert, dass bewusst gestaltete Minispiele, die bestimmte Lernphasen und Gestaltungsmerkmale abdecken, besser geeignet sind als vollwertige ernsthafte Spiele, die nur auf das Endergebnis des Projekts abzielen. 
Durch Anpassung der Minispiele können gezielt Fähigkeiten im räumlichen Kontext oder auch andere Fähigkeiten abgefragt werden, solange die Spiele so klein wie möglich und leicht verständlich gehalten werden und nur gezielte Aspekte abfragen~\cite{devisch2018mini, michael2005serious, ritterfeld2009serious}.

\subsubsection{Visuelle Hinweise}

Ein sensorischer oder visueller Hinweis ist ein Maß oder ein Signal, das von einem Wahrnehmenden aus sensorischen Eingängen extrahiert werden kann und das den Zustand einer Eigenschaft der Umwelt angibt, die der Wahrnehmende wahrnehmen möchte.

Bei der Verwendung von visuellen Hinweisen gibt es eine Reihe an Faktoren, welche es zu beachten gilt. Zum einen sind räumliche, aufmerksamkeitssensitive Darstellungen sehr effektiv~\cite{bonanni2005attention}. Hierbei können die Darstellungsformen dazu genutzt werden die Aufmerksamkeit zum Beispiel auf bestimmte Bereiche eines User Interfaces zu lenken~\cite{keller1994visual}. Unterschiedliche Möglichkeiten umfassen hierbei Farbe, Animation oder Ton~\cite{keller1994visual}. Diese exogenen Hinweise können dem Nutzer helfen, sich auch in unbekannten Umgebungen zurechtzufinden~\cite{bonanni2005attention}.

Geht es um spezielle Situationen, wie zum Beispiel dem Informieren eines Fahrers in einem Auto über eine Gefahrensituation, existieren unterschiedliche Herangehensweisen um das Informieren zu bewerkstelligen~\cite{keller1994visual}. In solchen Fällen werden jedoch textuelle Informationen den grafischen vorgezogen~\cite{green1995driver}.

In einer Studie wurden Teilnehmern Warnungen für 30 unterschiedliche Gefahren gezeigt und von 75 Autofahrern bewertet~\cite{green1995driver}. 
Dabei ergab sich, dass die bevorzugten Warnhinweise nicht immer im Standard-Autobahnzeichensatz der USA enthalten waren. 
Im Allgemeinen wurden Textwarnungen gegenüber symbolischen Warnungen leicht bevorzugt~\cite{green1995driver}.
Fahrern wurde informativer Text für unterschiedliche Situationen in einen Head-Up Display präsentiert, wie beispielsweise Warnungen vor Baustellen oder Stau sowie Geschwindigkeitsbegrenzungen. 
Im Allgemeinen war Text, der im vorderen, linken Blickfeld angezeigt wurde, der von den Fahrern bevorzugte Anhaltspunkt und führte zu höchster Verständlichkeit~\cite{green1995driver}.

Ein hierbei wichtiger Aspekt ist allerdings die Herkunft des Fahrers, da sich kulturelle Unterschiede auf die Aufmerksamkeit des Fahrzeugführers auswirken~\cite{shinohara2017visual}.
Kulturelle Unterschiede bewirken, dass sich Fahrer im Straßenverkehr auf unterschiedliche Dinge konzentrieren und im Anschluss an unterschiedliche Details erinnern~\cite{shinohara2017visual}.

Zur Vorbereitung auf die Objekte oder Vorgänge in der Umgebung von Menschen können ebenfalls 3D Marker verwendet werden, die in die Richtung des Objekts oder Geschehens weisen. Eine 3D Darstellung ist nach Chittaro und Burigat mindestens genauso effektiv wie eine 2D Darstellung. Sie bietet jedoch den Vorteil, dass Nutzer auch in der dritten Dimension, der Höhe, auf wichtige Punkte hingewiesen werden können~\cite{chittaro20043d}.
