% Achtung: Vor dem Verwenden dieser Vorlage unbedingt die readme lesen!
\documentclass{diplom-mi-eng}

%\usepackage[showframe]{geometry}

% debug only
%\usepackage{showframe} 

%\usepackage{longtable}
\usepackage{multirow}
\usepackage{tabularx}
\usepackage{rotating}
\usepackage{pdflscape}
\usepackage{url}
\usepackage{enumitem}
\usepackage{siunitx}
\usepackage{framed}
\usepackage{enumitem}
\usepackage{amsmath}
\usepackage{eurosym}
\usepackage{caption}
%\usepackage{capt-of}
%\usepackage{changepage}
\usepackage{microtype}

\newcolumntype{x}{>{\raggedright\arraybackslash}X}
\newcolumntype{y}{>{\raggedleft\arraybackslash}X}

\usepackage{ntheorem}
\theoremseparator{:}
\newtheorem{hyp}{Hypothesis}

\newcommand{\divider}{
	\begin{center}
		\small
		$\ast$~$\ast$~$\ast$
	\end{center}
}

\usepackage[dvipsnames]{xcolor}
%\documentclass[xcolor=dvipsnames]{beamer}
\usepackage{listings}
% Inline code command
\def\code#1{\texttt{#1}}

% Question numbering
\newcounter{questionCounter}[chapter]% Hypothesis counter
% Representation of hypothesis counter: H-<chap>.<hypCNT>
\renewcommand{\thequestionCounter}{Q\arabic{questionCounter}}
\newcommand{\question}[1]{%
	\refstepcounter{questionCounter}% Step hypothesis counter
	\thequestionCounter% Print hypothesis counter
%	\label{#1} % Mark with label
}

\usepackage{color}
\definecolor{background}{HTML}{F5F5F5}
\definecolor{jsonStr}{HTML}{800000}
\definecolor{stringColor}{HTML}{a36900}
\definecolor{delim}{RGB}{20,105,176}
\definecolor{punct}{RGB}{20,105,176}
\colorlet{numb}{magenta!60!black}

\lstdefinelanguage{JavaScript} {
	morekeywords={
		break,const,continue,delete,do,while,export,for,in,function,
		if,else,import,in,instanceOf,label,let,new,return,switch,this,
		throw,try,catch,typeof,var,void,with,yield
	},
	sensitive=false,
	morecomment=[l]{//},
	morecomment=[s]{/*}{*/},
	morestring=[b]",
	morestring=[d]'
}

\lstdefinelanguage{json}{
	numbers=left,
	numberstyle=\scriptsize,
	stepnumber=1,
	numbersep=8pt,
	showstringspaces=false,
	breaklines=true,
	frame=lines,
	backgroundcolor=\color{background},
	morestring=[b]",
	morestring=[d]',
	stringstyle=\ttfamily\color{jsonStr},
	literate=
		*{0}{{{\color{numb}0}}}{1}
		{1}{{{\color{numb}1}}}{1}
		{2}{{{\color{numb}2}}}{1}
		{3}{{{\color{numb}3}}}{1}
		{4}{{{\color{numb}4}}}{1}
		{5}{{{\color{numb}5}}}{1}
		{6}{{{\color{numb}6}}}{1}
		{7}{{{\color{numb}7}}}{1}
		{8}{{{\color{numb}8}}}{1}
		{9}{{{\color{numb}9}}}{1}
		{:}{{{\color{punct}{:}}}}{1}
		{,}{{{\color{punct}{,}}}}{1}
		{\{}{{{\color{delim}{\{}}}}{1}
		{\}}{{{\color{delim}{\}}}}}{1}
		{[}{{{\color{delim}{[}}}}{1}
		{]}{{{\color{delim}{]}}}}{1},
}

\lstset{
	frame=tb,
	framesep=5pt,
	basicstyle=\ttfamily\singlespacing\fontsize{9}{11}\selectfont,
	numbers=left,
	numberstyle=\scriptsize,
	showstringspaces=false,
	keywordstyle=\ttfamily\bfseries\color{Blue},
	identifierstyle=\ttfamily,
	stringstyle=\ttfamily\color{stringColor},
	commentstyle=\color{OliveGreen},
	rulecolor=\color{Gray},
	backgroundcolor=\color{background},
	xleftmargin=5pt, % the distynce of the frame on the left side
	xrightmargin=5pt, % .. right side
	aboveskip=\bigskipamount,
	belowskip=\bigskipamount
}

% TRANSSCRIPT
\newlength{\transcriptlen}

\NewDocumentCommand {\setmessage} { mo } {%
	\IfNoValueTF{#2}
	{\expandafter\newcommand\csname#1\endcsname{\item[#1:]}}%
	{\expandafter\newcommand\csname#1\endcsname{\item[#2:]}}%
	\IfNoValueTF{#2}
	{\settowidth{\transcriptlen}{#1}}%
	{\settowidth{\transcriptlen}{#2}}%
}

\setmessage{tut}[Tutorial Message]

\DeclareCaptionType{myText}[Text][List of Texts]

%\setlength{\belowcaptionskip}{50pt plus 0pt minus 5pt}

%\usepackage[marginpar]{todo}

% Default ist serifenlose-Schrift (Helvetica), wenn Serifenschrift (Palatino)
% gewünscht ist, einfach folgende Commands auskommentieren.
\renewcommand{\sfdefault}{phv}
\renewcommand{\rmdefault}{phv}
\renewcommand{\ttdefault}{pcr}



\newcommand{\projectName}{Resync}
\newcommand{\projectSubline}{Resync}


% Bitte folgende Variablen anpassen:
\author{Tobias Lahmann}
\title{\projectName: \projectSubline} 
\authorA{Sabrina Böhm}
\emailA{sabrina.boehm@uni-ulm.de}
\authorB{Luca Porta}
\emailB{luca.porta@uni-ulm.de}
\authorC{Tobias Lahmann}
\emailC{tobias.lahmann@uni-ulm.de}

\type{Anwendungsfach} 	%Art der Arbeit, z.B. Diplomarbeit, Masterarbeit,Bachelorarbeit
\jahr{2019}

\fakultaet{Engineering, \\Computer Science and Psychology}
\institut{Institute of Media Informatics}

\gutachterA{Prof. Dr.-Ing. Michael Weber}
%\gutachterB{Prof. ...}
\betreuer{Dennis Wolf, M.Sc.}

% Ende User-Variablen

\begin{document}

\frontmatter %%%%%%%%%%%%%%%%%%%%%%%%%%%%%%%%%%%%%%%%%%%%%%%%%%%%%%%%%%%%%%%%%

\maketitle	% Titelblatt, siehe diplom-mi.cls

\clearpage
\thispagestyle{empty}
{	\small\textsf
	\flushleft
	~\vfill
	Last updated: \today\\[1cm]
	\copyrightinfo\\[.5cm]	% Copyright Notice - siehe diplom-mi.cls
	Typesetting: PDF-\LaTeXe
}


\setstretch{1.2}	% Zeilenabstand ab hier 1.2

\begin{abstract}
	Die Untersuchung der Heranführung einer Person an ein Problem, wenn diese im Vorfeld keine Informationen über die Aufgabe hat. % Introduction. In one sentence, what’s the topic?
	Wir untersuchen die unterschiedlichen Arten jemanden aus einem trance-ähnlichen Zustand, wie er nach dem Schlafen auftreten kann, in einen Bewussten zu überführen und wie diese Person in diesem Vorgang unterstützt werden kann. % State the problem you tackle
	Vorbereitungen auf Aufgaben und noch spezieller die Lenkung der Aufmerksamkeit bei dieser wurde noch ungenügend untersucht. % Summarize (in one sentence) why nobody else has adequately answered the research question yet
	Wir untersuchen die Problemstellung im Kontext von VR und nutzen dies um dadurch unterschiedliche Parameter des 'Aufweckens' sowie Designprinzipien zu untersuchen. % Explain, in one sentence, how you tackled the research question
	In einer between-subject Studie wurden die Unterschiede der möglichen 'Aufweckarten' untersucht. % In one sentence, how did you go about doing the research that follows from your big idea
	Aktuell können unsere Vermutungen noch nicht bestätigt werden. % As a single sentence, what’s the key impact of your research?
\end{abstract}


\tableofcontents

\mainmatter %%%%%%%%%%%%%%%%%%%%%%%%%%%%%%%%%%%%%%%%%%%%%%%%%%%%%%%%%%%%%%%%%%

% Ab hier Kapitel einbinden
\chapter{Introduction}

This is the introduction

\section{Motivation}

Virtuelle und Augmentierte Umgebungen begleiten uns bereits seit einigen Jahren.  Die Produkte, welche dabei zum Einsatz kommen, sind zum jetzigen Zeitpunkt aufgrund der unpraktischen Größe nicht für das alltägliche Umfeld des Durchschnittsbürgers geeignet und angesichts der hohen Preise auch nicht sonderlich weit verbreitet.
Werden in Zukunft diese Systeme kleiner, leichter, oder sogar permanent mit dem menschlichen Körper verbunden, wird es möglich sein, jeden Bereich des Lebens zu augmentieren. AR/VR soll ein stetiger Begleiter sein und dem Nutzer auch in anspruchsvollen oder unerwarteten Situationen unter die Arme greifen. Eine solche unerwartete Situation könnte beispielsweise direkt nach dem Aufwachen aus dem Schlaf der Fall sein. So können in autonomen Autos Aufgaben vom Fahrer übernommen, im Nachtdienst eines Sicherheitsunternehmens kritische Vorgänge überwacht, oder am Morgen Herausforderungen vom Benutzer verlangt werden, welche ein hohes Maß an Aufmerksamkeit erfordern.
Wir möchten herausfinden wie schnell und effizient ein Benutzer auf diese Aufgaben Vorbereitet werden kann. Vor allem die Frage, zu welchem Zeitpunkt der Benutzer geweckt wird gilt es hierbei zu erforschen.

\section{Problemdefinition}

In unterschiedlichsten Fällen können Nutzer auf die Erledigung einer Aufgabe in einer virtuellen Umgebung nur schwer vorbereitet werden, wenn diese  plötzlich oder ohne Überleitung gestellt wird. So können Nutzern einer virtuellen oder augmentierten Realität (VR/AR) beim Wechsel der Umgebung, oder beim Wechsel in die digitale Umgebung, Informationen fehlen, welche notwendig sind um sich schnell, zuverlässig und ohne potenzielle Fehlerquellen an diese zu gewöhnen.

Das Projekt \projectName \  soll es einem Nutzer ermöglichen alle relevanten Informationen innerhalb kürzester Zeit aufzunehmen. Des Weiteren soll untersucht werden auf welche Art und Weise dieser Vorgang zuverlässig durchgeführt werden kann. Wichtig ist hierbei vorallem die Qualität der erbrachten Leistung.

\section{Herangehensweise}\label{sec:approach}


\chapter{Verwandte Forschung}\label{sec:relatedWork}
Um einen Überblick über die Thematik zu erlangen untersuchten wir die bestehende verwandte Forschung. Dies haben wir in einzelne Teilgebiete unterteilt. Zuerst betrachteten wir den Schlafzustand, aus welchem heraus wir unsere Untersuchungen durchführen wollten. Anschließend das unmittelbar auf den Schlaf folgende Aufwachen und als dritten Abschnitt das Erledigen von anfallenden Aufgaben nach den ersten zwei Phasen.



\section{Schlafen}\label{sec:relatedWork.schlafen}

- \cite{dinges1985assessing}:\\

Schlafentzug verursacht tiefere Kurzschlaf-Phasen~\cite{dinges1985assessing}. Sollte eine optimale Performance in Aufgaben benötigt werden sollten Kurzschlaf-Phasen vermieden werden~\cite{dinges1985assessing}. Schlummern sowie Nickerchen sollten gemacht werden bevor ein gravierender Schlafentzug eintritt~\cite{dinges1985assessing}.\\
Bei stupiden langanhaltenden Arbeitsvorgängen kann es schnell passieren, dass man müde wird. Umso länger man im gleichbleibenden Trott ist wird der Schlafmangel immer größer. Diese Probleme kann man einerseits durch Nickerchen/schlafen lindern, aber es erhöht das Risiko, dass die Person bei abruptem Erwachen Schwierigkeiten hat, zu funktionieren. Mit unterschiedlichen Schlafentzugszeiten (6, 18, 30, 42 oder 54) wurde eine Studie durchgeführt, um zu testen wie gut die Probanden auf Ereignisse reagieren können. Nach den verschiedenen zeiten war es den Nutzern gestattet 2 Stunden zu schlafen. Schlafentzug erhöhte die Menge an Tiefschlaf in den Nickerchen, was mit einer größeren Abnahme der kognitiven Leistung nach dem Nickerchen verbunden war.\\
Die Manipulation von zunehmenden Schlafmangel führte zu einem tieferen Schlaf, was siginifikante Leistungseinbußen hervorruf. Dies war für die kognitive Leistung am dramatischsten. Die direkt nach dem schlafen auftretende Veriwrrung wird Schlafträgheit genannt. Probanden die 42 bis 54 Stunden lang schlaflos waren wurde diese Schlafträgheit durch tiefgreifende Desorientierung, Unfähigkeit und Verwirrung gekennzeichnet.
\texttt{sabrina fazit: paper is schwer verständlich für mich und Hauptaussagen stehen oben}\\

- \cite{kraemer2000time}:\\

Abhängig von der Tageszeit existieren Unterschiede in der Performance, so wie der Selbsteinschätzung und anderer psychologischer Parameter bei voll ausgeschlafenen Probanden (12 Stunden Schlaf)~\cite{kraemer2000time}.



\section{Aufwachen}\label{sec:relatedWork.aufwachen}

Bis zu 2 Stunden nach dem Aufwachen kann die subjektive Aufmerksamkeit und die kognitive Leistungsfähigkeit noch beeinträchtigt sein~\cite{jewett1999time}. Eine Herunterregulierung der Körpertemperatur und der damit einhergehende geringere geistige Leistungsfähigkeit könnte der Auslöser sein für den Zustand der Schlafträgheit~\cite{dinges1990you}. Sollte dies stimmen kann erwartet werden, dass jegliche Aktivität, die die Körpertemperatur erhöht dem schlaffen Gefühl entgegenwirkt, das nach dem Schlafen einige Zeit einsetzt und erst mit der Zeit abgebaut wird~\cite{jewett1999time}. Jewett et. al. fanden heraus, dass aber weder die Helligkeit der Umgebung noch andere Aktivitäten, die kurz nach dem Aufwachen erledigt wurden (Essen, duschen, etc.) signifikant die Aufmerksamkeit noch die Schlafträgheit oder deren Abbau beeinflussten~\cite{jewett1999time}.

In einer Situation von Müdigkeit, die direkt nach dem Aufwachen einsetzt und erst über die Zeit abgebaut wird konnten Probanden einer Studie noch einfache soziale Interaktion durchführen~\cite{dinges1990you}. Die funktionale Deafferenzierung, wie sie von Broughton genannt wurde~\cite{broughton1968sleep} um die niedrigen Hirnaktivitäten nach dem Aufwachen zu beschreiben, erschweren die Aufbringung der mentalen Kapazitäten für komplexe Aufgaben nach dem Erwachen. Daher müssen, in all den Situationen, welche eine erhöhte Leistungsfähigkeit benötigen, die unumgänglichen Effekte der Schlafträgheit im Vorfeld beachtet und denen, die diese Aufgabe erledigen sollen, einfache Tools zur Unterstützung gegeben werden~\cite{ferrara2000sleep}. Aktuell könnten alarmierende Faktoren verwendet werden um diese Ziele zu erreichen, aber weitere Forschung muss bestätigen welcher der Faktoren am effizientesten ist~\cite{ferrara2000sleep}.
\section{Aufgaben erledigen}\label{sec:relatedWork.aufgaben}

Räumliche, aufmerksamkeitssensitive Darstellungen sind effektiv~\cite{bonanni2005attention}. Exogene Hinweise können dem Nutzer helfen sich auch in unbekannten Umgebungen zurechtzufinden~\cite{bonanni2005attention}.

Es existieren unterschiedliche Herangehensweisen um Fahrer in Autos über eine auftretende Gefahrensituation zu informieren. Hierbei wurden textuelle Informationen den grafischen vorgezogen.~\cite{green1995driver}

Kulturelle Unterschiede bewirken, dass sich Fahrer im Straßenverkehr auf unterschiedliche Dinge konzentrieren und im Anschluss an unterschiedliche Details erinnern~\cite{shinohara2017visual}.

Zur Vorbereitung auf die Objekte oder Vorgänge in der Umgebung von Menschen können 3D Marker verwendet werden, die in die Richtung des Objekts oder Geschehens weisen. Eine 3D Darstellung ist nach Chittaro und Burigat mindestens genauso effektiv wie eine 2D Darstellung. Sie bietet jedoch den Vorteil, dass Nutzer auch in der dritten Dimension, der Höhe, auf wichtige Punkte hingewiesen werden können~\cite{chittaro20043d}.\\

-\cite{aschoff1998human}\\
Wahrnehmung von Zeitintervallen und wie das mit Körpertemperatur und Dauer des Wachseins zusammenhängt: Die menschliche Zeitwahrnehmung kann in zwei verschiedene Klassen eingeteilt werden, die sich in ihrer Interaktion mit dem zirkadianen(=tagesrythmischen) System unterscheiden: Kurze Zeitintervalle im Sekundenbereich (bis zu ca. 2 min) werden nicht von Veränderungen des Schlaf-Wach-Zyklus beeinflusst, jedoch unter Bedingungen der zeitlichen Isolation können Veränderungen aufgezeichnet werden. Die Zeitschätzung wurde bei sieben Probanden wüber einen gewissen zeitraum untersucht, bei dem die Probanden von Zeithinweisen isoliert wurden. Kurze und lange Zeitintervalle werden über verschiedene Mechanismen subjektiv erlebt. Dabei weisen die kurzen Intervalle wenig auffallendes und bei den langen Intervallen (~ 1 Stunde) interessante Feststellungen auf. Die Nutzer mussten 1 Stunde Zeitintervalle einschätzen und es stellte sich heraus, dass die Einschätzungen in Verbindung mit der Zeit in der der Nutzer schon wach ist  korreliert. Zudem gab es bei kleinen Zeitintervallen (~ 2 Minuten) eine negative Korrelation mit der Temperatur und eine positive mit der Beleuchtungsintesität. Bei längeren Zeitintervallen gibt es keine signifikanten Auffälligkeiten im Bezug auf die Lichtintensität.

\input{chapters/30GameDesignAndDevelopment}

\section{Implementierung}

Softwareseitig haben wir eine digitale Testumgebung mit der Unity 3D\footnote{~Unity3D~\url{https://unity3d.com}} Game-Engine erstellt. Für die Untersuchung der Hypothesen wurde eine ruhige Umgebung benötigt, die die Probanden in der richtigen Art und Weise dabei unterstützt in einen schlafähnlichen Zustand zu gelangen. 
Das gewählte Interface für die hier durchgeführten Untersuchungen ist das HTC Vive\footnote{~HTC Vive~\url{https://www.vive.com}} \textit{Head Mounted Device} (HMD) mit den zugehörigen Controllern. 
Bewegung (Locomotion) ist, bis auf Kopfbewegungen, innerhalb der digitalen Umgebung nicht vorgesehen, da die Teilnehmer sitzend Aufgaben erledigen. 
Die Eingabemethoden zur Bewältigung der gestellten Aufgaben werden mit Zeigeoperationen innerhalb der virtuellen Umgebung realisiert. 

\subsection{Codekonventionen und Syntax}

Um eine einheitliche Code-Qualität zu gewährleisten haben wir einige Konventionen festgelegt. Diese sollen außerdem sicherstellen, dass Code, der von unterschiedlichen Entwicklern produziert wurde von allen Beteiligten schnell verstanden und verändert werden kann.
Ein Auszug aus diesen Konventionen kann in der folgenden Auflistung gesehen werden: 
\begin{itemize}
    \item Verwende "`PascalCasing"' für Klassen- und Methodennamen
    \item Verwende "`camelCasing"' für Methoden-Argumente und lokale Variablen
    \item Verwende Substantive oder Substantiv-Ausdrücke als Klassennamen.
    \item Interfaces soll ein großes "`I"' vorangestellt werden. Interfacenamen sind Substantive (-Ausdrücke) oder Adjektive.
    \item \ldots
\end{itemize}

Diese halten sich an die offiziellen .NET Codekonventionen von Microsoft~\cite{online:condeConventions}.

\subsection{Projektstruktur}

Bei der Implementierung des Projekts Haben wir einen Modularen Ansatz gewählt. Einzelne Abschnitte, wie beispielsweise die Ruhephase oder die Durchführung von Aufgaben sind in einzelne Unity-Objekte verpackt worden, welche automatisch die für sie relevanten Abläufe durchführen. 

Manche Komponenten müssen auch auf den globalen zustand zugreifen und alle Komponenten müssen Daten erfassen und abspeichern. Um diesen Punkt zu lösen und die Modularität nicht zu zerstören wird das Unity Eventsystem genutzt und als Schnittstelle zwischen einer globalen Manager-Komponente und den einzelnen kleinen Teilen verwendet. 
Diese globale Komponente übernimmt neben dem Gesamtablauf auch die Speicherung der Bewegungsinformationen des Probanden. 

Um festzustellen, ob Studienteilnehmer eher unruhig waren während der Ruhephase wird der Vorwärts-Vektor des HMD sowie die Position dieses innerhalb der Virtuellen Umgebung aufgezeichnet.

Für die Aufgaben werden die folgenden werte erfasst:
\begin{itemize}
    \item Gegebene und erwartete Antwort
    \item Korrektheit der gegebenen Antwort
    \item Zeit der Gegebenen Antwort
\end{itemize}

\subsection{VR-Umgebung}
\todoTob{In-Study umgebung Beschreiben -> Interessant für die Aussagen im Anhang (Partikel etc.)}

Um einen möglichst hohen Entspannungsgrad zu erreichen haben wir uns dazu entschieden die Probanden sowohl visuell mittels VR, als auch auditiv durch Over-Ear-Kopfhörer von der Außenwelt abzuschotten. Dazu implementierten wir eine 360-Grad-Weltraumumgebung. Die Wahl für dieses Szenario trafen wir, da wir die Teilnehmer in eine ruhige, dunkle Umgebung hineinsetzen wollten, welche zeitgleich aber weder unheimlich, noch beklemmend wirkt. Genau deshalb erachteten wir eine Umgebung mit viel Platz auf allen Seiten mit einem ansehnlichen Sternenhimmel, unter welchem sich die Probanden wohl fühlen sollen, als passend. Damit die Umgebung nicht starr und somit unnatürlich wirkte, fügten wir bunte, an Glühwürmchen erinnernde Partikel hinzu, welche langsam aber stetig hoch in den Himmel steigen. 
Auf auditiver Ebene entschieden wir uns für eine sphärische, ruhige, und vor allem durchgehende Musik, welche auch für Meditationen verwendet wird. Dies soll zwei Zwecke erfüllen. Zum einen sollte die Immersion der Weltraumumgebung verstärkt werden und dem Teilnehmer somit das Gefühl vermitteln, dass er sich an einem anderen Ort befindet und zum anderen soll die durchgängige Musik kleinere, störende Geräusche daran hindern, den Probanden abzulenken.
Durch die Synergie aus VR-Umgebung und Musik erhofften wir uns die Probanden zum entspannen beziehungsweise zum schlafen zu bringen.   
\todoAll{Korrekturlesen ob alles passt}


\chapter{Durchführung der Studie}

\section{Zeitplan}
Die Daten beziehen sich auf den Zeitpunkt zu dem der jeweilige Schritt abgeschlossen sein sollte.
\begin{itemize}
    \item \textbf{Dezember 2018} Recherche bestehender Forschung
    \item \textbf{Dezember 2018} Beginnen mit der Implementierung und Modellierung der digitalen Unitiy3D Umgebung %.Digitale Umgebung in Unity3D erstellen und mit erster Testphase validieren
    \item \textbf{Juni 2019} Erste Studie durchführen
    \item \textbf{Juni 2019} Abschluss der Studie und Erweiterung auf Zehn weitere Probanden
    \item \textbf{Juli 2019} Einführung auf eine weitere 15-köpfige Testgruppe, welche mit Ton geweckt wird und Implementierung und Entwurf dieser neuen angepassten Studie
    \item \textbf{August 2019} Abschluss der gesamten Studie
    \item \textbf{September 2019} Auswertung der Ergebnisse
    \item \textbf{Oktober 2019} Erstellen der Ausarbeitung und Präsentation
    
    
    
    
    
    
    \item \textbf{Februar 2019} Auswertung der durchgeführten Studie.
    \item \textbf{März 2019} Entwurf der zweiten Studie
    \item \textbf{April 2019} Erweiterte Umgebung in Unity3D erstellen und für den zweiten Studiendurchlauf vorbereiten
    \item \textbf{Mai 2019} Durchführung der zweiten Studie für erweiterte Ergebnisse
    \item \textbf{Juni 2019} Auswertung der Ergebnisse von zweiter Studie
    \item \textbf{September 2019} Dokumentation der Ergebnisse und des Vorgehens
\end{itemize}

\section{Studiendesign}

\todoTob{das hier raus weil bei Lösungsansatz schon die Umgebung etc alles beschrieben wurde und das ja eher zu Vorarbeit zählt? Oder das aus Lösungsansatz hier rein?}

\section{Ergebnisse}

\begin{table*}
	\caption{Numerische Auflistung der Ergebnisse der Frage 'Please enter your age in years'.}~\label{tab:sc_results_gender}
	
	\setlength\tabcolsep{3pt}
	\renewcommand{\arraystretch}{1.4}% for the vertical padding
	\begin{tabularx}{\textwidth}{ | x || x | x | x |}
		\hline
		           & Männlich & Weiblich & Divers \\ \hline\hline
		Absolutwerte   & 33       & 12       & 0                     \\ \hline
		Prozentwerte & 73.3\%   & 26.7\%   & 0\%                   \\ \hline
	\end{tabularx}
\end{table*}

\begin{table*}
	\caption{Numerische Auflistung der Ergebnisse der Frage 'Please select your gender'.}~\label{tab:sc_results_age}
	
	\setlength\tabcolsep{3pt}
	\renewcommand{\arraystretch}{1.4}% for the vertical padding
	\begin{tabularx}{\textwidth}{ | x | x | x | x | x | x | }
		\hline
		Min & Max & Range & Median & Mean  & Standard Deviation \\ \hline\hline
		19  & 30  & 11    & 23     & 23.04 & 2.53              \\ \hline
	\end{tabularx}
\end{table*}


\begin{table*}
	\caption{Verteilung der Antworten zur Frage 'How much experience do you have with VR?'.}~\label{tab:sc_results_age}
	
	\setlength\tabcolsep{3pt}
	\renewcommand{\arraystretch}{1.4}% for the vertical padding
	\begin{tabularx}{\textwidth}{ | x | x | x | x | x | x | x | }
		\hline
		[A1] No experience at all & [A2] Almost no experience & [A3] Less than average experience & [A4] Some experience & [A5] More than average experience  & [A6] Experienced & [A7] Very highly experienced \\ \hline\hline
		10  & 15  & 3    & 10     & 2 & 2 & 3              \\ \hline
		22.2\%  & 33.3\%  & 6.7\%    & 22.2\%     & 4.4\% & 4.4\% & 6.7\%              \\ \hline
	\end{tabularx}
\end{table*}

\begin{table*}
	\caption{Verteilung der Antworten zur Frage 'How much experience do you have with AR?'.}~\label{tab:sc_results_age}
	
	\setlength\tabcolsep{3pt}
	\renewcommand{\arraystretch}{1.4}% for the vertical padding
	\begin{tabularx}{\textwidth}{ | x | x | x | x | x | x | x | }
		\hline
		[A1] No experience at all & [A2] Almost no experience & [A3] Less than average experience & [A4] Some experience & [A5] More than average experience  & [A6] Experienced & [A7] Very highly experienced \\ \hline\hline
		17  & 10  & 7    & 8     & 2 & 1   & 0           \\ \hline
		37.7\%  & 22.2\%  & 15.5\%    & 17.7\%     & 4.4\% & 2.2\%   & 0.0\%           \\ \hline
	\end{tabularx}
\end{table*}

\begin{table*}
	\caption{Verteilung der Antworten zur Frage 'What subject, if any, did you study or are you currently studying?'.}~\label{tab:sc_results_age}
	
	\setlength\tabcolsep{3pt}
	\renewcommand{\arraystretch}{1.4}% for the vertical padding
	\begin{tabularx}{\textwidth}{ | x | x | x | x | x | x | x | x | x | x | }
		\hline
		Biologie & Informatik & Informationssystemtechnik & Mathematik & Medieninformatik & Physik & Psychologie & Software Engineering & Wirtschaftsmathematik & Wirtschaftsphysik \\ \hline\hline
		1  & 8  & 1    & 1     & 18 & 3 & 2 & 8 & 1 & 2           \\ \hline
		2.2\%  & 17.8\%  & 2.2\%    & 2.2\%     & 40\% & 6.7\% & 4.4\% & 17.8\% & 2.2\% & 4.4\%           \\ \hline
	\end{tabularx}
\end{table*}

\todoTob{Mehr Ergebnisse einfügen}
\todoAll{Ergebnisse (wertungsfrei) beschreiben}


\chapter{Diskussion}

\todoAll{Warum sind die Leute nicht eingeschlafen?}
\todoAll{Warum gab es nicht so viele Fehler? (Oder so viele Fehler?)}
\todoAll{Waren die leute zu sehr abgelenkt von VR?}


\chapter{Schlussfolgerung}

\todoAll{Schlussfolgerung Schreiben}
\todoAll{Wann wecken?}
\todoAll{Wie schnell wecken?}
\todoAll{Welche darstellung?}
\todoAll{Wie vorbereiten?}
\todoAll{Auf die Hypothesen eingehen}

Beachtet man die in Kapitel 3 aufgelisteten Hypothesen, so wird klar, dass keine der im voraus definierten Hypothesen komplett bestätigt wird. Zwar lässt sich beim Ordnen der Zahlen tatsächlich ein leichter Trend im Bezug auf Hypothese 2 wahrnehmen, jedoch fällt diese nicht besonders signifikant aus. Ansonsten lässt sich diese Hypothese jedoch nicht anhand er gesammelten Werte bestätigen. Hypothese 1 wird ebenfalls nicht durch die Werte bestätigt, da keine gravierenden Tendenzen zu beobachten sind. Beim Colour-Matching geht der Trend sogar eher in die Richtung, dass Menschen die mit einem akustischen Signal geweckt werden tendenziell eher etwas schneller sind und auch weniger Fehler machen als Probanden, welche einem Lichtreiz zum Wecken ausgesetzt wurden.

Auch in Bezug auf Hypothese 3 und 4 lässt sich nicht herauslesen, dass eine der Beiden Gruppen, welche mit Licht geweckt wurden bedeutsam besser oder schlechter als die andere Gruppe abgeschnitten hat. Den einzigen auszumachenden, wenn jedoch nicht besonders signifikanten Unterschied erkennt man beim Ordnen der Zahlen. Hier zeigen die Daten, dass die Fade5 Gruppe deutlichere Extrema im Bezug auf die Geschwindigkeit aufwiesen. Sowohl der schnellste als auch der langsamste gehörten dieser Gruppe an. Die Fade20 Gruppe hingegen bewegt sich deutlicher im Mittelfeld der Grafik ~\ref{fig:orderingMistakeTimeScatterplot}.

\todoAll{Korrekturlesen}

% \todos

\appendix		% Ab hier Appendices einbinden
\chapter{Anhang 1}

\begin{table*}
	\caption{Numerische Auflistung der Ergebnisse der Frage "`Please select your gender"'.}~\label{tab:sc_results_gender}
	
	\setlength\tabcolsep{3pt}
	\renewcommand{\arraystretch}{1.4}% for the vertical padding
	\begin{tabularx}{\textwidth}{ | x || r | r | }
		\hline
		Geschlecht & Absolutwerte 	& Prozentwerte \\ \hline\hline
		Männlich & 33 & 73.3\% \\ \hline
		Weiblich & 12 & 26.7\% \\ \hline
		Divers & 0 & 0.0\% \\ \hline
	\end{tabularx}
\end{table*}

\begin{table*}
	\caption{Numerische Auflistung der Ergebnisse der Frage "`Please enter your age in years"'.}~\label{tab:sc_results_age}
	
	\setlength\tabcolsep{3pt}
	\renewcommand{\arraystretch}{1.4}% for the vertical padding
	\begin{tabularx}{\textwidth}{ | x | x | x | x | x | x | }
		\hline
		Min & Max & Range & Median & Mean  & Standard Deviation \\ \hline\hline
		19  & 30  & 11    & 23     & 23.04 & 2.53              \\ \hline
	\end{tabularx}
\end{table*}

\begin{table*}
	\caption{Verteilung der Antworten zur Frage "`How much experience do you have with VR?"'.}~\label{tab:sc_results_expVR}
	
	\setlength\tabcolsep{3pt}
	\renewcommand{\arraystretch}{1.4}% for the vertical padding
	\begin{tabularx}{\textwidth}{ | x || r | r | }
		\hline
		Studienfach 						& Absolutwerte 	& Prozentwerte \\ \hline\hline
		[A1] No experience at all 			& 10 			& 22.2\% \\ \hline
		[A2] Almost no experience 			& 15 			& 33.3\% \\ \hline
		[A3] Less than average experience 	& 3 			& 6.7\% \\ \hline
		[A4] Some experience 				& 10 			& 22.2\% \\ \hline
		[A5] More than average experience 	& 2 			& 4.4\% \\ \hline
		[A6] Experienced 					& 2 			& 4.4\% \\ \hline
		[A7] Very highly experienced 		& 3 			& 6.7\% \\ \hline
	\end{tabularx}
\end{table*}

\begin{table*}
	\caption{Numerische Auflistung der Ergebnisse der Frage "`How much experience do you have with VR?"'.}~\label{tab:sc_numbers_expVR}
	
	\setlength\tabcolsep{3pt}
	\renewcommand{\arraystretch}{1.4}% for the vertical padding
	\begin{tabularx}{\textwidth}{ | x | x | x | x | x | x | }
		\hline
		Min & Max & Range & Median & Mean  & Standard Deviation \\ \hline\hline
		1  & 7  & 6    & 2     & 2.93 & 1.78              \\ \hline
	\end{tabularx}
\end{table*}

\begin{table*}
	\caption{Verteilung der Antworten zur Frage "`How much experience do you have with AR?"'.}~\label{tab:sc_results_expAR}
	
	\setlength\tabcolsep{3pt}
	\renewcommand{\arraystretch}{1.4}% for the vertical padding
	\begin{tabularx}{\textwidth}{ | x || r | r | }
		\hline
		Studienfach 						& Absolutwerte 	& Prozentwerte \\ \hline\hline
		[A1] No experience at all 			& 17 			& 37.7\% \\ \hline
		[A2] Almost no experience 			& 10 			& 22.2\% \\ \hline
		[A3] Less than average experience 	& 7 			& 15.5\% \\ \hline
		[A4] Some experience 				& 8 			& 17.7\% \\ \hline
		[A5] More than average experience 	& 2 			& 4.4\% \\ \hline
		[A6] Experienced 					& 1 			& 2.2\% \\ \hline
		[A7] Very highly experienced 		& 0 			& 0.0\% \\ \hline
	\end{tabularx}
\end{table*}

\begin{table*}
	\caption{Numerische Auflistung der Ergebnisse der Frage "`How much experience do you have with AR?"'.}~\label{tab:sc_numbers_expAR}
	
	\setlength\tabcolsep{3pt}
	\renewcommand{\arraystretch}{1.4}% for the vertical padding
	\begin{tabularx}{\textwidth}{ | x | x | x | x | x | x | }
		\hline
		Min & Max & Range & Median & Mean  & Standard Deviation \\ \hline\hline
		1  & 6  & 5    & 2     & 2.36 & 1.38              \\ \hline
	\end{tabularx}
\end{table*}

\begin{table*}
	\caption{Verteilung der Antworten zur Frage "`What subject, if any, did you study or are you currently studying?"'.}~\label{tab:sc_results_study}
	
	\setlength\tabcolsep{3pt}
	\renewcommand{\arraystretch}{1.4}% for the vertical padding
	\begin{tabularx}{\textwidth}{ | x || r | r | }
		\hline
		Studienfach & Absolutwerte & Prozentwerte \\ \hline\hline
		Biologie & 1 & 2.2\% \\ \hline
		Informatik & 8 & 17.8\% \\ \hline
		Informationssystemtechnik & 1 & 2.2\% \\ \hline
		Mathematik & 1 & 2.2\% \\ \hline
		Medieninformatik & 18 & 40.0\% \\ \hline
		Physik & 3 & 6.7\% \\ \hline
		Psychologie & 2 & 4.4\% \\ \hline
		Software Engineering & 8 & 17.8\% \\ \hline
		Wirtschaftsmathematik & 1 & 2.2\% \\ \hline
		Wirtschaftsphysik & 2 & 4.4\% \\ \hline
	\end{tabularx}
\end{table*}

\begin{table*}
	\caption{Verteilung der Einstellungen des Stuhls.}~\label{tab:sc_results_chair}
	
	\setlength\tabcolsep{3pt}
	\renewcommand{\arraystretch}{1.4}% for the vertical padding
	\begin{tabularx}{\textwidth}{ | x || r | r | }
		\hline
		Winkeleinstellungen	in Grad	& Absolutwerte 	& Prozentwerte \\ \hline\hline
		0 							& 7 			& 15.6\% \\ \hline
		30 							& 23			& 51.1\% \\ \hline
		60	 						& 10 			& 22.2\% \\ \hline
		90							& 5 			& 11.1\% \\ \hline
	\end{tabularx}
\end{table*}

\begin{itemize}
	\captionof{anno}{Anmerkungen und Hinweise von Studienteilnehmern}
	\item "`Die Musik war sehr störend, um in einen Ruhezustand zu kommen"'
	\item "`Die VR Umgebung war schön gestaltet, aber die rumschwebenden Partikel waren eher verwirrend, ich dachte ich kann mit diesen interagieren"'
	\item "`Der Stuhl war sehr entspannend und bequem"'
	\item "`Es fiel mir schwer einzuschlafen, da ich zum 1. mal VR gemacht habe und dann neugierig war"'
	\item "`Die Musik war sehr angenehm"'
	\item "`Das lange gedrückt halten zur Interaktion war störend"'
	\item "`haptisches Feedback durch Controller wäre gut gewesen"'
	\item "`Die Brille war sehr unangenehm"'
	\item "`Der Ton fürs Wecken hat mich erschrocken"'
	\item "`Mit meiner Brille war es unangenehm die VR Brille zu tragen"'
	\item "`Ich konnte mich sehr gut entspannen, richtig eingeschlafen bin ich      aber nicht"'
	\item "`Die Interaktion mit dem Controller war sehr intuitiv"'
	\item "`Mir kam die Zeit zum entspannen deutlich länger als 15 Minuten vor "'
	\item "`Egal wie ich die Brille verstellte, richtig scharf konnte ich nie sehen "'
	\item "`Noch fünf bis zehn Minuten länger und ich wäre komplett eingeschlafen "'

\end{itemize}

\begin{table*}
	\caption{Wahrgenommene Schlafdauer.}~\label{tab:sleepduration}
	
	\setlength\tabcolsep{3pt}
	\renewcommand{\arraystretch}{1.4}% for the vertical padding
	\begin{tabularx}{\textwidth}{ | x || r | r | }
		\hline
		wahrgenommene Schlafdauer in min & Absolutwerte & Prozentwerte \\ \hline\hline
		8						   	     & 2			   & 4.4\% \\ \hline
		10   					         & 5			   & 11.1\% \\ \hline
		11						   	     & 1 		   & 2.2\% \\ \hline
		12						   	     & 3			   & 6.7\% \\ \hline
		13							     & 2			   & 4.4\% \\ \hline
		14							     & 1			   & 2.2\% \\ \hline
		10-15	      					 & 3		 & 6.7\% \\ \hline
		15							     & 13		 & 28.9\% \\ \hline
		15-20							 & 1		 & 2.2\% \\ \hline
		17								 & 2		 & 4.4\% \\ \hline
		18								 & 3		 & 6.7\% \\ \hline
		18,5							 & 1		 & 2.2\% \\ \hline
		19								 & 1		 & 2.2\% \\ \hline
		20								 & 6		 & 13.3\% \\ \hline
		30								 & 1		 & 2.2\% \\ \hline
	\end{tabularx}
\end{table*}
\todoAll{Als Idee: Diese Tabelle etwas kürzen und zusammenfassen. Also z.B. Zeitintervalle [5-10), [10-15), [15], (15-20], (20-25], (25-30]. Dennis Fragen!}

\begin{table*}
	\caption{Verteilung der Antworten zur Frage "`Hast du geschlafen?"' .}~\label{tab:sleepstatus}
	
	\setlength\tabcolsep{3pt}
	\renewcommand{\arraystretch}{1.4}% for the vertical padding
	\begin{tabularx}{\textwidth}{ | x || r | r | }
		\hline
		Schlafmodus					& Absolutwerte 	& Prozentwerte \\ \hline\hline
		geschlafen 					& 9 			& 20.0\% \\ \hline
		gedöst/kurz vor eingeschlafen	& 12			& 26.7\% \\ \hline
		meditiert					& 3			& 6.7\% \\ \hline
		nicht geschlafen			& 21 			& 46.7\% \\ \hline
	\end{tabularx}
\end{table*}


\backmatter %%%%%%%%%%%%%%%%%%%%%%%%%%%%%%%%%%%%%%%%%%%%%%%%%%%%%%%%%%%%%%%%%%

\listoffigures	% Abbildungsverzeichnis
\listoftables	% Tabellenverzeichnis

%\bibliographystyle{natdin}
\bibliographystyle{IEEEtranS}	% alternativer Stil
\bibliography{MCI-Resync-Documentation}

\cleardoublepage
\clearscrheadfoot
% \declaration		% Erklärung, siehe diplom-mi-eng.cls

\end{document}
