% Achtung: Vor dem Verwenden dieser Vorlage unbedingt die readme lesen!
\documentclass{diplom-mi-eng}

%\usepackage[showframe]{geometry}

% debug only
%\usepackage{showframe} 

%\usepackage{longtable}
\usepackage{multirow}
\usepackage{tabularx}
\usepackage{rotating}
\usepackage{pdflscape}
\usepackage{url}
\usepackage{enumitem}
\usepackage{siunitx}
\usepackage{framed}
\usepackage{enumitem}
\usepackage{amsmath}
\usepackage{eurosym}
\usepackage{caption}
\usepackage{xargs}                      % Use more than one optional parameter in a new commands
\usepackage[pdftex,dvipsnames]{xcolor}  % Coloured text etc.

\newcolumntype{x}{>{\raggedright\arraybackslash}X}
\newcolumntype{y}{>{\raggedleft\arraybackslash}X}

\usepackage{ntheorem}
\theoremseparator{:}
\newtheorem{hyp}{Hypothese}

% TODO notes
\usepackage[colorinlistoftodos,prependcaption,textsize=tiny]{todonotes}
\newcommandx{\todoSab}[2][1=]{\todo[linecolor=red,backgroundcolor=red!25,bordercolor=red,#1]{#2}}
\newcommandx{\todoLuc}[2][1=]{\todo[linecolor=blue,backgroundcolor=blue!25,bordercolor=blue,#1]{#2}}
\newcommandx{\todoTob}[2][1=]{\todo[linecolor=OliveGreen,backgroundcolor=OliveGreen!25,bordercolor=OliveGreen,#1]{#2}}
\newcommandx{\todoAll}[2][1=]{\todo[linecolor=Plum,backgroundcolor=Plum!25,bordercolor=Plum,#1]{#2}}
\newcommand{\todos}{\newpage\listoftodos[Todos]}

% Default ist serifenlose-Schrift (Helvetica), wenn Serifenschrift (Palatino)
% gewünscht ist, einfach folgende Commands auskommentieren.
\renewcommand{\sfdefault}{phv}
\renewcommand{\rmdefault}{phv}
\renewcommand{\ttdefault}{pcr}

\newcommand{\projectName}{Resync}
\newcommand{\projectSubline}{Resync}


% Bitte folgende Variablen anpassen:
\title{\projectName: \projectSubline} 
\authorA{Sabrina Böhm}
\emailA{sabrina.boehm@uni-ulm.de}
\authorB{Luca Porta}
\emailB{luca.porta@uni-ulm.de}
\authorC{Tobias Lahmann}
\emailC{tobias.lahmann@uni-ulm.de}
\matnr{828398}			% Matrikelnummer

\type{Anwendungsfach} 	%Art der Arbeit, z.B. Diplomarbeit, Masterarbeit,Bachelorarbeit
\jahr{2019}

\fakultaet{Ingenieurwissenschaften, \\Informatik und Psychologie}
\institut{Institut für Medieninformatik}

\gutachterA{Prof. Dr. Enrico Rukzio}
%\gutachterB{Prof. ...}
\betreuer{Dennis Wolf, M.Sc.}

% Ende User-Variablen

\begin{document}
\renewcommand{\contentsname}{Inhaltsverzeichnis}
\frontmatter %%%%%%%%%%%%%%%%%%%%%%%%%%%%%%%%%%%%%%%%%%%%%%%%%%%%%%%%%%%%%%%%%

\maketitle	% Titelblatt, siehe diplom-mi.cls
\todoAll{Einen vernünftigen Subtitel hinzufügen}

\clearpage
\thispagestyle{empty}
{	\small\sffamily
	\flushleft
	~\vfill
	Fassung vom \today\\[1cm]
	\copyrightinfo\\[.5cm]	% Copyright Notice - siehe diplom-mi.cls
	Satz: PDF-\LaTeXe
}


\setstretch{1.2}	% Zeilenabstand ab hier 1.2

\begin{abstract}
	Die Untersuchung der Heranführung einer Person an ein Problem, wenn diese im Vorfeld keine Informationen über die Aufgabe hat. % Introduction. In one sentence, what’s the topic?
	Wir untersuchen die unterschiedlichen Arten jemanden aus einem trance-ähnlichen Zustand, wie er nach dem Schlafen auftreten kann, in einen Bewussten zu überführen und wie diese Person in diesem Vorgang unterstützt werden kann. % State the problem you tackle
	Vorbereitungen auf Aufgaben und noch spezieller die Lenkung der Aufmerksamkeit bei dieser wurde noch ungenügend untersucht. % Summarize (in one sentence) why nobody else has adequately answered the research question yet
	Wir untersuchen die Problemstellung im Kontext von VR und nutzen dies um dadurch unterschiedliche Parameter des 'Aufweckens' sowie Designprinzipien zu untersuchen. % Explain, in one sentence, how you tackled the research question
	In einer between-subject Studie wurden die Unterschiede der möglichen 'Aufweckarten' untersucht. % In one sentence, how did you go about doing the research that follows from your big idea
	Aktuell können unsere Vermutungen noch nicht bestätigt werden. % As a single sentence, what’s the key impact of your research?
\end{abstract}


\tableofcontents

\mainmatter %%%%%%%%%%%%%%%%%%%%%%%%%%%%%%%%%%%%%%%%%%%%%%%%%%%%%%%%%%%%%%%%%%

% Ab hier Kapitel einbinden
\chapter{Introduction}

This is the introduction

\section{Motivation}

Virtuelle und Augmentierte Umgebungen begleiten uns bereits seit einigen Jahren.  Die Produkte, welche dabei zum Einsatz kommen, sind zum jetzigen Zeitpunkt aufgrund der unpraktischen Größe nicht für das alltägliche Umfeld des Durchschnittsbürgers geeignet und angesichts der hohen Preise auch nicht sonderlich weit verbreitet.
Werden in Zukunft diese Systeme kleiner, leichter, oder sogar permanent mit dem menschlichen Körper verbunden, wird es möglich sein, jeden Bereich des Lebens zu augmentieren. AR/VR soll ein stetiger Begleiter sein und dem Nutzer auch in anspruchsvollen oder unerwarteten Situationen unter die Arme greifen. Eine solche unerwartete Situation könnte beispielsweise direkt nach dem Aufwachen aus dem Schlaf der Fall sein. So können in autonomen Autos Aufgaben vom Fahrer übernommen, im Nachtdienst eines Sicherheitsunternehmens kritische Vorgänge überwacht, oder am Morgen Herausforderungen vom Benutzer verlangt werden, welche ein hohes Maß an Aufmerksamkeit erfordern.
Wir möchten herausfinden wie schnell und effizient ein Benutzer auf diese Aufgaben Vorbereitet werden kann. Vor allem die Frage, zu welchem Zeitpunkt der Benutzer geweckt wird gilt es hierbei zu erforschen.

\section{Herangehensweise}\label{sec:approach}  

Im ersten Abschnitt des Projekts betrieben wir Recherche zu den Themen Schlafen, Aufwachen, VR/AR sowie zum Bereich Aufgaben bewältigen. Nachdem reichlich Recherche betrieben wurde stellten wir Hypothesen auf, die hauptsächlich die Parameter des Weckens, sowie auch die effektive Aufgabenbewältigung betreffen. Mit diesen Hypothesen sind wir in die nächste Phase eingestiegen. 
Wir erstellten eine virtuelle Umgebung mittels Unity 3D\footnote{~Unity3D~\url{https://unity3d.com}} um eine entspannende Atmosphäre zu erschaffen. Um den Probanden eine entspannte physische Atmosphäre zu bieten wurde für die Studie ein bequemer Bürostuhl mit verstellbarer Lehne genutzt. Nachdem wir die erste Studie durchgeführt hatten, änderten wir den 'Aufweckparameter' und führten eine zweite Studie mit neuen Probanden durch. \todoSab{Hier welche Zeitform? -> Präsens}
\todoSab{Hier sollte eher so etwas stehen, wie: 'Welche methoden haben wir gewählt und warum?' 'Welche Technologien haben wir verwendet und warum?' 'Wie haben wir mögliche Probleme einer Studie bewältigt und welche Daten wollen wir überhaupt erheben'. Also die grundlegende Herangehensweise. Wie das Projekt abgelaufen ist soll in 'Studiendurchführung'. Aber nicht zu viel, weil es da ja eigentlich ausführlich beschrieben wird.}
\todoTob{das fällt mir grad irgendwie schwer beim formulieren.. vielleicht kannst du das zuerst machen und ich ergänz dann oder so}


\chapter{Verwandte Forschung}\label{sec:relatedWork}
Um einen Überblick über die Thematik zu erlangen untersuchten wir die bestehende verwandte Forschung. Dies haben wir in einzelne Teilgebiete unterteilt. Zuerst betrachteten wir den Schlafzustand, aus welchem heraus wir unsere Untersuchungen durchführen wollten. Anschließend das unmittelbar auf den Schlaf folgende Aufwachen und als dritten Abschnitt das Erledigen von anfallenden Aufgaben nach den ersten zwei Phasen.



\section{Schlafen}\label{sec:relatedWork.schlafen}

- \cite{dinges1985assessing}:\\

Schlafentzug verursacht tiefere Kurzschlaf-Phasen~\cite{dinges1985assessing}. Sollte eine optimale Performance in Aufgaben benötigt werden sollten Kurzschlaf-Phasen vermieden werden~\cite{dinges1985assessing}. Schlummern sowie Nickerchen sollten gemacht werden bevor ein gravierender Schlafentzug eintritt~\cite{dinges1985assessing}.\\
Bei stupiden langanhaltenden Arbeitsvorgängen kann es schnell passieren, dass man müde wird. Umso länger man im gleichbleibenden Trott ist wird der Schlafmangel immer größer. Diese Probleme kann man einerseits durch Nickerchen/schlafen lindern, aber es erhöht das Risiko, dass die Person bei abruptem Erwachen Schwierigkeiten hat, zu funktionieren. Mit unterschiedlichen Schlafentzugszeiten (6, 18, 30, 42 oder 54) wurde eine Studie durchgeführt, um zu testen wie gut die Probanden auf Ereignisse reagieren können. Nach den verschiedenen zeiten war es den Nutzern gestattet 2 Stunden zu schlafen. Schlafentzug erhöhte die Menge an Tiefschlaf in den Nickerchen, was mit einer größeren Abnahme der kognitiven Leistung nach dem Nickerchen verbunden war.\\
Die Manipulation von zunehmenden Schlafmangel führte zu einem tieferen Schlaf, was siginifikante Leistungseinbußen hervorruf. Dies war für die kognitive Leistung am dramatischsten. Die direkt nach dem schlafen auftretende Veriwrrung wird Schlafträgheit genannt. Probanden die 42 bis 54 Stunden lang schlaflos waren wurde diese Schlafträgheit durch tiefgreifende Desorientierung, Unfähigkeit und Verwirrung gekennzeichnet.
\texttt{sabrina fazit: paper is schwer verständlich für mich und Hauptaussagen stehen oben}\\

- \cite{kraemer2000time}:\\

Abhängig von der Tageszeit existieren Unterschiede in der Performance, so wie der Selbsteinschätzung und anderer psychologischer Parameter bei voll ausgeschlafenen Probanden (12 Stunden Schlaf)~\cite{kraemer2000time}.



\section{Aufwachen}\label{sec:relatedWork.aufwachen}

Bis zu 2 Stunden nach dem Aufwachen kann die subjektive Aufmerksamkeit und die kognitive Leistungsfähigkeit noch beeinträchtigt sein~\cite{jewett1999time}. Eine Herunterregulierung der Körpertemperatur und der damit einhergehende geringere geistige Leistungsfähigkeit könnte der Auslöser sein für den Zustand der Schlafträgheit~\cite{dinges1990you}. Sollte dies stimmen kann erwartet werden, dass jegliche Aktivität, die die Körpertemperatur erhöht dem schlaffen Gefühl entgegenwirkt, das nach dem Schlafen einige Zeit einsetzt und erst mit der Zeit abgebaut wird~\cite{jewett1999time}. Jewett et. al. fanden heraus, dass aber weder die Helligkeit der Umgebung noch andere Aktivitäten, die kurz nach dem Aufwachen erledigt wurden (Essen, duschen, etc.) signifikant die Aufmerksamkeit noch die Schlafträgheit oder deren Abbau beeinflussten~\cite{jewett1999time}.

In einer Situation von Müdigkeit, die direkt nach dem Aufwachen einsetzt und erst über die Zeit abgebaut wird konnten Probanden einer Studie noch einfache soziale Interaktion durchführen~\cite{dinges1990you}. Die funktionale Deafferenzierung, wie sie von Broughton genannt wurde~\cite{broughton1968sleep} um die niedrigen Hirnaktivitäten nach dem Aufwachen zu beschreiben, erschweren die Aufbringung der mentalen Kapazitäten für komplexe Aufgaben nach dem Erwachen. Daher müssen, in all den Situationen, welche eine erhöhte Leistungsfähigkeit benötigen, die unumgänglichen Effekte der Schlafträgheit im Vorfeld beachtet und denen, die diese Aufgabe erledigen sollen, einfache Tools zur Unterstützung gegeben werden~\cite{ferrara2000sleep}. Aktuell könnten alarmierende Faktoren verwendet werden um diese Ziele zu erreichen, aber weitere Forschung muss bestätigen welcher der Faktoren am effizientesten ist~\cite{ferrara2000sleep}.
\section{Aufgaben erledigen}\label{sec:relatedWork.aufgaben}

Räumliche, aufmerksamkeitssensitive Darstellungen sind effektiv~\cite{bonanni2005attention}. Exogene Hinweise können dem Nutzer helfen sich auch in unbekannten Umgebungen zurechtzufinden~\cite{bonanni2005attention}.

Es existieren unterschiedliche Herangehensweisen um Fahrer in Autos über eine auftretende Gefahrensituation zu informieren. Hierbei wurden textuelle Informationen den grafischen vorgezogen.~\cite{green1995driver}

Kulturelle Unterschiede bewirken, dass sich Fahrer im Straßenverkehr auf unterschiedliche Dinge konzentrieren und im Anschluss an unterschiedliche Details erinnern~\cite{shinohara2017visual}.

Zur Vorbereitung auf die Objekte oder Vorgänge in der Umgebung von Menschen können 3D Marker verwendet werden, die in die Richtung des Objekts oder Geschehens weisen. Eine 3D Darstellung ist nach Chittaro und Burigat mindestens genauso effektiv wie eine 2D Darstellung. Sie bietet jedoch den Vorteil, dass Nutzer auch in der dritten Dimension, der Höhe, auf wichtige Punkte hingewiesen werden können~\cite{chittaro20043d}.\\

-\cite{aschoff1998human}\\
Wahrnehmung von Zeitintervallen und wie das mit Körpertemperatur und Dauer des Wachseins zusammenhängt: Die menschliche Zeitwahrnehmung kann in zwei verschiedene Klassen eingeteilt werden, die sich in ihrer Interaktion mit dem zirkadianen(=tagesrythmischen) System unterscheiden: Kurze Zeitintervalle im Sekundenbereich (bis zu ca. 2 min) werden nicht von Veränderungen des Schlaf-Wach-Zyklus beeinflusst, jedoch unter Bedingungen der zeitlichen Isolation können Veränderungen aufgezeichnet werden. Die Zeitschätzung wurde bei sieben Probanden wüber einen gewissen zeitraum untersucht, bei dem die Probanden von Zeithinweisen isoliert wurden. Kurze und lange Zeitintervalle werden über verschiedene Mechanismen subjektiv erlebt. Dabei weisen die kurzen Intervalle wenig auffallendes und bei den langen Intervallen (~ 1 Stunde) interessante Feststellungen auf. Die Nutzer mussten 1 Stunde Zeitintervalle einschätzen und es stellte sich heraus, dass die Einschätzungen in Verbindung mit der Zeit in der der Nutzer schon wach ist  korreliert. Zudem gab es bei kleinen Zeitintervallen (~ 2 Minuten) eine negative Korrelation mit der Temperatur und eine positive mit der Beleuchtungsintesität. Bei längeren Zeitintervallen gibt es keine signifikanten Auffälligkeiten im Bezug auf die Lichtintensität.
\section{Übergänge in und aus der Virtuellen Realität}\label{sec:relatedWork.vr}

\todoTob{2.4 zusammen-/umschreiben}

Die Studie \texttt{The Dream is Collapsing} im Bereich Virtual Reality (VR) hat die Erfahrungen der Nutzer mit Immersion, Präsenz, Simulatorkrankheiten und Lerneffekten untersucht. Die momentane Erfahrungen, VR zu verlassen und in die Realität zurückzukehren, sind noch nicht gut erörtert.\\
Der Akt des Ein- und Ausstiegs aus VR also der Moment des An- und Ausziehens des Headsets spielt eine wichtige Rolle in der gesamten Benutzererfahrung, erhält aber wenig Aufmerksamkeit. Eine Erkenntnis ist, dass der 'Moment des Ausstiegs' eine ungenutzte Gelegenheit sein kann bestimmte Auswirkungen beim Nutzer herzvorzurufen. Die Konstruktion für diesen Moment könnte zu vielen Zwecken genutzt werden. Zum Beispiel könnten Designer jede Überraschung beim Entfernen des Headsets verringern, indem sie den Benutzer auf alle Änderungen in der realen Umgebung aufmerksam machen bzw. darauf eingehen, um den Übergang möglichst 'unauffällig' zu gestalten. So könnte sich beispielsweise eine VR-Anwendung an das verblassende Licht in der Realität anpassen.
Andernfalls könnten gewünschte 'harte cuts' für bestimmte Szenarien konstruiert werden, wie zum Beispiel Horror Spiele oder das Erzwingen bestimmter Angstsituationen.\\
In der Studie werden vier Gruppen gebildet, die einzelne Bereiche abdecken: Gaming, Illusionen, Wahrnehmungsverfälschung und kognitive Aufgaben. \\

\begin{itemize}
	\item Gaming: Ein großer Teil der kommerziellen Bemühungen der VR-Entwicklung richtet sich auf Unterhaltung und Gaming. Gaming verwendet eine große Anzahl von Mechaniken, um immersive Spiele zu entwickeln.
	\item Illusion: In letzter Zeit wurde die Aufmerksamkeit auf VR-Illusionen erhöht. Mit Schwerpunkt auf haptische Illusionen und wandelnde Illusionen. Der Moment des Ausstiegs aus Illusionen ist besonders interessant, da er dem Moment entspricht, in dem die Nutzer erkennen, dass sie einer Illusion ausgesetzt waren und sich schnell innerhalb der realen Umgebung neu orientieren müssen.
	\item Wahrnehmungsverfälschung: Es gibt Veränderungen in der Wahrnehmung, die sich daraus ergeben, dass Menschen unterschiedliche körperliche Merkmale haben, wie bspw. das Alter oder die Herkunft. Ein wichtiger Aspekt stellt auch die Körpergröße dar. In diesem Fall angelehnt an den Proteus-Effekt, bei dem Teilnehmer unterschiedlicher Größe unterschiedliche Reaktionen auf Reize aufzeigen. Der Fokus liegt hier auf der Erforschung der Umgebung und der Erledigung einer Aufgabe aus einer anderen Höhenperspektive. Der Moment des Ausstiegs beinhaltet sowohl eine Körperneuausrichtung, als auch eine perspektivische Neuausrichtung und Reflexion.
	\item Kognitive Aufgaben: Virtuelle Umgebungen werden unter Anderem als Lehrmittel verwendet und gewisse Literatur hat den Lerneffekt von VR untersucht. Die Hervorhebung des Lernens und der Bildung stellt eine kognitive Belastung für den Nutzer dar. Der Zeitpunkt des Ausstiegs, in welchem diese Last abgebaut wird, kann zu einer anderen Benutzererfahrung beim Verlassen der virtuellen Umgebung führen.
\end{itemize}

Durch eine thematischen Analyse kristalisierten sich fünf Aspekte hervor: Raum, Kontrolle, Zeit, Sozialität und sensorische Anpassung begutachtet.\\
Ein auffälliger Punkt ist die räumliche Desorientierung unabhängig von der Komplexität der VR-Szene.\\ 
 Die Teilnehmer beschreiben einen Übergang vom VR-Verlassen hin zur Realität, zum Beispiel zuerst mental und dann physisch.
Bisher war die Erfahrung von VR ausschließlich innerhalb des VR-Headsets gebunden, aber der Moment des Ausstiegs kann eine Gelegenheit darstellen, diese strikte Grenze zwischen virtueller und realer Welt in Frage zu stellen.\\
Teilnehmer beschreiben Veränderungen in ihrem Kontrollgefühl als sie den Übergang zwischen VR und Realität vollzogen haben. Nutzer fühlen sich unter Anderem erleichtert nachdem sie die VR Brille abgenommen haben und meinten dass diese 'Welt' vertrauensvoller ist. Es hat mit Sicherheit zu tun, die dadurch entsteht, dass man weiß, dass das Headset entfernt werden kann, um sofort in die reale Umgebung zurückzukehren.\\
Zudem fühlen sich Probanden schnell desorientiert durch Unterschiede der VR zur Realität, die noch nicht umgesetzt werden können, wie zum Beispiel nicht statische Objekte und Umgebungen. Der globale Orientierungssinn kann also durch statische Ausrichtung der VR Elemente beibehalten werden.
Bis heute existiert eine Lücke zwischen dem menschlichen Verständnis von VR-Erfahrungen im Moment des Ausstiegs aus VR, der wirklichen Erfahrung in VR und dem Auftreten von VR-Nachwirkungen. \cite{knibbe2018dream}\\


-\cite{bonanni2005attention}\\
AR kann im Bezug auf Aufgabenbewältigung herangezogen werden. 
Die Objekte und Oberflächen einer Umgebung können mit digitalen Schnittstellen überlagert werden, um sie einfacher und sicherer für anstehende Aufgaben zu gestalten. Sobald Informationen überall im Raum projiziert werden können, ist es wichtig, die Informationen so zu gestalten, dass die Aufmerksamkeit der Benutzer optimal genutzt wird und keine negativen Auswirkungen durch beispielsweise Überlagerung gewisser Elemente und Überforderung des Probanden hervozurufen. Pilotstudien und Nutzerauswertungen zeigen, dass räumliche, aufmerksamkeitsstarke Projektionen am nützlichsten waren. Exogene Hinweise können für den Nutzer auch in vertrauten Umgebungen nützlich sein. Unter Berücksichtigung der Position eines Nutzers und seiner Leistung können Schnittstellen bereitgestellt werden, die die Aufgaben unterstützen und nicht stören. Multimodale erweiterte Interaktionen können eine Vielzahl von Aktivitäten verbessern, einschließlich verfahrenstechnischer Aktivitäten.\\

- \cite{chittaro20043d}\\
3D-Lokalisierung als Navigationshilfe in virtuellen Umgebungen: Die Navigationsunterstützung durch die Benutzeroberflächen von Virtual Environments ist oft unzureichend und meist zu komplex, insbesondere bei großen VE. Ein schlecht aufgebautes Nutzerinterface führt dazu, dass Menschen desorientiert werden und sich verlaufen. Diese Probleme treten immer häufiger in großen/ komplexen virtuellen Umgebungen auf und wollen eigentlich vermieden werden. Als Navigationshilfe sollte ein einfach zu steurendes Werkzeug genügen mit wenig Funktionen, so werden Fehler in der Interaktion vermieden. Zudem führen unnötig angezeigte Informationen ebenso zu Verwirrung der Probanden. Auf diese solltet verzichtet werden und man sollte stattdessen durch kleine Interaktionen oder 'Nachfragen' Zugriff zu anderen Features erhalten. \\
In der Studie wurden 4 Gruppen aufgestellt wobei die erste einen 3D Pfeil zur Hilfe in der virtuellen Umgebung hatte, die zweite Gruppe einen 3D Pfeil, die dritte Gruppe eine 2D-Hilfe, die auf einer Radarmetapher basiert und als letztes die vierte Gruppe, welcher jegliche Hilfe verwährt blieb. Es stellte sich heraus, dass die Gruppe ohne Hilfe deutlich schlechter im Bezug auf die Zeit in der Aufgaben bewältigt wurden abschnitten, jedoch bei den anderen drei Gruppen keine signifikanten Unterschiede aufgetreten sind.


\chapter{Lösungsansatz}

Im Folgenden werden unsere Herangehensweisen an das Problem beschrieben. Wir untersuchen den Übergang zwischen einem schläfrigen Zustand, ohne kognitive Beanspruchung, und einem wachen Zustand, in dem Aufgaben von der Testperson übernommen werden können. Hierzu bietet sich aus zeitlichen Gründen eine Untersuchung mittels virtueller Realität an. Wir betrachten außerdem eine Reihe von Aufgaben, die in VR durchgeführt werden und einen Bezug auf reale Situationen im Kontext Autofahren aufweisen.

\section{Idee}

Wir möchten die Teilnehmer der Studie dazu bringen einen Zustand zu erreichen in dem das fehlerfreie Erledigen von Aufgaben eine gewisse geistige Anstrengung aufweist. Dies ist vergleichbar damit, dass im Straßenverkehr das autonome Fahren dem Fahrer erlaubt seine Aufmerksamkeit von der Straße zu nehmen, seinen Sitz zu kippen und die Augen zu schließen. Sollte der Fahrer daraufhin von seinem Fahrzeug aufgefordert werden in eine Situation einzugreifen und eine Entscheidung zu treffen, wie beispielsweise einen schwierigen Abschnitt der Strecke selbst zu fahren, oder eine Entscheidung zu treffen, muss das Fahrzeug den Fahrer aufwecken und informieren. In diesem Zustand könnten Informationen nur schwierig aufgenommen werden und eine Aufgabe, die in diesem Zeitraum gestellt wird könnte so mit geringerem Erfolg erledigt werden, als wenn der Fahrer voll aufnahmefähig ist.

Des Weiteren können auch andere schwierige Entscheidungen von autonomen Fahrzeugen an den Benutzer übergeben werden. Die `"Moral Machine'' von Awad et al.~\cite{awad2018moral} versucht dieses Thema durch das Trainieren eines neuronalen Netzwerks anzugehen. Hierbei wird die moralische Entscheidung über einen unumgänglichen Tod und ethische Konflikte durch das Netzwerk bewertet~\cite{awad2018moral}. Hierbei ist hervorzuheben, dass jegliche programmierbare Entscheidung den Tod Beteiligter verhindern soll. 
Es kann nicht gesagt werden, dass menschliche Fahrer in allen Situationen die richtige Entscheidung treffen, jedoch kann es ebenfalls nicht festgelegt werden, wie eine Maschine in einem kritischen Fall reagieren soll, ohne eine komplexe ethische Diskussion anzuregen~\cite{awad2018moral}.
Bevor man autonome Autos befähigen kann ethische Entscheidungen zu treffen, muss ein globales Gespräch geführt werden um Präferenzen von Anbietern und Nutzern zu vermitteln.

Wir untersuchen in dieser Arbeit nicht die Auswirkungen des Schlafens, oder des Mangels an Schlaf auf das Gehirn, sondern viel mehr die Überführung von einem trägen oder schläfrigen Zustand in einen Wachen. 

Um eine möglichst kontrollierbare Testumgebung mit vergleichbaren Ergebnissen zu haben nutzen wir VR in einem separaten Raum der Universität Ulm. Hier werden die Teilnehmer gebeten mit dem VR-HMD in einer initialen Phase zu entspannen. Das Ziel ist es die beschriebene Situation der geringeren Aufnahmefähigkeit und Schläfrigkeit zu erzeugen, wie er auch nach dem Schlafen auftreten kann. 

Im Anschluss an die Ruhephase werden die Nutzer auf unterschiedliche Arten "`aufgeweckt' und ihnen werden drei Aufgaben gestellt. Die von uns untersuchten Arten des Weckens sind Licht und Ton. Wobei die Einstellung des Lichts noch in zwei weitere Gruppen unterteilt ist. Für Gruppe eins wird das licht innerhalb von 5 Sekunden von 0\% Intensität auf 100\% erhöht, bei Gruppe zwei geschieht dies Über einen Zeitraum von 20 Sekunden. 
Allen drei Gruppen werden im Anschluss die Aufgaben iterativ präsentiert.

Auf dieser Grundlage formulieren und untersuchen wir die folgenden Hypothesen:

\begin{hyp}[H\ref{hyp:lichtSchneller}]\label{hyp:lichtSchneller}
	Menschen, die mit Licht geweckt werden, können sich in kürzerer Zeit auf eine gestellte Aufgabe einstellen, als Menschen, die mit Ton geweckt werden.
\end{hyp}

\begin{hyp}[H\ref{hyp:lichtErfolgreicher}]\label{hyp:lichtErfolgreicher}
	Menschen die mit Licht geweckt werden, können eine gestellte Aufgabe mit weniger Fehlern erledigen, als Menschen, die mit Ton geweckt werden.
\end{hyp}

\begin{hyp}[H\ref{hyp:langKurzSchneller}]\label{hyp:langKurzSchneller}
	Menschen die langsam geweckt werden können sich in kürzerer Zeit auf eine gestellte Aufgabe einstellen, als Menschen, die abrupt aus dem Schlaf gerissen werden.
\end{hyp}

und 

\begin{hyp}[H\ref{hyp:langKurzErfolgreicher}]\label{hyp:langKurzErfolgreicher}
	Menschen die langsam geweckt werden können eine gestellte Aufgabe mit weniger Fehlern erledigen, als Menschen, die abrupt aus dem Schlaf gerissen werden.
\end{hyp}

\section{Zeitplan}
Die Daten beziehen sich auf den Zeitpunkt zu dem der jeweilige Schritt abgeschlossen sein sollte.
\begin{itemize}
	\item \textbf{Dezember 2018} Recherche bestehender Forschung
	\item \textbf{Januar - Mai 2019} Beginnen mit der Implementierung und Modellierung der digitalen Umgebung in Unity3D und Vorbereitung für den Studiendurchlauf
	\item \textbf{Juni 2019} Erste Studie durchführen
	\item \textbf{Juni 2019} Abschluss der Studie und Erweiterung auf Zehn weitere Probanden
	\item \textbf{Juli 2019} Einführung auf eine weitere 15-köpfige Testgruppe, welche mit Ton geweckt wird und Implementierung und Entwurf dieser neuen angepassten Studie
	\item \textbf{August 2019} Abschluss der gesamten Studie
	\item \textbf{September 2019} Auswertung der Ergebnisse
	\item \textbf{Oktober - November 2019} Erstellen der Ausarbeitung und Präsentation
\end{itemize}

\section{Testaufbau}
Sitzend werden Probanden erst in einen entspannten Zustand versetzt. In diesem verweilen sie möglichst ohne Ablenkung, bis sich eine Gelassenheit oder Trägheit einstellt. Diese kann von entspanntem Sitzen bis hin zum Schlaf führen. Eine genaue Zeitspanne hierfür kann zwischen Probanden variieren und muss in Tests bestimmt werden.

Nachfolgend wird der Teilnehmer aus diesem Zustand geleitet und mit einer Aufgabe konfrontiert. Während der Erledigung dieser werden unterschiedliche Parameter aufgezeichnet und später ausgewertet. Die erfassten Parameter sind die folgenden:

\begin{itemize}
	\item Zeit in der eine Aufgabe erledigt wird
	\item Fehlerrate bei der Erledigung der Aufgabe
	\item Blickrichtung
\end{itemize}

Die erste Studie umfasst ungefähr 30 Minuten, hierfür werden die Teilnehmer mit fünf Euro entlohnt. Der Ablauf der Studie Umfasst folgende Punkte:

\begin{enumerate}
	\item \textbf{5 Minuten} Vorbereitung und Einführung in den Studienablauf inklusive der Bedienung der VR-Umgebung
	\item \textbf{15 Minuten} Beruhigungsphase bis hin zum Schlafen
	\item \textbf{3-5 Minuten} Aufgaben lösen
	\item \textbf{5 Minuten} Fragebögen beantworten
\end{enumerate}

Es handelt sich um eine between subject Studie. Im ersten Durchlauf erfassen wir die genannten Parameter unter der Betrachtung der Zeit, in der die virtuelle Umgebung erhellt wird. Die genauen Zeiten werden in einer Testphase während des Implementierens eingegrenzt.

Zudem werden die Probanden entweder mit Ton oder durch einen Lichtreiz "`geweckt"'. Auch dieser Parameter wird in einer Testphase experimentell angenähert. Nähere Informationen zum genauen Studienablauf werden im Abschnitt der Studiendurchführung erörtert.

\subsection{Studienumgebung}

Um die Studie durchführen zu können, haben wir unser Equipment in einem kleinen abgeschotteten Hinterzimmer eines Computerlabors aufgebaut. Dadurch dass die Probanden keine stehenden oder gar laufenden Bewegungen vollziehen mussten, reichte ein Bereich, in dem sich der Stuhl geradeso drehen konnte. Der Raum war mit einem Fenster versehen und so kam immer Tageslicht ins Zimmer herein, was die Probanden jedoch durch das Tragen der VR Brille während der Durchführung der Studie nicht wahrnehmen konnten, sondern nur in dem Anfangsgespräch und beim anschließenden Fragebogenausfüllen. Da im Nebenzimmer ebenfalls Studien durchgeführt wurden, waren Störgeräusche nicht immer zu verhindern. So kam es vor, dass manche Probanden ohne jegliche Störgeräusche die Studie absolvieren konnten, wohingegen bei einigen anderen verschiedenste, relativ leise, dennoch hörbare Störungen auftraten. Wir achteten nicht nur darauf, dass die räumlichen Begebenheiten bei jedem Teilnehmer der Studie gleich blieben, sondern auch dass die Art der Begrüßung und des Vorworts stets gleich oder zumindest ähnlich abliefen. Darüberhinaus legten wir auch großen Wert darauf, dass Faktoren wie zum Beispiel die Stuhllokalisation und die Lichteinstellung für alle Probanden gleich waren.




\section{Implementierung}

Softwareseitig haben wir eine digitale Testumgebung mit der Unity 3D\footnote{~Unity3D~\url{https://unity3d.com}} Game-Engine erstellt. Für die Untersuchung der Hypothesen wurde eine ruhige Umgebung benötigt, die die Probanden in der richtigen Art und Weise dabei unterstützt in einen schlafähnlichen Zustand zu gelangen. 
Das gewählte Interface für die hier durchgeführten Untersuchungen ist das HTC Vive\footnote{~HTC Vive~\url{https://www.vive.com}} \textit{Head Mounted Device} (HMD) mit den zugehörigen Controllern. 
Bewegung (Locomotion) ist, bis auf Kopfbewegungen, innerhalb der digitalen Umgebung nicht vorgesehen, da die Teilnehmer sitzend Aufgaben erledigen. 
Die Eingabemethoden zur Bewältigung der gestellten Aufgaben werden mit Zeigeoperationen innerhalb der virtuellen Umgebung realisiert. 

\subsection{Codekonventionen und Syntax}

Um eine einheitliche Code-Qualität zu gewährleisten haben wir einige Konventionen festgelegt. Diese sollen außerdem sicherstellen, dass Code, der von unterschiedlichen Entwicklern produziert wurde von allen Beteiligten schnell verstanden und verändert werden kann.
Ein Auszug aus diesen Konventionen kann in der folgenden Auflistung gesehen werden: 
\begin{itemize}
    \item Verwende "`PascalCasing"' für Klassen- und Methodennamen
    \item Verwende "`camelCasing"' für Methoden-Argumente und lokale Variablen
    \item Verwende Substantive oder Substantiv-Ausdrücke als Klassennamen.
    \item Interfaces soll ein großes "`I"' vorangestellt werden. Interfacenamen sind Substantive (-Ausdrücke) oder Adjektive.
    \item \ldots
\end{itemize}

Diese halten sich an die offiziellen .NET Codekonventionen von Microsoft~\cite{online:condeConventions}.

\subsection{Projektstruktur}

Bei der Implementierung des Projekts Haben wir einen Modularen Ansatz gewählt. Einzelne Abschnitte, wie beispielsweise die Ruhephase oder die Durchführung von Aufgaben sind in einzelne Unity-Objekte verpackt worden, welche automatisch die für sie relevanten Abläufe durchführen. 

Manche Komponenten müssen auch auf den globalen zustand zugreifen und alle Komponenten müssen Daten erfassen und abspeichern. Um diesen Punkt zu lösen und die Modularität nicht zu zerstören wird das Unity Eventsystem genutzt und als Schnittstelle zwischen einer globalen Manager-Komponente und den einzelnen kleinen Teilen verwendet. 
Diese globale Komponente übernimmt neben dem Gesamtablauf auch die Speicherung der Bewegungsinformationen des Probanden. 

Um festzustellen, ob Studienteilnehmer eher unruhig waren während der Ruhephase wird der Vorwärts-Vektor des HMD sowie die Position dieses innerhalb der Virtuellen Umgebung aufgezeichnet.

Für die Aufgaben werden die folgenden werte erfasst:
\begin{itemize}
    \item Gegebene und erwartete Antwort
    \item Korrektheit der gegebenen Antwort
    \item Zeit der Gegebenen Antwort
\end{itemize}

\subsection{VR-Umgebung}
\todoTob{In-Study umgebung Beschreiben -> Interessant für die Aussagen im Anhang (Partikel etc.)}

Um einen möglichst hohen Entspannungsgrad zu erreichen haben wir uns dazu entschieden die Probanden sowohl visuell mittels VR, als auch auditiv durch Over-Ear-Kopfhörer von der Außenwelt abzuschotten. Dazu implementierten wir eine 360-Grad-Weltraumumgebung. Die Wahl für dieses Szenario trafen wir, da wir die Teilnehmer in eine ruhige, dunkle Umgebung hineinsetzen wollten, welche zeitgleich aber weder unheimlich, noch beklemmend wirkt. Genau deshalb erachteten wir eine Umgebung mit viel Platz auf allen Seiten mit einem ansehnlichen Sternenhimmel, unter welchem sich die Probanden wohl fühlen sollen, als passend. Damit die Umgebung nicht starr und somit unnatürlich wirkte, fügten wir bunte, an Glühwürmchen erinnernde Partikel hinzu, welche langsam aber stetig hoch in den Himmel steigen. 
Auf auditiver Ebene entschieden wir uns für eine sphärische, ruhige, und vor allem durchgehende Musik, welche auch für Meditationen verwendet wird. Dies soll zwei Zwecke erfüllen. Zum einen sollte die Immersion der Weltraumumgebung verstärkt werden und dem Teilnehmer somit das Gefühl vermitteln, dass er sich an einem anderen Ort befindet und zum anderen soll die durchgängige Musik kleinere, störende Geräusche daran hindern, den Probanden abzulenken.
Durch die Synergie aus VR-Umgebung und Musik erhofften wir uns die Probanden zum entspannen beziehungsweise zum schlafen zu bringen.   
\todoAll{Korrekturlesen ob alles passt}


\chapter{Durchführung der Studie}

\section{Zeitplan}
Die Daten beziehen sich auf den Zeitpunkt zu dem der jeweilige Schritt abgeschlossen sein sollte.
\begin{itemize}
    \item \textbf{Dezember 2018} Recherche bestehender Forschung
    \item \textbf{Dezember 2018} Beginnen mit der Implementierung und Modellierung der digitalen Unitiy3D Umgebung %.Digitale Umgebung in Unity3D erstellen und mit erster Testphase validieren
    \item \textbf{Juni 2019} Erste Studie durchführen
    \item \textbf{Juni 2019} Abschluss der Studie und Erweiterung auf Zehn weitere Probanden
    \item \textbf{Juli 2019} Einführung auf eine weitere 15-köpfige Testgruppe, welche mit Ton geweckt wird und Implementierung und Entwurf dieser neuen angepassten Studie
    \item \textbf{August 2019} Abschluss der gesamten Studie
    \item \textbf{September 2019} Auswertung der Ergebnisse
    \item \textbf{Oktober 2019} Erstellen der Ausarbeitung und Präsentation
    
    
    
    
    
    
    \item \textbf{Februar 2019} Auswertung der durchgeführten Studie.
    \item \textbf{März 2019} Entwurf der zweiten Studie
    \item \textbf{April 2019} Erweiterte Umgebung in Unity3D erstellen und für den zweiten Studiendurchlauf vorbereiten
    \item \textbf{Mai 2019} Durchführung der zweiten Studie für erweiterte Ergebnisse
    \item \textbf{Juni 2019} Auswertung der Ergebnisse von zweiter Studie
    \item \textbf{September 2019} Dokumentation der Ergebnisse und des Vorgehens
\end{itemize}

\section{Studiendesign}

\todoTob{das hier raus weil bei Lösungsansatz schon die Umgebung etc alles beschrieben wurde und das ja eher zu Vorarbeit zählt? Oder das aus Lösungsansatz hier rein?}

\section{Studienablauf}

Der Ablauf der Studie kann sich in drei Phasen untergliedern lassen: die Begrüßungsphase, die VR-Phase und die Abschlussphase. Im Folgenden wird der genaue Ablauf einer Durchführung der Studie beschrieben.

\subsubsection{Einleitungsphase}

Auf die allgemeine Begrüßung der Probanden folgte eine kurze Befragung zu den Erfahrungen mit VR und anderen Gaming-Medien. Dies war wichtig um daraufhin eine individuellere Erklärung der von uns verwendeten Geräte zu ermöglichen. 
Nachdem die Teilnehmer über die Ziele der Studie informiert wurden, erklärten wir ebenfalls die Funktionalitäten des Stuhls. Hierbei war es wichtig, dass Teilnehmer über die Einstellungsmöglichkeiten Bescheid wissen und dadurch eine möglichst entspannte Haltung einnehmen können. 
Anschließend wurden die Versuchsteilnehmer mit der VR-Brille vertraut gemacht. Im besonderen, wie sie die Brille für ihre Bedürfnisse entsprechend einzustellen haben und, falls der Proband ein Brillenträger war, was es noch zusätzlich zu beachten gibt.

Zusätzlich zur Einführung wurden den Probanden die üblichen Informationsbögen und eine Einverständniserklärung übergeben und die benötigten Unterschriften eingeholt. 

\begin{figure}[H]
	\centering
	\includegraphics[width=0.6\textwidth]{./images/studie_awf.jpeg}
	\caption{Ein Proband beim Durchführen der Studie.}
	\label{fig:study_setup}
\end{figure}

\subsubsection{VR-Phase}

Die VR-Phase beginnt mir einer einführenden Erklärung der Umgebung und der Bedienung innerhalb dieser. Die Interaktion mit dem Trigger des Controllers lässt dem Nutzer einen Laser als Indikator erscheinen. 
Durch gleichzeitiges Zeigen und Betätigen des Triggers kann mit einzelnen Sphären innerhalb der VR-Umgebung interagiert werden.
Außerdem kann mit bekannten UI-Elementen aus anderen Programmen auf die gleiche Weise interagiert werden.
Diese Mechanik lässt sich vor der Ruhephase ausprobieren und sie stellt die grundlegende Mechanik dar, welche auch in den drei später folgenden Aufgaben von verwendet wird.

Nach dem Bestätigen der Instruktion wurde den Teilnehmern ein erster "`Self Assessment Manikin-Fragebogen"'~\cite{bradley1994measuring} (SAM) gezeigt. Die Auswahlmöglichkeiten für den Nutzer behandeln die Dimensionen Pleasure/Vergnügen, Arousal/Erregung und Dominance/Überlegenheit und die Symbolik ist in Abbildung~\ref{fig:sam_questionnaire} gezeigt.

\todoTob{pleasure und arrousal ist falsch rum}

\begin{figure}
	\centering
	\begin{subfigure}{0.5\textwidth}
		\includegraphics[width=\textwidth]{./images/F1_large.jpg}
		\caption{Self assessment manikin}
		\label{fig:sam_questionnaire}
	\end{subfigure}%
	\hfill
	\begin{subfigure}{0.25\textwidth}
		\includegraphics[width=\textwidth]{./images/rsme.png}
		\caption{Rating scale mental efford}
		\label{fig:rsme_questionnaire}
	\end{subfigure}
	\caption{Beispiele Self assessment manikin und rating scale mental efford.}
\end{figure}

Die Ruhephase beträgt bei jedem Probanden exakt 15 Minuten. Diese Information wurde den Teilnehmern allerdings vor der Studie nicht mitgeteilt.

Jeder Proband wurde innerhalb dieser Phase vom Studienbetreuer subjektiv beobachtet. Hierbei erfassten wir die Informationen zur Stuhleinstellung und ruhiges beziehungsweise unruhiges Verhalten innerhalb der VR-Umgebung. 
Letzteres wurde nach zehn Minuten der Ruhephase wiederholt wodurch zwei subjektive Parameter zur Aktivität des Teilnehmers entstanden. 
Bei der zweiten Aufzeichnung wurde der Fokus auf die Atmung des Versuchsteilnehmers gelegt.

Nach dem 'aufwecken' der Teilnehmer folgten die Aufgaben, welche in Abschnitt~\ref{sec:tasks} beschrieben wurden. Die Reihenfolge der einzelnen Aufgaben variierte hierbei nicht.

Sobald der Proband alle drei Aufgaben absolviert hat, wurde abermals der SAM-Fragebogen präsentiert. 
Zusätzlich hierzu wurde die Einschätzung der eigenen kognitiven Anstrengung mittels "`Rating Scale Mental Effort"' Fragebogen (RSME) abgefragt~\cite{wierwille1983validated}. Die Skala hierzu kann in Abbildung~\ref{fig:rsme_questionnaire} gesehen werden. 

Nach der Beantwortung aller Fragen in der VR-Umgebung endet die VR-Phase und das HMD kann abgenommen werden.

\subsubsection{Abschlussphase}

Nachdem der Hauptteil der Studie komplettiert wurde und das VR-Headset abgenommen werden konnte, mussten die Probanden noch eine demographischen Umfrage bearbeiten, welche diesmal nicht in VR stattfand, sondern an einem Computer im selben Raum. 
Dieser Fragebogen besteht zu Beginn aus allgemeinen, anonymen Fragen zur Person, Alter des Probanden, Geschlecht, gegebenenfalls der Studiengang, sowie ob der Proband schon vor der Studie Erfahrungen mit VR/AR gemacht hat. 
Anschließend musste der Versuchsteilnehmer Fragen zur Studie ausfüllen. Ob sich der Proband vor beziehungsweise nach der Studie müde gefühlt hat oder auch ob ein permanentes Tragen einer VR/AR-Brille vorstellbar wäre.
Nachdem dieser Fragebogen vollständig ausgefüllt wurde, befragten wir dir Teilnehmer noch nach ihrer Einschätzung zur Dauer der Ruhephase und notierten dies für Vergleiche. 
Zusätzlichen waren wir hier noch offen für allgemeine Anmerkungen ehe wir ihnen das Geld überreichten und sie verabschiedeten.

\section{Ergebnisse}

\begin{table*}
	\caption{Numerische Auflistung der Ergebnisse der Frage 'Please enter your age in years'.}~\label{tab:sc_results_gender}
	
	\setlength\tabcolsep{3pt}
	\renewcommand{\arraystretch}{1.4}% for the vertical padding
	\begin{tabularx}{\textwidth}{ | x || x | x | x |}
		\hline
		           & Männlich & Weiblich & Divers \\ \hline\hline
		Absolutwerte   & 33       & 12       & 0                     \\ \hline
		Prozentwerte & 73.3\%   & 26.7\%   & 0\%                   \\ \hline
	\end{tabularx}
\end{table*}

\begin{table*}
	\caption{Numerische Auflistung der Ergebnisse der Frage 'Please select your gender'.}~\label{tab:sc_results_age}
	
	\setlength\tabcolsep{3pt}
	\renewcommand{\arraystretch}{1.4}% for the vertical padding
	\begin{tabularx}{\textwidth}{ | x | x | x | x | x | x | }
		\hline
		Min & Max & Range & Median & Mean  & Standard Deviation \\ \hline\hline
		19  & 30  & 11    & 23     & 23.04 & 2.53              \\ \hline
	\end{tabularx}
\end{table*}


\begin{table*}
	\caption{Verteilung der Antworten zur Frage 'How much experience do you have with VR?'.}~\label{tab:sc_results_age}
	
	\setlength\tabcolsep{3pt}
	\renewcommand{\arraystretch}{1.4}% for the vertical padding
	\begin{tabularx}{\textwidth}{ | x | x | x | x | x | x | x | }
		\hline
		[A1] No experience at all & [A2] Almost no experience & [A3] Less than average experience & [A4] Some experience & [A5] More than average experience  & [A6] Experienced & [A7] Very highly experienced \\ \hline\hline
		10  & 15  & 3    & 10     & 2 & 2 & 3              \\ \hline
		22.2\%  & 33.3\%  & 6.7\%    & 22.2\%     & 4.4\% & 4.4\% & 6.7\%              \\ \hline
	\end{tabularx}
\end{table*}

\begin{table*}
	\caption{Verteilung der Antworten zur Frage 'How much experience do you have with AR?'.}~\label{tab:sc_results_age}
	
	\setlength\tabcolsep{3pt}
	\renewcommand{\arraystretch}{1.4}% for the vertical padding
	\begin{tabularx}{\textwidth}{ | x | x | x | x | x | x | x | }
		\hline
		[A1] No experience at all & [A2] Almost no experience & [A3] Less than average experience & [A4] Some experience & [A5] More than average experience  & [A6] Experienced & [A7] Very highly experienced \\ \hline\hline
		17  & 10  & 7    & 8     & 2 & 1   & 0           \\ \hline
		37.7\%  & 22.2\%  & 15.5\%    & 17.7\%     & 4.4\% & 2.2\%   & 0.0\%           \\ \hline
	\end{tabularx}
\end{table*}

\begin{table*}
	\caption{Verteilung der Antworten zur Frage 'What subject, if any, did you study or are you currently studying?'.}~\label{tab:sc_results_age}
	
	\setlength\tabcolsep{3pt}
	\renewcommand{\arraystretch}{1.4}% for the vertical padding
	\begin{tabularx}{\textwidth}{ | x | x | x | x | x | x | x | x | x | x | }
		\hline
		Biologie & Informatik & Informationssystemtechnik & Mathematik & Medieninformatik & Physik & Psychologie & Software Engineering & Wirtschaftsmathematik & Wirtschaftsphysik \\ \hline\hline
		1  & 8  & 1    & 1     & 18 & 3 & 2 & 8 & 1 & 2           \\ \hline
		2.2\%  & 17.8\%  & 2.2\%    & 2.2\%     & 40\% & 6.7\% & 4.4\% & 17.8\% & 2.2\% & 4.4\%           \\ \hline
	\end{tabularx}
\end{table*}

\todoTob{Mehr Ergebnisse einfügen}
\todoAll{Ergebnisse (wertungsfrei) beschreiben}


\chapter{Diskussion}

\todoAll{Warum sind die Leute nicht eingeschlafen?}
\todoAll{Warum gab es nicht so viele Fehler? (Oder so viele Fehler?)}
\todoAll{Waren die leute zu sehr abgelenkt von VR?}


\chapter{Schlussfolgerung}

\todoAll{Schlussfolgerung Schreiben}
\todoAll{Wann wecken?}
\todoAll{Wie schnell wecken?}
\todoAll{Welche darstellung?}
\todoAll{Wie vorbereiten?}
\todoAll{Auf die Hypothesen eingehen}

Beachtet man die in Kapitel 3 aufgelisteten Hypothesen, so wird klar, dass keine der im voraus definierten Hypothesen komplett bestätigt wird. Zwar lässt sich beim Ordnen der Zahlen tatsächlich ein leichter Trend im Bezug auf Hypothese 2 wahrnehmen, jedoch fällt diese nicht besonders signifikant aus. Ansonsten lässt sich diese Hypothese jedoch nicht anhand er gesammelten Werte bestätigen. Hypothese 1 wird ebenfalls nicht durch die Werte bestätigt, da keine gravierenden Tendenzen zu beobachten sind. Beim Colour-Matching geht der Trend sogar eher in die Richtung, dass Menschen die mit einem akustischen Signal geweckt werden tendenziell eher etwas schneller sind und auch weniger Fehler machen als Probanden, welche einem Lichtreiz zum Wecken ausgesetzt wurden.

Auch in Bezug auf Hypothese 3 und 4 lässt sich nicht herauslesen, dass eine der Beiden Gruppen, welche mit Licht geweckt wurden bedeutsam besser oder schlechter als die andere Gruppe abgeschnitten hat. Den einzigen auszumachenden, wenn jedoch nicht besonders signifikanten Unterschied erkennt man beim Ordnen der Zahlen. Hier zeigen die Daten, dass die Fade5 Gruppe deutlichere Extrema im Bezug auf die Geschwindigkeit aufwiesen. Sowohl der schnellste als auch der langsamste gehörten dieser Gruppe an. Die Fade20 Gruppe hingegen bewegt sich deutlicher im Mittelfeld der Grafik ~\ref{fig:orderingMistakeTimeScatterplot}.

\todoAll{Korrekturlesen}

\todos

\appendix		% Ab hier Appendices einbinden
\chapter{Anhang 1}

\begin{table*}
	\caption{Numerische Auflistung der Ergebnisse der Frage "`Please select your gender"'.}~\label{tab:sc_results_gender}
	
	\setlength\tabcolsep{3pt}
	\renewcommand{\arraystretch}{1.4}% for the vertical padding
	\begin{tabularx}{\textwidth}{ | x || r | r | }
		\hline
		Geschlecht & Absolutwerte 	& Prozentwerte \\ \hline\hline
		Männlich & 33 & 73.3\% \\ \hline
		Weiblich & 12 & 26.7\% \\ \hline
		Divers & 0 & 0.0\% \\ \hline
	\end{tabularx}
\end{table*}

\begin{table*}
	\caption{Numerische Auflistung der Ergebnisse der Frage "`Please enter your age in years"'.}~\label{tab:sc_results_age}
	
	\setlength\tabcolsep{3pt}
	\renewcommand{\arraystretch}{1.4}% for the vertical padding
	\begin{tabularx}{\textwidth}{ | x | x | x | x | x | x | }
		\hline
		Min & Max & Range & Median & Mean  & Standard Deviation \\ \hline\hline
		19  & 30  & 11    & 23     & 23.04 & 2.53              \\ \hline
	\end{tabularx}
\end{table*}

\begin{table*}
	\caption{Verteilung der Antworten zur Frage "`How much experience do you have with VR?"'.}~\label{tab:sc_results_expVR}
	
	\setlength\tabcolsep{3pt}
	\renewcommand{\arraystretch}{1.4}% for the vertical padding
	\begin{tabularx}{\textwidth}{ | x || r | r | }
		\hline
		Studienfach 						& Absolutwerte 	& Prozentwerte \\ \hline\hline
		[A1] No experience at all 			& 10 			& 22.2\% \\ \hline
		[A2] Almost no experience 			& 15 			& 33.3\% \\ \hline
		[A3] Less than average experience 	& 3 			& 6.7\% \\ \hline
		[A4] Some experience 				& 10 			& 22.2\% \\ \hline
		[A5] More than average experience 	& 2 			& 4.4\% \\ \hline
		[A6] Experienced 					& 2 			& 4.4\% \\ \hline
		[A7] Very highly experienced 		& 3 			& 6.7\% \\ \hline
	\end{tabularx}
\end{table*}

\begin{table*}
	\caption{Numerische Auflistung der Ergebnisse der Frage "`How much experience do you have with VR?"'.}~\label{tab:sc_numbers_expVR}
	
	\setlength\tabcolsep{3pt}
	\renewcommand{\arraystretch}{1.4}% for the vertical padding
	\begin{tabularx}{\textwidth}{ | x | x | x | x | x | x | }
		\hline
		Min & Max & Range & Median & Mean  & Standard Deviation \\ \hline\hline
		1  & 7  & 6    & 2     & 2.93 & 1.78              \\ \hline
	\end{tabularx}
\end{table*}

\begin{table*}
	\caption{Verteilung der Antworten zur Frage "`How much experience do you have with AR?"'.}~\label{tab:sc_results_expAR}
	
	\setlength\tabcolsep{3pt}
	\renewcommand{\arraystretch}{1.4}% for the vertical padding
	\begin{tabularx}{\textwidth}{ | x || r | r | }
		\hline
		Studienfach 						& Absolutwerte 	& Prozentwerte \\ \hline\hline
		[A1] No experience at all 			& 17 			& 37.7\% \\ \hline
		[A2] Almost no experience 			& 10 			& 22.2\% \\ \hline
		[A3] Less than average experience 	& 7 			& 15.5\% \\ \hline
		[A4] Some experience 				& 8 			& 17.7\% \\ \hline
		[A5] More than average experience 	& 2 			& 4.4\% \\ \hline
		[A6] Experienced 					& 1 			& 2.2\% \\ \hline
		[A7] Very highly experienced 		& 0 			& 0.0\% \\ \hline
	\end{tabularx}
\end{table*}

\begin{table*}
	\caption{Numerische Auflistung der Ergebnisse der Frage "`How much experience do you have with AR?"'.}~\label{tab:sc_numbers_expAR}
	
	\setlength\tabcolsep{3pt}
	\renewcommand{\arraystretch}{1.4}% for the vertical padding
	\begin{tabularx}{\textwidth}{ | x | x | x | x | x | x | }
		\hline
		Min & Max & Range & Median & Mean  & Standard Deviation \\ \hline\hline
		1  & 6  & 5    & 2     & 2.36 & 1.38              \\ \hline
	\end{tabularx}
\end{table*}

\begin{table*}
	\caption{Verteilung der Antworten zur Frage "`What subject, if any, did you study or are you currently studying?"'.}~\label{tab:sc_results_study}
	
	\setlength\tabcolsep{3pt}
	\renewcommand{\arraystretch}{1.4}% for the vertical padding
	\begin{tabularx}{\textwidth}{ | x || r | r | }
		\hline
		Studienfach & Absolutwerte & Prozentwerte \\ \hline\hline
		Biologie & 1 & 2.2\% \\ \hline
		Informatik & 8 & 17.8\% \\ \hline
		Informationssystemtechnik & 1 & 2.2\% \\ \hline
		Mathematik & 1 & 2.2\% \\ \hline
		Medieninformatik & 18 & 40.0\% \\ \hline
		Physik & 3 & 6.7\% \\ \hline
		Psychologie & 2 & 4.4\% \\ \hline
		Software Engineering & 8 & 17.8\% \\ \hline
		Wirtschaftsmathematik & 1 & 2.2\% \\ \hline
		Wirtschaftsphysik & 2 & 4.4\% \\ \hline
	\end{tabularx}
\end{table*}

\begin{table*}
	\caption{Verteilung der Einstellungen des Stuhls.}~\label{tab:sc_results_chair}
	
	\setlength\tabcolsep{3pt}
	\renewcommand{\arraystretch}{1.4}% for the vertical padding
	\begin{tabularx}{\textwidth}{ | x || r | r | }
		\hline
		Winkeleinstellungen	in Grad	& Absolutwerte 	& Prozentwerte \\ \hline\hline
		0 							& 7 			& 15.6\% \\ \hline
		30 							& 23			& 51.1\% \\ \hline
		60	 						& 10 			& 22.2\% \\ \hline
		90							& 5 			& 11.1\% \\ \hline
	\end{tabularx}
\end{table*}

\begin{itemize}
	\captionof{anno}{Anmerkungen und Hinweise von Studienteilnehmern}
	\item "`Die Musik war sehr störend, um in einen Ruhezustand zu kommen"'
	\item "`Die VR Umgebung war schön gestaltet, aber die rumschwebenden Partikel waren eher verwirrend, ich dachte ich kann mit diesen interagieren"'
	\item "`Der Stuhl war sehr entspannend und bequem"'
	\item "`Es fiel mir schwer einzuschlafen, da ich zum 1. mal VR gemacht habe und dann neugierig war"'
	\item "`Die Musik war sehr angenehm"'
	\item "`Das lange gedrückt halten zur Interaktion war störend"'
	\item "`haptisches Feedback durch Controller wäre gut gewesen"'
	\item "`Die Brille war sehr unangenehm"'
	\item "`Der Ton fürs Wecken hat mich erschrocken"'
	\item "`Mit meiner Brille war es unangenehm die VR Brille zu tragen"'
	\item "`Ich konnte mich sehr gut entspannen, richtig eingeschlafen bin ich      aber nicht"'
	\item "`Die Interaktion mit dem Controller war sehr intuitiv"'
	\item "`Mir kam die Zeit zum entspannen deutlich länger als 15 Minuten vor "'
	\item "`Egal wie ich die Brille verstellte, richtig scharf konnte ich nie sehen "'
	\item "`Noch fünf bis zehn Minuten länger und ich wäre komplett eingeschlafen "'

\end{itemize}

\begin{table*}
	\caption{Wahrgenommene Schlafdauer.}~\label{tab:sleepduration}
	
	\setlength\tabcolsep{3pt}
	\renewcommand{\arraystretch}{1.4}% for the vertical padding
	\begin{tabularx}{\textwidth}{ | x || r | r | }
		\hline
		wahrgenommene Schlafdauer in min & Absolutwerte & Prozentwerte \\ \hline\hline
		8						   	     & 2			   & 4.4\% \\ \hline
		10   					         & 5			   & 11.1\% \\ \hline
		11						   	     & 1 		   & 2.2\% \\ \hline
		12						   	     & 3			   & 6.7\% \\ \hline
		13							     & 2			   & 4.4\% \\ \hline
		14							     & 1			   & 2.2\% \\ \hline
		10-15	      					 & 3		 & 6.7\% \\ \hline
		15							     & 13		 & 28.9\% \\ \hline
		15-20							 & 1		 & 2.2\% \\ \hline
		17								 & 2		 & 4.4\% \\ \hline
		18								 & 3		 & 6.7\% \\ \hline
		18,5							 & 1		 & 2.2\% \\ \hline
		19								 & 1		 & 2.2\% \\ \hline
		20								 & 6		 & 13.3\% \\ \hline
		30								 & 1		 & 2.2\% \\ \hline
	\end{tabularx}
\end{table*}
\todoAll{Als Idee: Diese Tabelle etwas kürzen und zusammenfassen. Also z.B. Zeitintervalle [5-10), [10-15), [15], (15-20], (20-25], (25-30]. Dennis Fragen!}

\begin{table*}
	\caption{Verteilung der Antworten zur Frage "`Hast du geschlafen?"' .}~\label{tab:sleepstatus}
	
	\setlength\tabcolsep{3pt}
	\renewcommand{\arraystretch}{1.4}% for the vertical padding
	\begin{tabularx}{\textwidth}{ | x || r | r | }
		\hline
		Schlafmodus					& Absolutwerte 	& Prozentwerte \\ \hline\hline
		geschlafen 					& 9 			& 20.0\% \\ \hline
		gedöst/kurz vor eingeschlafen	& 12			& 26.7\% \\ \hline
		meditiert					& 3			& 6.7\% \\ \hline
		nicht geschlafen			& 21 			& 46.7\% \\ \hline
	\end{tabularx}
\end{table*}


\backmatter %%%%%%%%%%%%%%%%%%%%%%%%%%%%%%%%%%%%%%%%%%%%%%%%%%%%%%%%%%%%%%%%%%

\listoffigures	% Abbildungsverzeichnis
\listoftables	% Tabellenverzeichnis

%\bibliographystyle{natdin}
\bibliographystyle{IEEEtranS}	% alternativer Stil
\bibliography{MCI-Resync-Documentation}

\cleardoublepage
\clearscrheadfoot
% \declaration		% Erklärung, siehe diplom-mi-eng.cls

\end{document}
